\documentclass[12pt,a4paper,twoside,openright,BCOR10mm,headsepline,titlepage,abstracton,chapterprefix,final]{scrreprt}

\usepackage{ae}
\usepackage[ngerman, english]{babel}
%\usepackage{SIunits}

\usepackage{amsmath}
\usepackage{amssymb}
\usepackage{amsfonts}
\usepackage{xcolor}
\usepackage{setspace}

% load hyperref as the last package to avoid redefinitions of e.g. footnotes after hyperref invokation

\RequirePackage{ifpdf}  % flag for pdf or dvi backend
\ifpdf
  \usepackage[pdftex]{graphicx}
  \usepackage[pdftitle={RTFM on Imaging Theory and Basics of Optical Raytracing},%
              pdfsubject={},%
              pdfauthor={M. Esslinger, J. Hartung, U. Lippmann},%
              pdfkeywords={},%
              bookmarks=true,%
%              colorlinks=true,%
              urlcolor=blue,%
              pdfpagelayout=TwoColumnRight,%
              pdfpagemode=UseNone,%
              pdfstartview=Fit,%
	      pdfpagelabels,
              pdftex]{hyperref}
\else
  \usepackage[dvips]{graphicx}
  \usepackage[colorlinks=false,dvips]{hyperref}
\fi
%\DeclareGraphicsRule{.jpg}{eps}{.jpg}{`convert #1 eps:-}

\usepackage{ae}
%\usepackage[ngerman, english]{babel}

%\usepackage{SIunits}
\newcommand\elementarycharge{\textrm{e}}
\newcommand\sccm{\textrm{sccm}}
\newcommand\mbar{\milli\textrm{bar}}


\usepackage{amsmath}
%\usepackage{amssymb}
\usepackage{setspace}

%\widowpenalty = 1000


\newcommand*{\doi}[1]{\href{http://dx.doi.org/\detokenize{#1}}{doi: \detokenize{#1}}}

\newcommand\Vector[1]{{\mathbf{#1}}}

\newcommand\vacuum{0}

\newcommand\location{r}
\newcommand\Location{\Vector{r}}


\newcommand\wavenumber{k}
\newcommand\vacuumWavenumber{\wavenumber_{\vacuum}}
\newcommand\Wavevector{\Vector{\wavenumber}}

\newcommand\Nabla{\Vector{\nabla}}
\newcommand\Laplace{\Delta}
\newcommand\timederivative[1]{\dot{{#1}}}
\newcommand\Tensor[1]{\hat{#1}}
\newcommand\conjugate[1]{\overline{#1}}
\newcommand\transpose[1]{#1^{T}}
\newcommand\Norm[1]{\left| #1 \right|}
\newcommand{\ket}[1]{\left\vert{#1}\right\rangle}
\newcommand{\bra}[1]{\left\langle{#1}\right\vert}
\newcommand{\braket}[2]{\left\langle{#1}\vert{#2}\right\rangle}
\newcommand{\bracket}[1]{\left\langle{#1}\right\rangle}

\newcommand{\scpm}[2]{(#1\,\cdot\,#2)}

\newcommand\Greenfunction{\Tensor{G}}

\newcommand\scalarEfield{E}
\newcommand\scalarBfield{B}
\newcommand\scalarHfield{H}
\newcommand\scalarDfield{D}
\newcommand\scalarTipfield{T}
\newcommand\scalarSamplefield{S}
\newcommand\scalarDipolarmoment{p}
\newcommand\Efield{\Vector{\scalarEfield}}
\newcommand\Bfield{\Vector{\scalarBfield}}
\newcommand\Hfield{\Vector{\scalarHfield}}
\newcommand\Dfield{\Vector{\scalarDfield}}
\newcommand\Dipolarmoment{\Vector{\scalarDipolarmoment}}

\newcommand\permeability{\Tensor{\mu}}
\newcommand\vacuumpermeability{\mu_{\vacuum}}
\newcommand\permittivity{\Tensor{\epsilon}}
\newcommand\generalPermittivity{\Tensor{\tilde\epsilon}}
\newcommand\vacuumpermittivity{\epsilon_{\vacuum}}
\newcommand\scalarpermittivity{\epsilon}
\newcommand\conductivity{\Tensor{\sigma}}
\newcommand\susceptibility{\Tensor{\chi}}
\newcommand\currentdensity{\Vector{j}}
\newcommand\Current{\Vector{I}}
\newcommand\chargedensity{\rho}
\newcommand\PoyntingVector{\Vector{S}}

\newcommand{\remark}[1]{{\color{red}$\blacksquare$}\footnote{{\color{red}#1}}}

\newif\ifdraft
\draftfalse % \drafttrue




\begin{document}

\section{Pilot Ray Concept}

\subsection{Motivation}

In systems like Offner systems or off-axis parabolic mirrors, the center-of-field chief ray is no more aligned with the axis of rotational symmetry of the surfaces.
In these systems, it is no longer sufficient to obtain first order properties from the central curvature on the axis of symmetry of each surface.
In other systems, like freeshape Schiefspiegler telescopes, there is no axis of symmetry at all.

In this section, we develop a description for the first order properties of optical systems that can describe rotationally symmetric and non-symmetric systems with a single formalism.
It is a formalism for rays close to the center-of-field chief ray, which we call \emph{pilot ray} in the following.
This pilot ray takes the role of the optical axis in rotationally symmetric systems. 
By choosing the term pilot ray, we want to emphasize that we mean neither a mechanical axis of symmetry of individual lens elements nor the extraordinary axis in anisotropic crystals.
In most systems, the intial pilot ray direction is along the $z$ axis. 
An exception is found, for example, in Scheimpflug systems,

\subsection{Near-Pilot Rays}
Our ansatz is to perform a real ray trace on the pilot ray and describe a small area around it by a Taylor expansion around the real ray intersection point.
From this Taylor expansion, we derive first order properties, like focal length, magnification and pupil positions.
These properties are used to aim all other rays into the pupil.
The concept is a parabasal (paraxial) approximation around a non-paraxial pilot ray.

We consider near-pilot rays, that have an intersection point and a wavevector close to the pilot ray.

\begin{tabular}{ l | l | l }
		    & pilot ray & near-pilot ray \\
& & \\ \hline & & \\
intersection point  & $\Vector{r}$ & $\tilde{\Vector{r}} = \Vector{r} + \Delta\Vector{r}$ \\
initial wavevector  & $\Wavevector$ & $\tilde{\Wavevector} = \Wavevector + \Delta\Wavevector$ \\
surface normal      & $\Vector{n} = \Vector{n}(\Vector{r})$ & $\tilde{\Vector{n}}(\tilde{\Vector{r}}) = \Vector{n} + \Delta\Vector{n}$ \\
& & \\ \hline & & \\
out-of-plane wavevector component &  $\Wavevector_{\perp} = (\Wavevector \Vector{n})\Vector{n}$ & $\tilde{\Wavevector}_{\perp} = (\tilde{\Wavevector} \tilde{\Vector{n}})\tilde{\Vector{n}}$ \\
in-plane wavevector component     &  $\Wavevector_{\parallel} = \Wavevector - \Wavevector_{\perp}$ & $\tilde{\Wavevector}_{\parallel} = \tilde{\Wavevector} - \tilde{\Wavevector}_{\perp}$ \\
& & \\ \hline & & \\
wavevector after refraction       &  $\Wavevector_2 = \Wavevector_{\parallel} + \xi \Vector{n}$ & $\tilde{\Wavevector}_2 = \tilde{\Wavevector}_{\parallel} + \tilde{\xi} \tilde{\Vector{n}}$
\end{tabular}
\\[2ex]
The refraction conserves the in-plane wavevector component. 
The out-of-plane or normal wavevector component $\xi, \tilde{\xi}$ after the refraction depends on the material permittivity and the in-plane component.

We are interested in the change of the refracted wavevector $\Delta\Wavevector_2 = \tilde{\Wavevector}_2 - \Wavevector_2$ induced by $\Delta\Vector{r}$ and $\Delta\Wavevector$.

\begin{eqnarray}
 \Delta\Wavevector_2 &=& \tilde{\Wavevector}_2 - \Wavevector_2 \\
 &=& ( \tilde{\Wavevector}_{\parallel} - \Wavevector_{\parallel} ) + ( \tilde{\xi} \tilde{\Vector{n}} - \xi \Vector{n} ) \\
% &=& ( \tilde{\Wavevector} - \Wavevector)  - (\tilde{\Wavevector}_{\perp} - \Wavevector_{\perp}) + ( \tilde{\xi} - \xi ) \Vector{n} + \tilde{\xi} \Delta\Vector{n} \\
% &=& \Delta\Wavevector - (\tilde{\Wavevector}_{\perp} - \Wavevector_{\perp}) + ( \tilde{\xi} - \xi ) \Vector{n} + \tilde{\xi} \Delta\Vector{n} \\
% &=& \Delta\Wavevector - ((\tilde{\Wavevector} \tilde{\Vector{n}})\tilde{\Vector{n}} - (\Wavevector \Vector{n})\Vector{n}) + ( \tilde{\xi} - \xi ) \Vector{n} + \tilde{\xi} \Delta\Vector{n} \\
% &=& \Delta\Wavevector - 
% \left(
%   (\tilde{\Wavevector} \tilde{\Vector{n}})\Vector{n}+(\tilde{\Wavevector} \tilde{\Vector{n}})\Delta\Vector{n} - (\Wavevector \Vector{n})\Vector{n}
% \right) 
% + ( \tilde{\xi} - \xi ) \Vector{n} + \tilde{\xi} \Delta\Vector{n} \\
% &=& \Delta\Wavevector - 
% \left(
%   (\tilde{\Wavevector} (\Vector{n} + \Delta\Vector{n}) )\Vector{n}+(\tilde{\Wavevector} (\Vector{n} + \Delta\Vector{n}) )\Delta\Vector{n} - (\Wavevector \Vector{n})\Vector{n}
% \right) 
% + ( \tilde{\xi} - \xi ) \Vector{n} + \tilde{\xi} \Delta\Vector{n} \\
% &=& \Delta\Wavevector - 
% \left(
%   (\tilde{\Wavevector} \Vector{n} )\Vector{n} + (\tilde{\Wavevector} \Delta\Vector{n} )\Vector{n} + (\tilde{\Wavevector} \Vector{n} )\Delta\Vector{n} + (\tilde{\Wavevector} \Delta\Vector{n} )\Delta\Vector{n} - (\Wavevector \Vector{n})\Vector{n}
% \right) 
% + ( \tilde{\xi} - \xi ) \Vector{n} + \tilde{\xi} \Delta\Vector{n} \\
% &=& \Delta\Wavevector - 
% \left(
%   (\Delta\Wavevector \Vector{n} )\Vector{n} + (\tilde{\Wavevector} \Delta\Vector{n} )\Vector{n} + (\tilde{\Wavevector} \Vector{n} )\Delta\Vector{n} + (\tilde{\Wavevector} \Delta\Vector{n} )\Delta\Vector{n}
% \right) 
% + ( \tilde{\xi} - \xi ) \Vector{n} + \tilde{\xi} \Delta\Vector{n} \\
 &=& \Delta\Wavevector - 
 \bigg(
   (\Delta\Wavevector \Vector{n} )\Vector{n} 
   + (\Wavevector \Delta\Vector{n} )\Vector{n}  + (\Delta\Wavevector \Delta\Vector{n} )\Vector{n} 
   + (\Wavevector \Vector{n} )\Delta\Vector{n}  + 
  \nonumber \\ &&
   + (\Delta\Wavevector \Vector{n} )\Delta\Vector{n} 
   + (\Wavevector \Delta\Vector{n} )\Delta\Vector{n}
   + (\Delta\Wavevector \Delta\Vector{n} )\Delta\Vector{n}
 \bigg) 
 + ( \tilde{\xi} - \xi ) \Vector{n} + \tilde{\xi} \Delta\Vector{n}
\end{eqnarray}
We are insterested in the limit $\Delta\Wavevector, \Delta\Vector{r} \rightarrow 0$. We approximate $\Delta\Vector{n}$ in first order
\begin{eqnarray}
 \Delta n_i \approx \left. \frac{\partial n_i}{\partial r_j} \right|_{\Vector{r}} \Delta r_j
\end{eqnarray}
to be proportional to $\Delta\Vector{r}$. 
We neglect all higher order terms $\Delta\Wavevector^2$ and $\Delta\Vector{r}^2$.
Like the ABCD-matrix formalism, 
\begin{eqnarray}
 y_2 &=& A y_1 + B y^\prime_1 \\
 y^\prime_2 &=& C y_1 + D y^\prime_1
\end{eqnarray}
we also neglect cross-terms $\Delta\Wavevector\Delta\Vector{r}$. 
This leaves the lowest order terms
\begin{eqnarray}
% \Delta\Wavevector_2 &\approx& 
%   \Delta\Wavevector 
%   - (\Delta\Wavevector \Vector{n} )\Vector{n} 
%   - (\Wavevector \Delta\Vector{n} )\Vector{n} 
%   - (\Wavevector \Vector{n} )\Delta\Vector{n}   
%   + ( \tilde{\xi} - \xi ) \Vector{n} 
%   + \tilde{\xi} \Delta\Vector{n} 
% \\
 \Delta \wavenumber_{2,i} &\approx& 
   \left( \delta_{ij}  - n_i n_j \right) \Delta\wavenumber_j
   +
   \left(
     - \wavenumber_j  n_i 
     - \wavenumber_m n_m \delta_{ij}  
     + \xi \delta_{ij}
   \right) \Delta n_j
   + ( n_i ) \Delta\xi 
\end{eqnarray}
The explicit form of $\Delta\xi = \tilde{\xi} - \xi$ can be found once the material dispersion is specified.

\subsection{Isotropic Media}
We assume a scalar, complex valued permittivity $\scalarpermittivity_2$.
\begin{eqnarray}
 \xi &=& \sqrt{\omega^2 \vacuumpermeability \scalarpermittivity_2 - \Wavevector_{\parallel} \cdot \Wavevector_{\parallel}} \\
 \Delta \xi &=& \sqrt{\omega^2 \vacuumpermeability \scalarpermittivity_2 - \tilde{\Wavevector}_{\parallel} \cdot \tilde{\Wavevector}_{\parallel}} - \sqrt{\omega^2 \vacuumpermeability \scalarpermittivity_2 - \Wavevector_{\parallel} \cdot \Wavevector_{\parallel}} \\ 
 \tilde{\Wavevector}_{\parallel} &\approx& \Wavevector_\parallel + \Delta\Wavevector - (\Delta\Wavevector \Vector{n})\Vector{n} - (\Wavevector\Vector{n})\Delta\Vector{n} - (\Wavevector \Delta\Vector{n})\Vector{n} \\
 \tilde{\wavenumber}_{\parallel, i} &\approx& \wavenumber_{\parallel,i} + \Delta\wavenumber_i - (\Delta\wavenumber_j n_j) n_i - (\wavenumber_j n_j) \Delta n_i - (\wavenumber_j \Delta n_j) n_i \\
 \tilde{\wavenumber}_{\parallel, i} \cdot \tilde{\wavenumber}_{\parallel, i} 
   &\approx& \wavenumber_{\parallel,i} \cdot \wavenumber_{\parallel,i} + 
     2 \wavenumber_{\parallel,i} ( \Delta\wavenumber_i - (\Delta\wavenumber_j n_j) n_i - (\wavenumber_j n_j) \Delta n_i - (\wavenumber_j \Delta n_j) n_i )
 \\
 \tilde{\wavenumber}_{\parallel, i} \cdot \tilde{\wavenumber}_{\parallel, i} &\approx& \wavenumber_{\parallel,i} \cdot \wavenumber_{\parallel,i} +  2 \mu \\
 \sqrt{x- 2 \mu} &\approx& \sqrt{x} + \left. \frac{\partial \sqrt{x-2\mu}}{\partial \mu} \right|_{0} \mu = \sqrt{x} - \frac{\mu}{\sqrt{x}} \\
 \Delta \xi &\approx& \sqrt{\omega^2 \vacuumpermeability \scalarpermittivity_2 - \Wavevector_{\parallel} \cdot \Wavevector_{\parallel} - 2 \mu } - \sqrt{\omega^2 \vacuumpermeability \scalarpermittivity_2 - \Wavevector_{\parallel} \cdot \Wavevector_{\parallel}} \\ 
 \Delta \xi &\approx& - \frac{\wavenumber_{\parallel,i}}{\xi} ( \Delta\wavenumber_i - (\Delta\wavenumber_j n_j) n_i - (\wavenumber_j n_j) \Delta n_i - (\wavenumber_j \Delta n_j) n_i )
\end{eqnarray}


\end{document}
