\documentclass[12pt,a4paper,twoside,openright,BCOR10mm,headsepline,titlepage,abstracton,chapterprefix,final]{scrreprt}

\usepackage{ae}
\usepackage[ngerman, english]{babel}
%\usepackage{SIunits}

\usepackage{amsmath}
\usepackage{amssymb}
\usepackage{amsfonts}
\usepackage{xcolor}
\usepackage{setspace}

% load hyperref as the last package to avoid redefinitions of e.g. footnotes after hyperref invokation

\RequirePackage{ifpdf}  % flag for pdf or dvi backend
\ifpdf
  \usepackage[pdftex]{graphicx}
  \usepackage[pdftitle={RTFM on Imaging Theory and Basics of Optical Raytracing},%
              pdfsubject={},%
              pdfauthor={M. Esslinger, J. Hartung, U. Lippmann},%
              pdfkeywords={},%
              bookmarks=true,%
%              colorlinks=true,%
              urlcolor=blue,%
              pdfpagelayout=TwoColumnRight,%
              pdfpagemode=UseNone,%
              pdfstartview=Fit,%
	      pdfpagelabels,
              pdftex]{hyperref}
\else
  \usepackage[dvips]{graphicx}
  \usepackage[colorlinks=false,dvips]{hyperref}
\fi
%\DeclareGraphicsRule{.jpg}{eps}{.jpg}{`convert #1 eps:-}

\usepackage{ae}
%\usepackage[ngerman, english]{babel}

%\usepackage{SIunits}
\newcommand\elementarycharge{\textrm{e}}
\newcommand\sccm{\textrm{sccm}}
\newcommand\mbar{\milli\textrm{bar}}


\usepackage{amsmath}
%\usepackage{amssymb}
\usepackage{setspace}

%\widowpenalty = 1000


\newcommand*{\doi}[1]{\href{http://dx.doi.org/\detokenize{#1}}{doi: \detokenize{#1}}}

\newcommand\Vector[1]{{\mathbf{#1}}}

\newcommand\vacuum{0}

\newcommand\location{r}
\newcommand\Location{\Vector{r}}


\newcommand\wavenumber{k}
\newcommand\vacuumWavenumber{\wavenumber_{\vacuum}}
\newcommand\Wavevector{\Vector{\wavenumber}}

\newcommand\Nabla{\Vector{\nabla}}
\newcommand\Laplace{\Delta}
\newcommand\timederivative[1]{\dot{{#1}}}
\newcommand\Tensor[1]{\hat{#1}}
\newcommand\conjugate[1]{\overline{#1}}
\newcommand\transpose[1]{#1^{T}}
\newcommand\Norm[1]{\left| #1 \right|}
\newcommand{\ket}[1]{\left\vert{#1}\right\rangle}
\newcommand{\bra}[1]{\left\langle{#1}\right\vert}
\newcommand{\braket}[2]{\left\langle{#1}\vert{#2}\right\rangle}
\newcommand{\bracket}[1]{\left\langle{#1}\right\rangle}

\newcommand{\scpm}[2]{(#1\,\cdot\,#2)}

\newcommand\Greenfunction{\Tensor{G}}

\newcommand\scalarEfield{E}
\newcommand\scalarBfield{B}
\newcommand\scalarHfield{H}
\newcommand\scalarDfield{D}
\newcommand\scalarTipfield{T}
\newcommand\scalarSamplefield{S}
\newcommand\scalarDipolarmoment{p}
\newcommand\Efield{\Vector{\scalarEfield}}
\newcommand\Bfield{\Vector{\scalarBfield}}
\newcommand\Hfield{\Vector{\scalarHfield}}
\newcommand\Dfield{\Vector{\scalarDfield}}
\newcommand\Dipolarmoment{\Vector{\scalarDipolarmoment}}

\newcommand\permeability{\Tensor{\mu}}
\newcommand\vacuumpermeability{\mu_{\vacuum}}
\newcommand\permittivity{\Tensor{\epsilon}}
\newcommand\generalPermittivity{\Tensor{\tilde\epsilon}}
\newcommand\vacuumpermittivity{\epsilon_{\vacuum}}
\newcommand\scalarpermittivity{\epsilon}
\newcommand\conductivity{\Tensor{\sigma}}
\newcommand\susceptibility{\Tensor{\chi}}
\newcommand\currentdensity{\Vector{j}}
\newcommand\Current{\Vector{I}}
\newcommand\chargedensity{\rho}
\newcommand\PoyntingVector{\Vector{S}}

\newcommand\ordi{\text{ord}}
\newcommand\eo{\text{eo}}

\newcommand{\timeavg}[1]{{\langle\,#1\,\rangle}}

\newcommand{\remark}[1]{{\color{red}$\blacksquare$}\footnote{{\color{red}#1}}}
\newcommand{\chk}[1]{\color{green}{$\checkmark$#1}}


\newif\ifdraft
\draftfalse % \drafttrue




\begin{document}

Such equations are in principle analytically solvable. But on the one hand it is very complicated and on the other hand 
it is not known whether there exists such a user-friendly numerically stable form of $t$ like in the former cases.\footnote{
In fact the word ``biconic" is misleading because it suggests that
it is a surface of second degree which is wrong. \eqref{eq:implicitbiconic}
shows clearly that it is a surface of degree four.}

Let us first assume that $\Vector{r} = \Vector{r}_0 + t \Vector{d}$ is the insertion of the ray into
\eqref{eq:implicitbiconic}. Then $a(x(t), y(t)) =: Q_2(t)$ and $b(x(t), y(t)) =: Q_1(t)$ become quadratic polynomials 
in $t$. Further $z(t) =: L(t)$ becomes linear in $t$. Therefore this leads to
\begin{align}
 (Q_2(t) - 1) \underbrace{L^2(t)}_{\ge0} + \underbrace{(L(t) - Q_1(t))^2}_{\ge0} &= 0\,.\label{eq:biconict}
\end{align}
From \eqref{eq:biconic} and \eqref{eq:implicitbiconic} it follows that $1 - Q_2(t) \ge 0$ such that the
prefactor in \eqref{eq:biconict} is non-positive. Therefore the intersection equations lead to a real
fourth order polynomial which has to be solved. The quadratic polynomial $Q_2(t) - 1$ can be factorized
\begin{align}
 Q_2(t) - 1 &= A_2 (t - t_{Q21})(t - t_{Q22}) \le 0\,,
\end{align}
where $A_2 = (1 + c_x) \rho_x^2 d_x^2 + (1 + c_y) \rho_y^2 d_y^2$. If $A_2 > 0$ then at least one zero must exist and
$t \in [t_{Q21}, t_{Q22}]$. If $A_2 = 0$ then $Q_2(t)$ is not quadratic anymore. If $A_2 < 0$ then in 
\begin{align}
 Q_2(t) - 1 &= A_2 \left(\left(t-\frac{t_{Q21} + t_{Q22}}{2}\right)^2 - \frac{1}{4}(t_{Q21} - t_{Q22})^2\right)\,,
\end{align}
the expression in brackets has to be positive or zero. This is only true if the zeros are either the same or they are complex, because
for $t_{Q21/2} = \tau \pm i \sigma$ this expression becomes
\begin{align}
 Q_2(t) - 1 &= A_2 \left(\left(t-\tau\right)^2 + \frac{1}{2}\sigma^2\right)<0\,,
\end{align}
for real $t$. However the final equation is given by
\begin{align}
 a t^4 + b t^3 + c t^2 + d t + e &= 0\,,
\end{align}
where\remark{Simplify coefficients!}
\begin{subequations}
 \begin{align}
  a &= c_{x}^{2} d_{x}^{4} + 2 \, c_{x} c_{y} d_{x}^{2} d_{y}^{2} + c_{y}^{2} d_{y}^{4} \nonumber\\&
       + {\left(c_{x}^{2} k_{x} + c_{x}^{2}\right)} d_{x}^{2} d_{z}^{2} + {\left(c_{y}^{2} k_{y} + c_{y}^{2}\right)} d_{y}^{2} d_{z}^{2}\,,\\
  b &= 4 \, c_{x}^{2} d_{x}^{3} x_{0} + 4 \, c_{x} c_{y} d_{x} d_{y}^{2} x_{0} + 4 \, c_{x} c_{y} d_{x}^{2} d_{y} y_{0} + 4 \, c_{y}^{2} d_{y}^{3} y_{0}\nonumber\\&
       + 2 \, {\left(c_{x}^{2} k_{x} + c_{x}^{2}\right)} d_{x} d_{z}^{2} x_{0} + 2 \, {\left(c_{y}^{2} k_{y} + c_{y}^{2}\right)} d_{y} d_{z}^{2} y_{0}\nonumber\\&
       + 2 \, {\left(c_{x}^{2} k_{x} + c_{x}^{2}\right)} d_{x}^{2} d_{z} z_{0} + 2 \, {\left(c_{y}^{2} k_{y} + c_{y}^{2}\right)} d_{y}^{2} d_{z} z_{0}\nonumber\\&
       - 2 \, c_{x} d_{x}^{2} d_{z} - 2 \, c_{y} d_{y}^{2} d_{z}\,,\\
  c &= 6 \, c_{x}^{2} d_{x}^{2} x_{0}^{2} + 2 \, c_{x} c_{y} d_{y}^{2} x_{0}^{2} + 8 \, c_{x} c_{y} d_{x} d_{y} x_{0} y_{0} + 2 \, c_{x} c_{y} d_{x}^{2} y_{0}^{2}\nonumber\\&
     + 6 \, c_{y}^{2} d_{y}^{2} y_{0}^{2} + {\left(c_{x}^{2} k_{x} + c_{x}^{2}\right)} d_{z}^{2} x_{0}^{2} + {\left(c_{y}^{2} k_{y} + c_{y}^{2}\right)} d_{z}^{2} y_{0}^{2}\nonumber\\&
     + 4 \, {\left(c_{x}^{2} k_{x} + c_{x}^{2}\right)} d_{x} d_{z} x_{0} z_{0} + 4 \, {\left(c_{y}^{2} k_{y} + c_{y}^{2}\right)} d_{y} d_{z} y_{0} z_{0}\nonumber\\&
     + {\left(c_{x}^{2} k_{x} + c_{x}^{2}\right)} d_{x}^{2} z_{0}^{2} + {\left(c_{y}^{2} k_{y} + c_{y}^{2}\right)} d_{y}^{2} z_{0}^{2}\nonumber\\&
     - 4 \, c_{x} d_{x} d_{z} x_{0} - 4 \, c_{y} d_{y} d_{z} y_{0} - 2 \, c_{x} d_{x}^{2} z_{0} - 2 \, c_{y} d_{y}^{2} z_{0}\,,\\
  d &= 4 \, c_{x}^{2} d_{x} x_{0}^{3} + 4 \, c_{x} c_{y} d_{y} x_{0}^{2} y_{0}\nonumber\\&
     + 4 \, c_{x} c_{y} d_{x} x_{0} y_{0}^{2} + 4 \, c_{y}^{2} d_{y} y_{0}^{3}\nonumber\\&
     + 2 \, {\left(c_{x}^{2} k_{x} + c_{x}^{2}\right)} d_{z} x_{0}^{2} z_{0} + 2 \, {\left(c_{y}^{2} k_{y} + c_{y}^{2}\right)} d_{z} y_{0}^{2} z_{0} 
     + 2 \, {\left(c_{x}^{2} k_{x} + c_{x}^{2}\right)} d_{x} x_{0} z_{0}^{2} + 2 \, {\left(c_{y}^{2} k_{y} + c_{y}^{2}\right)} d_{y} y_{0} z_{0}^{2}\nonumber\\&
     - 2 \, c_{x} d_{z} x_{0}^{2} - 2 \, c_{y} d_{z} y_{0}^{2} - 4 \, c_{x} d_{x} x_{0} z_{0} - 4 \, c_{y} d_{y} y_{0} z_{0}\,,\\
  e &= c_{x}^{2} x_{0}^{4} + 2 \, c_{x} c_{y} x_{0}^{2} y_{0}^{2} + c_{y}^{2} y_{0}^{4} + {\left(c_{x}^{2} k_{x} + c_{x}^{2}\right)} x_{0}^{2} z_{0}^{2}\nonumber\\&
    + {\left(c_{y}^{2} k_{y} + c_{y}^{2}\right)} y_{0}^{2} z_{0}^{2} - 2 \, c_{x} x_{0}^{2} z_{0} - 2 \, c_{y} y_{0}^{2} z_{0}\,.
 \end{align}
\end{subequations}
Then we may consider the nature of the zeros (see Wiki) by considering the following combinations, namely
the discriminant $\Delta$, the quadratic coefficient of the depressed quartic $P$, the linear coefficient of
the depressed quartic $Q$, $\Delta_0$ which is zero if the quartic has a triple root and $D$ which is zero
if the quartic has two double roots:
\begin{subequations}
 \begin{align}
  \Delta &= 256 a^3 e^3 - 192 a^2 b d e^2 - 128 a^2 c^2 e^2 + 144 a^2 c d^2 e - 27 a^2 d^4 \nonumber\\ 
&+ 144 a b^2 c e^2 - 6 a b^2 d^2 e - 80 a b c^2 d e + 18 a b c d^3 + 16 a c^4 e \nonumber\\
&- 4 a c^3 d^2 - 27 b^4 e^2 + 18 b^3 c d e - 4 b^3 d^3 - 4 b^2 c^3 e + b^2 c^2 d^2\,,\\
   P &= 8ac - 3b^2\,,\\
   Q &= b^3+8da^2-4abc\,,\\
   \Delta_0 &= c^2 - 3bd + 12ae\,,\\
   D &= 64 a^3 e - 16 a^2 c^2 + 16 a b^2 c - 16 a^2 bd - 3 b^4\,.
 \end{align}
\end{subequations}

\begin{itemize}
\item If $\Delta < 0$  then the equation has two real roots and two complex conjugate roots.
\item If $\Delta > 0$  then the equation's four roots are either all real or all complex.
\begin{itemize}
\item If $P < 0$ and $D < 0$ then all four roots are real and distinct.
\item If $P > 0$ or $D > 0$ then there are two pairs of complex conjugate roots.
\end{itemize}
\item If $\Delta = 0$  then either the polynomial has a multiple root, 
  or it is the square of a quadratic polynomial. Here are the different cases that can occur:
  \begin{itemize}
\item If $P < 0$ and $D < 0$ and $\Delta_0\ne0$, there is a real double root and two real simple roots.
\item If $D > 0$ or ($P > 0$ and ($D \ne 0$ or $Q \ne 0$)), there is a real double root and two complex conjugate roots.
\item If $\Delta_0 = 0$ and $D \ne 0$, there is a triple root and a simple root, all real.
\item If $D = 0$, then:
  \begin{itemize}
\item If $P < 0$, there are two real double roots.
\item If $P > 0$ and $Q = 0$, there are two complex conjugate double roots.
\item If $ \Delta_0  = 0$, all four roots are equal to $-\frac{b}{4a}$
  \end{itemize}
  \end{itemize}
\end{itemize}
Some of the discussed cases show that the ray misses the surfaces.



\end{document}
