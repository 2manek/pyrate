\documentclass[12pt,a4paper,twoside,openright,BCOR10mm,headsepline,titlepage,abstracton,chapterprefix,final]{scrreprt}

\usepackage{ae}
\usepackage[ngerman, english]{babel}
%\usepackage{SIunits}

\usepackage{amsmath}
\usepackage{amssymb}
\usepackage{amsfonts}
\usepackage{xcolor}
\usepackage{setspace}

% load hyperref as the last package to avoid redefinitions of e.g. footnotes after hyperref invokation

\RequirePackage{ifpdf}  % flag for pdf or dvi backend
\ifpdf
  \usepackage[pdftex]{graphicx}
  \usepackage[pdftitle={RTFM on Imaging Theory and Basics of Optical Raytracing},%
              pdfsubject={},%
              pdfauthor={M. Esslinger, J. Hartung, U. Lippmann},%
              pdfkeywords={},%
              bookmarks=true,%
%              colorlinks=true,%
              urlcolor=blue,%
              pdfpagelayout=TwoColumnRight,%
              pdfpagemode=UseNone,%
              pdfstartview=Fit,%
	      pdfpagelabels,
              pdftex]{hyperref}
\else
  \usepackage[dvips]{graphicx}
  \usepackage[colorlinks=false,dvips]{hyperref}
\fi
%\DeclareGraphicsRule{.jpg}{eps}{.jpg}{`convert #1 eps:-}

\usepackage{ae}
%\usepackage[ngerman, english]{babel}

%\usepackage{SIunits}
\newcommand\elementarycharge{\textrm{e}}
\newcommand\sccm{\textrm{sccm}}
\newcommand\mbar{\milli\textrm{bar}}


\usepackage{amsmath}
%\usepackage{amssymb}
\usepackage{setspace}

%\widowpenalty = 1000


\newcommand*{\doi}[1]{\href{http://dx.doi.org/\detokenize{#1}}{doi: \detokenize{#1}}}

\newcommand\Vector[1]{{\mathbf{#1}}}

\newcommand\vacuum{0}

\newcommand\location{r}
\newcommand\Location{\Vector{r}}


\newcommand\wavenumber{k}
\newcommand\vacuumWavenumber{\wavenumber_{\vacuum}}
\newcommand\Wavevector{\Vector{\wavenumber}}

\newcommand\Nabla{\Vector{\nabla}}
\newcommand\Laplace{\Delta}
\newcommand\timederivative[1]{\dot{{#1}}}
\newcommand\Tensor[1]{\hat{#1}}
\newcommand\conjugate[1]{\overline{#1}}
\newcommand\transpose[1]{#1^{T}}
\newcommand\Norm[1]{\left| #1 \right|}
\newcommand{\ket}[1]{\left\vert{#1}\right\rangle}
\newcommand{\bra}[1]{\left\langle{#1}\right\vert}
\newcommand{\braket}[2]{\left\langle{#1}\vert{#2}\right\rangle}
\newcommand{\bracket}[1]{\left\langle{#1}\right\rangle}

\newcommand{\scpm}[2]{(#1\,\cdot\,#2)}

\newcommand\Greenfunction{\Tensor{G}}

\newcommand\scalarEfield{E}
\newcommand\scalarBfield{B}
\newcommand\scalarHfield{H}
\newcommand\scalarDfield{D}
\newcommand\scalarTipfield{T}
\newcommand\scalarSamplefield{S}
\newcommand\scalarDipolarmoment{p}
\newcommand\Efield{\Vector{\scalarEfield}}
\newcommand\Bfield{\Vector{\scalarBfield}}
\newcommand\Hfield{\Vector{\scalarHfield}}
\newcommand\Dfield{\Vector{\scalarDfield}}
\newcommand\Dipolarmoment{\Vector{\scalarDipolarmoment}}

\newcommand\permeability{\Tensor{\mu}}
\newcommand\vacuumpermeability{\mu_{\vacuum}}
\newcommand\permittivity{\Tensor{\epsilon}}
\newcommand\generalPermittivity{\Tensor{\tilde\epsilon}}
\newcommand\vacuumpermittivity{\epsilon_{\vacuum}}
\newcommand\scalarpermittivity{\epsilon}
\newcommand\conductivity{\Tensor{\sigma}}
\newcommand\susceptibility{\Tensor{\chi}}
\newcommand\currentdensity{\Vector{j}}
\newcommand\Current{\Vector{I}}
\newcommand\chargedensity{\rho}
\newcommand\PoyntingVector{\Vector{S}}

\newcommand\ordi{\text{ord}}
\newcommand\eo{\text{eo}}

\newcommand{\timeavg}[1]{{\langle\,#1\,\rangle}}

\newcommand{\remark}[1]{{\color{red}$\blacksquare$}\footnote{{\color{red}#1}}}
\newcommand{\chk}[1]{\color{green}{$\checkmark$#1}}



\begin{document}


\subsection{Conics}
Conics are spheres, rotationally symmetric ellipsoids, paraboloids and hyperboloids.
The word ``conic" is short for conic section. This refers to these figures as the
intersections of a plane with a cone.
In the vertex form their surface sag $z$ can be described by
\begin{eqnarray}
 z =  \frac
 { \rho ( x^2 + y^2 ) }
 { 1 + \sqrt{1 - (1+c) \rho^2  (x^2 + y^2)} }\,,
\end{eqnarray}
where $c$ is the conic constant. Depending on this parameter, the conic is a
\begin{eqnarray*}
-1 < c < 0 && \textrm{oblate ellipsoid} \\
     c = 0 && \textrm{sphere} \\
 0 < c < 1 && \textrm{prolate ellipsoid} \\
     c = 1 && \textrm{paraboloid} \\
     c > 1 && \textrm{hyperboloid}
\end{eqnarray*}
For an explicit solution of the intersection parameter $t$,
one can use the implicit form of the surface equation
\begin{align}
 \rho (1 + c) z^2 - 2 z + \rho (x^2 + y^2) &=0\,.
\end{align}
After insertion of \eqref{eq:ray}, the solution is given by
\begin{subequations}
\label{eq:intersectionconicsection}
\begin{eqnarray}
   F &=& d_z - \rho \left( d_x x_0 + d_y y_0 + d_z z_0 (1+c) \right)\,, \\
   G &=& \rho (x_0^2 + y_0^2 + z_0^2 (1+c)) - 2 z_0\,, \\
   H &=& - \rho ( 1 + c \, d_z^2 )\,, \\
   t &=& \frac{G}{ F + \sqrt{F^2 + H G} }\,.
\end{eqnarray}
\end{subequations}

The unit surface normal of such a conic section pointing in forward (positive $z$) direction is \remark{to do: simplify and display more beautifully}
\begin{eqnarray}
n[0] = - \frac{\rho x}{absn} \\
n[1] = - \frac{\rho y}{absn} \\
n[2] = \frac{1 - \rho ( 1 + c ) z}{absn}
\end{eqnarray}
with
\begin{align}
 absn &=& \sqrt{ \rho^2 x^2 + \rho^2 y^2 + \left( 1 - \rho ( 1 + c ) z \right)^2} \\
 absn &=& \sqrt{ \rho^2 x^2 + \rho^2 y^2 + \left( 1 - 2 \rho ( 1 + c ) z + \rho^2 ( 1 + c )^2 z^2\right)} \\
 absn &=& \sqrt{ \rho^2 x^2 + \rho^2 y^2 + 1 - 2 \rho ( 1 + c ) z + \rho^2 ( 1 + c )^2 z^2 }
\end{align}

\begin{align}
 \rho (1 + c) z^2 - 2 z + \rho (x^2 + y^2) &=& 0 \\
 \rho (1 + c) z^2  &=&  2 z - \rho (x^2 + y^2) \\
 \rho^2 (1 + c)^2 z^2  &=&  \rho ( 1+c) \left( 2 z - \rho (x^2 + y^2) \right) \\
 \rho^2 (1 + c)^2 z^2  &=&  \rho ( 1+c)  2 z - \rho^2 ( 1+c) (x^2 + y^2)   \\
\end{align}

\begin{align}
 absn &=& \sqrt{ \rho^2 x^2 + \rho^2 y^2 + 1 - 2 \rho ( 1 + c ) z + \left( \rho ( 1+c)  2 z - \rho^2 ( 1+c) (x^2 + y^2) \right) } \\
 absn &=& \sqrt{ \rho^2 x^2 + \rho^2 y^2 + 1 - 2 \rho ( 1 + c ) z + \rho ( 1+c)  2 z - \rho^2 ( 1+c) (x^2 + y^2) } \\
 absn &=& \sqrt{ \rho^2 x^2 + \rho^2 y^2 + 1 - \rho^2 ( 1+c) (x^2 + y^2) }
\end{align}

\begin{align}
 absn &=& \sqrt{ 1 + \rho^2 x^2 + \rho^2 y^2  - \rho^2 ( 1+c) (x^2 + y^2) } \\
 absn &=& \sqrt{ 1 - c (x^2 + y^2) }
\end{align}


\begin{eqnarray}
\Vector{n} &=& - \frac{1}{\sqrt{ 1 - c (x^2 + y^2)}} 
  \begin{pmatrix}
   \rho x \\
   \rho y \\
   \rho ( 1 + c ) z - 1
  \end{pmatrix}
\end{eqnarray}



\end{document}
