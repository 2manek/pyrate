\documentclass[12pt,a4paper,twoside,openright,BCOR10mm,headsepline,titlepage,abstracton,chapterprefix,final]{scrreprt}

\usepackage{ae}
\usepackage[ngerman, english]{babel}
%\usepackage{SIunits}

\usepackage{amsmath}
\usepackage{amssymb}
\usepackage{amsfonts}
\usepackage{xcolor}
\usepackage{setspace}

% load hyperref as the last package to avoid redefinitions of e.g. footnotes after hyperref invokation

\RequirePackage{ifpdf}  % flag for pdf or dvi backend
\ifpdf
  \usepackage[pdftex]{graphicx}
  \usepackage[pdftitle={RTFM on Imaging Theory and Basics of Optical Raytracing},%
              pdfsubject={},%
              pdfauthor={M. Esslinger, J. Hartung, U. Lippmann},%
              pdfkeywords={},%
              bookmarks=true,%
%              colorlinks=true,%
              urlcolor=blue,%
              pdfpagelayout=TwoColumnRight,%
              pdfpagemode=UseNone,%
              pdfstartview=Fit,%
	      pdfpagelabels,
              pdftex]{hyperref}
\else
  \usepackage[dvips]{graphicx}
  \usepackage[colorlinks=false,dvips]{hyperref}
\fi
%\DeclareGraphicsRule{.jpg}{eps}{.jpg}{`convert #1 eps:-}

\usepackage{ae}
%\usepackage[ngerman, english]{babel}

%\usepackage{SIunits}
\newcommand\elementarycharge{\textrm{e}}
\newcommand\sccm{\textrm{sccm}}
\newcommand\mbar{\milli\textrm{bar}}


\usepackage{amsmath}
%\usepackage{amssymb}
\usepackage{setspace}

%\widowpenalty = 1000


\newcommand*{\doi}[1]{\href{http://dx.doi.org/\detokenize{#1}}{doi: \detokenize{#1}}}

\newcommand\Vector[1]{{\mathbf{#1}}}

\newcommand\vacuum{0}

\newcommand\location{r}
\newcommand\Location{\Vector{r}}


\newcommand\wavenumber{k}
\newcommand\vacuumWavenumber{\wavenumber_{\vacuum}}
\newcommand\Wavevector{\Vector{\wavenumber}}

\newcommand\Nabla{\Vector{\nabla}}
\newcommand\Laplace{\Delta}
\newcommand\timederivative[1]{\dot{{#1}}}
\newcommand\Tensor[1]{\hat{#1}}
\newcommand\conjugate[1]{\overline{#1}}
\newcommand\transpose[1]{#1^{T}}
\newcommand\Norm[1]{\left| #1 \right|}
\newcommand{\ket}[1]{\left\vert{#1}\right\rangle}
\newcommand{\bra}[1]{\left\langle{#1}\right\vert}
\newcommand{\braket}[2]{\left\langle{#1}\vert{#2}\right\rangle}
\newcommand{\bracket}[1]{\left\langle{#1}\right\rangle}

\newcommand{\scpm}[2]{(#1\,\cdot\,#2)}

\newcommand\Greenfunction{\Tensor{G}}

\newcommand\scalarEfield{E}
\newcommand\scalarBfield{B}
\newcommand\scalarHfield{H}
\newcommand\scalarDfield{D}
\newcommand\scalarTipfield{T}
\newcommand\scalarSamplefield{S}
\newcommand\scalarDipolarmoment{p}
\newcommand\Efield{\Vector{\scalarEfield}}
\newcommand\Bfield{\Vector{\scalarBfield}}
\newcommand\Hfield{\Vector{\scalarHfield}}
\newcommand\Dfield{\Vector{\scalarDfield}}
\newcommand\Dipolarmoment{\Vector{\scalarDipolarmoment}}

\newcommand\permeability{\Tensor{\mu}}
\newcommand\vacuumpermeability{\mu_{\vacuum}}
\newcommand\permittivity{\Tensor{\epsilon}}
\newcommand\generalPermittivity{\Tensor{\tilde\epsilon}}
\newcommand\vacuumpermittivity{\epsilon_{\vacuum}}
\newcommand\scalarpermittivity{\epsilon}
\newcommand\conductivity{\Tensor{\sigma}}
\newcommand\susceptibility{\Tensor{\chi}}
\newcommand\currentdensity{\Vector{j}}
\newcommand\Current{\Vector{I}}
\newcommand\chargedensity{\rho}
\newcommand\PoyntingVector{\Vector{S}}

\newcommand\ordi{\text{ord}}
\newcommand\eo{\text{eo}}

\newcommand{\timeavg}[1]{{\langle\,#1\,\rangle}}

\newcommand{\remark}[1]{{\color{red}$\blacksquare$}\footnote{{\color{red}#1}}}
\newcommand{\chk}[1]{\color{green}{$\checkmark$#1}}



\begin{document}

\begin{eqnarray}
 0 &=& \left(-\Vector{k}^2 \delta_{ij} + k_i k_j + \omega^2 \mu_0 \permittivity_{ij} \right) E_{j\,0} \\
 0 &=& -\Vector{k}^2 E_{i\,0} + k_i k_j E_{j\,0} + \omega^2 \mu_0 D_{i\,0} \\
\end{eqnarray}

Without loss of generality, we assume $\Wavevector \propto \Vector{e}_z$. [https://de.wikipedia.org/w/index.php?title=Kristalloptik\&oldid=159476511]
From the Maxwell equation $\Nabla \Dfield = 0$ we conclude $\Dfield \perp \Wavevector$ and $\scalarDfield_z=0$.

\begin{eqnarray}
 - \wavenumber_z^2 E_{x\,0} + \omega^2 \mu_0 \scalarDfield_{x\,0} &=& 0\\ 
 - \wavenumber_z^2 E_{y\,0} + \omega^2 \mu_0 \scalarDfield_{y\,0} &=& 0\\
 0 &=& 0
\end{eqnarray}

We define the inverse permittivity as $\alpha_{ij} = (\permittivity^{-1})_{ij}$

\begin{eqnarray}
 - \wavenumber_z^2 (\alpha_{xx} \scalarDfield_{x\,0} + \alpha_{xy} \scalarDfield_{y\,0}) + \omega^2 \mu_0 \scalarDfield_{x\,0} &=& 0\\ 
 - \wavenumber_z^2 (\alpha_{yx} \scalarDfield_{x\,0} + \alpha_{yy} \scalarDfield_{y\,0}) + \omega^2 \mu_0 \scalarDfield_{y\,0} &=& 0
\end{eqnarray}

We introduce the $\Dfield$ polarisation angle $\phi$
\begin{eqnarray}
 \scalarDfield_{x\,0} &=& \scalarDfield_{0} \cos \phi \\
 \scalarDfield_{y\,0} &=& \scalarDfield_{0} \sin \phi
\end{eqnarray}

\begin{eqnarray}
 - \wavenumber_z^2 (\alpha_{xx} \scalarDfield_{0} \cos \phi + \alpha_{xy} \scalarDfield_{0} \sin \phi) + \omega^2 \mu_0 \scalarDfield_{0} \cos \phi &=& 0\\ 
 - \wavenumber_z^2 (\alpha_{yx} \scalarDfield_{0} \cos \phi + \alpha_{yy} \scalarDfield_{0} \sin \phi) + \omega^2 \mu_0 \scalarDfield_{0} \sin \phi &=& 0
\end{eqnarray}




\end{document}
