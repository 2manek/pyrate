\documentclass[12pt,a4paper,twoside,openright,BCOR10mm,headsepline,titlepage,abstracton,chapterprefix,final]{scrreprt}

\usepackage{ae}
\usepackage[ngerman, english]{babel}
%\usepackage{SIunits}

\usepackage{amsmath}
\usepackage{amssymb}
\usepackage{amsfonts}
\usepackage{xcolor}
\usepackage{setspace}
\usepackage{dsfont}

% load hyperref as the last package to avoid redefinitions of e.g. footnotes after hyperref invokation

\RequirePackage{ifpdf}  % flag for pdf or dvi backend
\ifpdf
  \usepackage[pdftex]{graphicx}
  \usepackage[pdftitle={RTFM on Imaging Theory and Basics of Optical Raytracing},%
              pdfsubject={},%
              pdfauthor={M. Esslinger, J. Hartung, U. Lippmann},%
              pdfkeywords={},%
              bookmarks=true,%
%              colorlinks=true,%
              urlcolor=blue,%
              pdfpagelayout=TwoColumnRight,%
              pdfpagemode=UseNone,%
              pdfstartview=Fit,%
	      pdfpagelabels,
              pdftex]{hyperref}
\else
  \usepackage[dvips]{graphicx}
  \usepackage[colorlinks=false,dvips]{hyperref}
\fi
%\DeclareGraphicsRule{.jpg}{eps}{.jpg}{`convert #1 eps:-}

%\usepackage{SIunits}

\newcommand{\vct}[1]{\mathbf{#1}}
\newcommand{\eps}[1]{\varepsilon_{#1}}
\newcommand{\an}[1]{\alpha_{#1}}
\newcommand{\deteps}{\det \varepsilon}
\newcommand{\tr}[1]{\text{tr}\left(#1\right)}
\newcommand{\scpm}[2]{{(#1\cdot#2)}}
\begin{document}

\section{Dispersion relation and its derivatives}

Maybe useful for simplification: Given a unit vector $e_i$ then there is a projector given by
\begin{align}
 Pe_{ij} &= \delta_{ij} - e_i e_j\,.
\end{align}
Therefore maybe this opens the possibility to simplify the double tensor constructions better
\begin{align}
 (a_i \eps{ij} \eps{jk} b_k) &= (a_i \eps{ij} \delta_{j\ell} \eps{\ell k} b_k) 
= (a_i \eps{ij} (Pe_{j\ell} + e_j e_\ell) \eps{\ell k} b_k) = (a_i \eps{ij} Pe_{j\ell} \eps{\ell k} b_k) + (a_i \eps{ij} e_j)(e_\ell \eps{\ell k} b_k)
\end{align}
And from three linear independent unity vectors $u_i, v_i, w_i$ it follows
\begin{align}
 \delta_{ij} = u_i u_j + v_i v_j + w_i w_j\,.
\end{align}



\subsection{General case}

Given the plane wave propagator in anisotropic medium. The permittivity tensor neither has to symmetric nor real.
For simplicity the wave vectors are scaled with the wave number such that they are without dimensions,
\begin{align}
 P_{ij}(\vct{k}) &= \eps{ij} - \delta_{ij} \vct{k}^2 + k_i k_j\,.
\end{align}
The eigenvalue problem of the wave equation $P_{ij} E_i = 0$ should have non trivial solutions and therefore the determinant of $P_{ij}$
must vanish (the following expression is only valid for a $3\times3$ matrix)
\begin{align}
 D \equiv \det P &= \frac{1}{6} (\tr{P}^3 - 3 \tr{P} \tr{P^2} + 2 \tr{P^3}) \stackrel{!}{=} 0\,.
\end{align}
From this we get
\begin{align}
 D(\vct{k}) &= \deteps - (\eps{\ell\ell}) (\eps{ij} k_i k_j) + (\eps{ij} \eps{j \ell} k_i k_\ell) + (\eps{ij} k_i k_j) \vct{k}^2\,. 
\end{align}
The first derivative is given by
\begin{align}
 \frac{\partial D}{\partial k_k} &= (\eps{ik} + \eps{ki}) [((-\eps{jj} + \vct{k}^2)k_i + \eps{i\ell} k_\ell] + 2(\eps{ij} k_i k_j) k_k\,. 
\end{align}
The second derivative is given by
\begin{align}
 \frac{\partial^2 D}{\partial k_i \partial k_j} &= (\eps{ij} + \eps{ji})(-\eps{\ell\ell} + \vct{k}^2) + 2 \delta_{ij} (\eps{k\ell} k_k k_\ell)
  + 2[(\eps{j \ell} + \eps{\ell j}) k_\ell k_i + (\eps{i \ell} + \eps{\ell i}) k_\ell k_j] \nonumber\\
  &+ \eps{i \ell} \eps{\ell j} + \eps{j \ell} \eps{\ell i}\,.
\end{align}
The third one is given by
\begin{align}
 \frac{\partial^3 D}{\partial k_i \partial k_j \partial k_k} &= 2 \biggl[
  (\eps{k\ell} + \eps{\ell k})k_\ell \delta_{ij}
  + (\eps{i\ell} + \eps{\ell i})k_\ell \delta_{jk}
  + (\eps{j\ell} + \eps{\ell j})k_\ell \delta_{ki} \nonumber\\&
  + (\eps{ij} + \eps{ji}) k_k
  + (\eps{jk} + \eps{kj}) k_i
  + (\eps{ki} + \eps{ik}) k_j
 \biggr]\,.
\end{align}
The fourth one is given by
\begin{align}
 \frac{\partial^4 D}{\partial k_i \partial k_j \partial k_m \partial k_n} &= 2 \biggl[
  (\eps{ij} + \eps{ji}) \delta_{mn}
  +(\eps{jm} + \eps{mj}) \delta_{ni}
  +(\eps{mn} + \eps{nm}) \delta_{ij} \nonumber\\&
  +(\eps{ni} + \eps{in}) \delta_{jm}
  +(\eps{im} + \eps{mi}) \delta_{nj}
  +(\eps{jn} + \eps{nj}) \delta_{im}  
 \biggr]\,.
\end{align}




\subsection{Isotropic case}
The isotropic case is given by $\eps{ij} = \varepsilon \delta_{ij}$ and in this case all expressions from above are simplified to
\begin{align}
 D_\text{iso}(\vct{k}) &= \varepsilon (\varepsilon - \vct{k}^2)^2\,.
\end{align}
The first derivative is
\begin{align}
 \frac{\partial D_\text{iso}}{\partial k_i} &= -4 \varepsilon (\varepsilon - \vct{k}^2) k_i\,.
\end{align}
The second derivative is given by
\begin{align}
 \frac{\partial^2 D_\text{iso}}{\partial k_i \partial k_j} &= -4 \varepsilon [ \delta_{ij} (\varepsilon - \vct{k}^2) - 2 k_i k_j ]\,.
\end{align}
The third derivative is given by
\begin{align}
 \frac{\partial^3 D_\text{iso}}{\partial k_i \partial k_j \partial k_k} &= 8 \varepsilon [ \delta_{ij} k_k + \delta_{jk} k_i + \delta_{ki} k_j ]\,.
\end{align}
The fourth derivative is given by
\begin{align}
 \frac{\partial^4 D_\text{iso}}{\partial k_i \partial k_j \partial k_m \partial k_n} &= 8 \varepsilon [ \delta_{ij} \delta_{mn} + \delta_{jm} \delta_{in} + \delta_{mi} \delta_{jn} ]\,.
\end{align}





\subsection{Perturbative case}

Consider $\eps{ij} = \varepsilon (\delta_{ij} + \beta \an{ij})$ where the first part is isotropic and the second is a small anisotropic correction.
($\beta$ bookkeeping parameter.)
Consider also $k_i = k_{(0)i} + \beta k_{(1)i} + \beta^2 k_{(2)i}$ and calculate the first order correction of $\vct{k}$ under anisotropic corrections.
First of all consider the determinant of $\eps{ij}$
\begin{align}
 \det\varepsilon &= \varepsilon^3 \left[1 + \beta (\an{jj}) + \frac{\beta^2}{2} ( (\an{jj})^2 - (\an{ij}\an{ji}) ) + \dots\right]\,.
\end{align}
Since the special property of the determinant (the gradient vanishes for the isotropic case) it is necessary to expand it up to second order in $\beta$.
Solving those equations order by order leads to
\begin{align}
 \vct{k}_{(0)}^2 &= \varepsilon\,,
\end{align}
the first order vanishes identically and the second order leads to a quadratic equation for $\vct{k}_{(1)}$ (the contribution from $\vct{k}_{(2)}$ also
vanishes due to the dispersion relation solution of $\vct{k}_{(0)}$).
This will later be important when finding a solution for the small vector deviation.


\section{Dispersion under small vector deviation}

\subsection{General case}

\begin{align}
 D(\vct{k} + \Delta \vct{k}) - D(\vct{k}) &= (\Delta \vct{k})^2 (\eps{ij} \Delta k_i \Delta k_j)\nonumber\\&
 + 2 (\eps{ij} \Delta k_i \Delta k_j) \scpm{\Delta \vct{k}}{\vct{k}} + (\Delta \vct{k})^2 (\Delta k_i (\eps{ij} + \eps{ji}) k_j)\nonumber\\&
 + (\Delta k_i \eps{ij} \eps{jk} \Delta k_k) + (\Delta k_i \Delta k_j \eps{ij})[-(\eps{\ell\ell}) + \vct{k}^2]\nonumber\\&
 + 2 \scpm{\Delta \vct{k}}{\vct{k}} (\Delta k_i (\eps{ij} + \eps{ji}) k_j) + (\Delta \vct{k})^2 (\eps{ij} k_i k_j)\nonumber\\&
 + [-(\eps{\ell\ell}) + \vct{k}^2](\Delta k_i (\eps{ij} + \eps{ji}) k_j) + 2 \scpm{\Delta \vct{k}}{\vct{k}} (\eps{ij} k_i k_j)\nonumber\\&
 + (\Delta k_i (\eps{i\ell}\eps{\ell j} + \eps{j\ell}\eps{\ell i}) k_j)\,.
\end{align}


\subsection{Isotropic case}

\begin{align}
 D_\text{iso}(\vct{k} + \Delta \vct{k}) - D_\text{iso}(\vct{k}) &= \varepsilon ((\Delta \vct{k})^2)^2 
 + 4 \varepsilon (\Delta \vct{k})^2 \scpm{\Delta \vct{k}}{\vct{k}}\nonumber\\&
 + \varepsilon \left(4 \scpm{\Delta \vct{k}}{\vct{k}}^2 -2 (\Delta \vct{k})^2 (-\varepsilon + \vct{k}^2)\right)\nonumber\\&
 + 4 \varepsilon \scpm{\Delta \vct{k}}{\vct{k}}(-\varepsilon + \vct{k}^2)\,.
\end{align}
For the solution of the determinant equation in the isotropic case this leads to 
$((\Delta \vct{k})^2 + 2 \scpm{\Delta \vct{k}}{\vct{k}})^2 = 0$.

\subsection{Perturbative case}

\end{document}