\documentclass[12pt,a4paper,twoside,openright,BCOR10mm,headsepline,titlepage,abstracton,chapterprefix,final]{scrreprt}

\usepackage{ae}
\usepackage[ngerman, english]{babel}
%\usepackage{SIunits}

\usepackage{amsmath}
\usepackage{amssymb}
\usepackage{amsfonts}
\usepackage{xcolor}
\usepackage{setspace}

% load hyperref as the last package to avoid redefinitions of e.g. footnotes after hyperref invokation

\RequirePackage{ifpdf}  % flag for pdf or dvi backend
\ifpdf
  \usepackage[pdftex]{graphicx}
  \usepackage[pdftitle={RTFM on Imaging Theory and Basics of Optical Raytracing},%
              pdfsubject={},%
              pdfauthor={M. Esslinger, J. Hartung, U. Lippmann},%
              pdfkeywords={},%
              bookmarks=true,%
%              colorlinks=true,%
              urlcolor=blue,%
              pdfpagelayout=TwoColumnRight,%
              pdfpagemode=UseNone,%
              pdfstartview=Fit,%
	      pdfpagelabels,
              pdftex]{hyperref}
\else
  \usepackage[dvips]{graphicx}
  \usepackage[colorlinks=false,dvips]{hyperref}
\fi
%\DeclareGraphicsRule{.jpg}{eps}{.jpg}{`convert #1 eps:-}

\usepackage{ae}
%\usepackage[ngerman, english]{babel}

%\usepackage{SIunits}
\newcommand\elementarycharge{\textrm{e}}
\newcommand\sccm{\textrm{sccm}}
\newcommand\mbar{\milli\textrm{bar}}


\usepackage{amsmath}
%\usepackage{amssymb}
\usepackage{setspace}

%\widowpenalty = 1000


\newcommand*{\doi}[1]{\href{https://doi.org/\detokenize{#1}}{doi: \detokenize{#1}}}

\newcommand\Vector[1]{{\mathbf{#1}}}

\newcommand\vacuum{0}

\newcommand\location{r}
\newcommand\Location{\Vector{r}}


\newcommand\wavenumber{k}
\newcommand\vacuumWavenumber{\wavenumber_{\vacuum}}
\newcommand\Wavevector{\Vector{\wavenumber}}

\newcommand\Nabla{\Vector{\nabla}}
\newcommand\Laplace{\Delta}
\newcommand\timederivative[1]{\dot{{#1}}}
\newcommand\Tensor[1]{\hat{#1}}
\newcommand\conjugate[1]{\overline{#1}}
\newcommand\transpose[1]{#1^{T}}
\newcommand\Norm[1]{\left| #1 \right|}
\newcommand{\ket}[1]{\left\vert{#1}\right\rangle}
\newcommand{\bra}[1]{\left\langle{#1}\right\vert}
\newcommand{\braket}[2]{\left\langle{#1}\vert{#2}\right\rangle}
\newcommand{\bracket}[1]{\left\langle{#1}\right\rangle}

\newcommand{\scpm}[2]{(#1\,\cdot\,#2)}

\newcommand\Greenfunction{\Tensor{G}}

\newcommand\scalarEfield{E}
\newcommand\scalarBfield{B}
\newcommand\scalarHfield{H}
\newcommand\scalarDfield{D}
\newcommand\scalarTipfield{T}
\newcommand\scalarSamplefield{S}
\newcommand\scalarDipolarmoment{p}
\newcommand\Efield{\Vector{\scalarEfield}}
\newcommand\Bfield{\Vector{\scalarBfield}}
\newcommand\Hfield{\Vector{\scalarHfield}}
\newcommand\Dfield{\Vector{\scalarDfield}}
\newcommand\Dipolarmoment{\Vector{\scalarDipolarmoment}}

\newcommand\permeability{\Tensor{\mu}}
\newcommand\vacuumpermeability{\mu_{\vacuum}}
\newcommand\permittivity{\Tensor{\epsilon}}
\newcommand\generalPermittivity{\Tensor{\tilde\epsilon}}
\newcommand\vacuumpermittivity{\epsilon_{\vacuum}}
\newcommand\scalarpermittivity{\epsilon}
\newcommand\conductivity{\Tensor{\sigma}}
\newcommand\susceptibility{\Tensor{\chi}}
\newcommand\currentdensity{\Vector{j}}
\newcommand\Current{\Vector{I}}
\newcommand\chargedensity{\rho}
\newcommand\PoyntingVector{\Vector{S}}

\newcommand\ordi{\text{ord}}
\newcommand\eo{\text{eo}}

\newcommand{\timeavg}[1]{{\langle\,#1\,\rangle}}

\newcommand{\remark}[1]{{\color{red}$\blacksquare$}\footnote{{\color{red}#1}}}
\newcommand{\chk}[1]{\color{green}{$\checkmark$#1}}


\newif\ifdraft
\draftfalse % \drafttrue




\begin{document}

\section{A nonlinear transfer formalisms}
The following section shows some suggestions for linear and non-linear transfer formalisms
to simplify and unify ray tracing. The transfer formalisms should have the following properties:

\begin{itemize}
 \item They should be physical and not ad-hoc (physical foundations, quick access to conserved quantities and so on)
 \item A linear transfer formalism like $ABCD$ should be the linearisation of a non-linear transfer formalism
 \item They should be formulated with respect to the optical axis
 \item They should incorporate changes of the optical axis in a unified manner
\end{itemize}

\remark{It would be cool to only stich these transfer functions together like in the ABCD formalism;
this means in particular, that a ray just should move by $(\Vector{X}_f, \Vector{d}_f) = F_n (F_{n-1}( \dots F_2 (F_1(\Vector{X}_i, \Vector{d}_i))))$ where
the $F_k$ are functions discussed in the following subsections.}

\subsection{Nonlinear XYUV formalism}
According to {{\tt \url{http://graphics.stanford.edu/courses/cs148-10-summer/docs/2006--degreve--reflection_refraction.pdf}}}
we will call the incident vector $\Vector{i}$ the reflected one $\Vector{r}$ and the refracted one $\Vector{t}$.
From reflection law we get the following ($\Vector{n}$ is pointing into medium $n_1$ and the incidence vector is pointing from $n_1$ to $n_2$)
\begin{align}
 \Vector{r} &= \Vector{i} - 2 \scpm{\Vector{i}}{\Vector{n}} \Vector{n}\,.\label{eq:reflection_vector}
\end{align}
From refraction we get\remark{I know that is not as elegant as the former 
solution by considering the field components at material boundaries :-)}
\begin{align}
 \Vector{t} &= \frac{n_1}{n_2} \Vector{i} 
 - \left[\frac{n_1}{n_2} \scpm{\Vector{i}}{\Vector{n}} 
      - \sqrt{1 - \left(\frac{n_1}{n_2}\right)^2 (1 - {\scpm{\Vector{i}}{\Vector{n}}}^2)}\right] \Vector{n}\,.\label{eq:refraction_vector}
\end{align}
Let us now consider the optical axis $\Vector{e}_z$ and some starting point of the ray at distance $d_1$ at the left hand side
of the optical surface $F(x, y)$, $(x,y,-d_1)$. The direction vector is given by $\Vector{d} = (d_x, d_y, \sqrt{1 - d_x^2 - d_y^2})$.
The components of the direction vector are equivalent to some aiming angles. For our convenience we take our
four variables $(x, y, d_x, d_y)$ as free starting parameters and try to calculate the new parameters at distance $d_2$ behind the
optical surface $F(x, y)$, $(x', y', {d_x}', {d_y}')$, via some nonlinear transform $(x', y', {d_x}', {d_y}') = T(x, y, d_x, d_y)$,
where $T$ is some function $T:\mathbb{R}^4 \to \mathbb{R}^4$.

By using the formulas from \ref{subsec:intersectionformulas} it is possible to invert the equation 
\begin{align}
-d_1 + t \sqrt{1-d_x^2-d_y^2} &= F(x + t d_x, y + t d_y)\,,\label{eq:genintersection} 
\end{align}
for $t$ either analytically or numerically. The equation tells
us where the ray started at $x, y$, $z = -d_1$ intersects the surface $F(x, y)$. Let the solution
of this equation $t_\text{intersect} = \sigma(x, y, d_x, d_y)$. (The function $\sigma$
is in general nonlinear in its arguments, even for the simplest case of a sphere.) Then by using the 
sigma notation the intersection point is also given by a nonlinear vector entity:
\begin{align}
 \Vector{p}_\text{i} &= \begin{pmatrix} x + \sigma(x,y,d_x,d_y) d_x \\ y + \sigma(x,y,d_x,d_y) d_y \\ -d_1 + \sigma(x,y,d_x,d_y) \sqrt{1 - d_x^2 - d_y^2} \end{pmatrix}\,.\label{eq:pintersect}
\end{align}
The vector perpendicular on the surface pointing in direction of medium $n_1$ (left hand side, negative $z$ direction) is given
by
\begin{align}
 \Vector{n}(x, y) &= 
  \frac{1}{\sqrt{1 + (\Vector{\nabla}_{x,y} F)^2}} 
  \begin{pmatrix} \frac{\partial F}{\partial y} \\ \frac{\partial F}{\partial y} \\ -1 \end{pmatrix}\,.\label{eq:surfacenormal}
\end{align}
This vector has to be evaluated at $\Vector{p}_\text{i}$ to get the appropriate results for
the transmission and reflection direction. Now we may use \eqref{eq:refraction_vector} or \eqref{eq:reflection_vector}
and \eqref{eq:surfacenormal} evaluated at \eqref{eq:pintersect} to calculate the direction vector $\Vector{d}'(p_{\text{i} x}, p_{\text{i} y})$.
To calculate $x'$ and $y'$ we have to calculate another ray which starts at $p_\text{intersect}$ and ends at the
distance $d_2$ from the plane. Therefore $p_{\text{i}\,z} + t_\text{new} \sqrt{1 - {d'}_x^2 - {d'}_y^2} = d_2$ and so
\begin{align}
t_\text{new} &= \frac{d_2 - p_{\text{i}\,z}}{\sqrt{1 - {d'}_x^2 - {d'}_y^2}}\,.
\end{align}
Thus the final coordinates are given by
\begin{subequations}
\begin{align}
 x' &= p_{\text{i}\,x} + \frac{d_2 - p_{\text{i}\,z}}{\sqrt{1 - {d'}_x^2 - {d'}_y^2}} {d'}_x\,,\\
 y' &= p_{\text{i}\,y} + \frac{d_2 - p_{\text{i}\,z}}{\sqrt{1 - {d'}_x^2 - {d'}_y^2}} {d'}_y\,.
\end{align}
\end{subequations}
Collecting all together: The first $t$ value is (via inverting \eqref{eq:genintersection})
\begin{align}
 t_\text{intersect} &= \sigma(x, y, d_x, d_y) = \sigma\,.\label{eq:sigmasol}
\end{align}
The intersection point is given by
\begin{align}
 \Vector{p}_\text{i} &= \begin{pmatrix} x + \sigma d_x \\ y + \sigma d_y \\ -d_1 + \sigma \sqrt{1 - d_x^2 - d_y^2} \end{pmatrix}\,.
\end{align}
The unit normal vector has to be evaluated and according to \eqref{eq:surfacenormal} given by
\begin{align}
 \Vector{N} &:= \Vector{n}(p_{\text{i}x}, p_{\text{j}y})\,.\label{eq:surfacenormal2}
\end{align}
For a refractive problem the new direction vector is derived from \eqref{eq:refraction_vector} and therefore
\begin{align}
 \Vector{d}' &= \frac{n_1}{n_2} \Vector{d} 
 - \left[\frac{n_1}{n_2} \scpm{\Vector{d}}{\Vector{N}} 
      + \sqrt{1 - \left(\frac{n_1}{n_2}\right)^2 (1 - {\scpm{\Vector{d}}{\Vector{N}}}^2)}\right] \Vector{N}\,.\label{eq:newdirection}
\end{align}
and for the final coordinates after inserting the intersection point coordinates
\begin{subequations}
\label{eq:newposition}
\begin{align}
 x' &= x + \sigma d_x + \frac{d_2 + d_1 - \sigma \sqrt{1 - d_x^2 - d_y^2}}{\sqrt{1 - {d'}_x^2 - {d'}_y^2}} {d'}_x\,,\\
 y' &= y + \sigma d_y + \frac{d_2 + d_1 - \sigma \sqrt{1 - d_x^2 - d_y^2}}{\sqrt{1 - {d'}_x^2 - {d'}_y^2}} {d'}_y\,.
\end{align}
\end{subequations}
Eqns. \eqref{eq:newdirection} and \eqref{eq:newposition} together with \eqref{eq:surfacenormal2}, \eqref{eq:surfacenormal},
\eqref{eq:pintersect} and \eqref{eq:sigmasol} give the final nonlinear transformation for a refractive optical surface between
two materials with constant optical indices. $\sigma$ depends on the starting variables, too.  
For, e.g., a conic section it is given by (see \eqref{eq:intersectionconicsection})
\begin{align}
 \sigma_{\text{conic section}} &= \frac{G}{ F + \sqrt{F^2 + H G} }\,,
\end{align}
where
\begin{eqnarray}
   F &=& \sqrt{1 - d_x^2 - d_y^2} - \rho \left( d_x x + d_y y - d_1 (1+c) \sqrt{1 - d_x^2 - d_y^2} \right)\,, \\
   G &=& \rho (x^2 + y^2 + d_1^2 (1+c)) + 2 d_1\,, \\
   H &=& - \rho ( 1 + c \, (1 - d_x^2 - d_y^2) )\,.
\end{eqnarray}
In a strictly linear approximation of the new coordinates where we neglect all dependencies of $x$, $y$, $d_x$, $d_y$,
\begin{align}
 \sigma_\text{conic section} &\approx \frac{d_1}{1+ \frac{1}{\rho d_1 (1 + c)}} \stackrel{\rho d_1 (1+c)\gg 1}{\approx} d_1 \left(1 - d_1 \rho(1+c) + \mathcal{O}(\rho d_1 (1+c))^2\right)\,. 
\end{align}
If we further assume that we may neglect nonlinear terms in the new direction equation \eqref{eq:newdirection}
which means [we also approximate $\sigma \approx d_1$]\\
\begin{subequations}
 \begin{align}
  {d'}_x &= {\left(\left[\frac{n_{1}}{n_{2}} - 1\right]\rho \sigma + \frac{n_{1}}{n_{2}}\right)} d_{x} + {\left(\frac{n_{1}}{n_{2}} - 1\right)} \rho x\,,\\
  {d'}_y &= {\left(\left[\frac{n_{1}}{n_{2}} - 1\right]\rho \sigma + \frac{n_{1}}{n_{2}}\right)} d_{y} + {\left(\frac{n_{1}}{n_{2}} - 1\right)} \rho y\,,
 \end{align}
\end{subequations}
we can simplify the new coordinates to (note: linear approximation! So the square roots are $\approx1$!)
\begin{subequations}
\label{eq:linearapprox}
\begin{align}
 x' &\approx {\left({\left(\rho d_{1} \left(\frac{n_{1}}{n_{2}} - 1\right) + \frac{n_{1}}{n_{2}}\right)} d_{2} + d_{1}\right)} d_{x} + {\left(\rho d_{2} {\left(\frac{n_{1}}{n_{2}} - 1\right)} + 1\right)} x\,,\\
 y' &\approx {\left({\left(\rho d_{1} \left(\frac{n_{1}}{n_{2}} - 1\right) + \frac{n_{1}}{n_{2}}\right)} d_{2} + d_{1}\right)} d_{y} + {\left(\rho d_{2} {\left(\frac{n_{1}}{n_{2}} - 1\right)} + 1\right)} y\,.
\end{align}
\end{subequations}
Therefore the matrix formulation is given by
\begin{align}
 &\begin{pmatrix} x' \\ y' \\ {d'}_x \\ {d'}_y \end{pmatrix} =\nonumber\\ 
 &\begin{pmatrix} \rho d_{2} {\left(\frac{n_{1}}{n_{2}} - 1\right)} + 1 & 0 & {\left(\rho d_{1} \left(\frac{n_{1}}{n_{2}} - 1\right) + \frac{n_{1}}{n_{2}}\right)} d_{2} + d_{1} & 0 \\ 
		  0 & \rho d_{2} {\left(\frac{n_{1}}{n_{2}} - 1\right)} + 1 & 0 & {\left(\rho d_{1} \left(\frac{n_{1}}{n_{2}} - 1\right) + \frac{n_{1}}{n_{2}}\right)} d_{2} + d_{1} \\ 
		  \left(\frac{n_{1}}{n_{2}} - 1\right) \rho & 0 & \left[\frac{n_{1}}{n_{2}} - 1\right]\rho d_1 + \frac{n_{1}}{n_{2}} & 0 \\ 
		  0 & \left(\frac{n_{1}}{n_{2}} - 1\right) \rho & 0 & \left[\frac{n_{1}}{n_{2}} - 1\right]\rho d_1 + \frac{n_{1}}{n_{2}}\end{pmatrix}\nonumber\\ 
 &\times\begin{pmatrix} x \\ y \\ d_x \\ d_y \end{pmatrix}\,,
\end{align}
which is a very rough approximation.
This should be equivalent to translation by distance $d_1$ to the optical surface,
refraction of the rays at the optical surface (conic section) and translation by distance
$d_2$. Further we could reduce this problem to $ABCD$ matrices because there is no mixing
between $x$ and $y$ axis,
\begin{align}
 \begin{pmatrix} r'  \\ d' \end{pmatrix} &= 
 \begin{pmatrix} \rho d_{2} {\left(\frac{n_{1}}{n_{2}} - 1\right)} + 1 & {\left(\rho d_{1} \left(\frac{n_{1}}{n_{2}} - 1\right) + \frac{n_{1}}{n_{2}}\right)} d_{2} + d_{1} \\ 
		\left(\frac{n_{1}}{n_{2}} - 1\right) \rho &  \left[\frac{n_{1}}{n_{2}} - 1\right]\rho d_1 + \frac{n_{1}}{n_{2}} \end{pmatrix}
 \begin{pmatrix} r  \\ d \end{pmatrix}\,.
\end{align}
This is exactly the transformation which can be obtained from using the linear $ABCD$ matrix formalism by considering
\begin{align}
 \begin{pmatrix} r'  \\ d' \end{pmatrix} &=
 \begin{pmatrix} 1 & d_2 \\ 0 & 1 \end{pmatrix}
 \begin{pmatrix} 1 & 0 \\ \left(\frac{n_1}{n_2} - 1\right)\rho & \frac{n_1}{n_2} \end{pmatrix}
 \begin{pmatrix} 1 & d_1 \\ 0 & 1 \end{pmatrix}
 \begin{pmatrix} r  \\ d \end{pmatrix}\,.
\end{align}
Thus the linear approximation relies on the following points:
\begin{itemize}
 \item Small angles and small deviations from optical axis (linear approximation)
 \item Approximation of $\sigma$ by distance $d_1$ to surface 
 \item Linear approximation of normal vector of surface [this implies approximation of surface by local curvature at $x=y=0$]
 \item At least for the $ABCD$ formalism: rotationally symmetric surface
 \item Properties of surface enter the calculation only via $\sigma$ and $\Vector{N}$: the linear dependence of $\rho$ comes from $\Vector{N}$
\end{itemize}
\remark{Is it possible by comparison of fully nonlinear transfer formalism and linear approximation, i.e. $ABCD$ formalism, to obtain the
aberrations?}

\subsection{5x5 formalism}
In computer vision there are homogeneous coordinates used to implement translation and rotation in a single matrix.
There is a further coordinate $c$ introduced such that $\bar{x} = x/c, \bar{y} = y/c$ where $x$, $y$ and $c$
are the homogeneous coordinates and $\bar{x}$ and $\bar{y}$ are the view coordinates. Then general transformations have the following block
structure
\begin{align}
 \begin{pmatrix} \Vector{r}' \\ c' \end{pmatrix} &= 
 \begin{pmatrix}
  R & \Vector{t} \\
  0 & 1
 \end{pmatrix}
 \begin{pmatrix}
  \Vector{r} \\ 
  c
 \end{pmatrix}\,.
\end{align}
Therefore
\begin{align}
 \Vector{r}' &= R \Vector{r} + c \Vector{t}\,,\label{eq:homotransform}\\
 c' &= c
\end{align}
Usually one sets the $c$ equal to one and therefore $\Vector{r}'$ contains
the transformed coordinates already. As one can see in \eqref{eq:homotransform}
$\Vector{r}'$ is also translated.

In the $XYUV$ formalism $\Vector{r}$ and $\Vector{r}'$ are given by $(x,y,d_x,d_y)$ and $(x',y',{d'}_x,{d'}_y)$
respectively. To get a feeling for the transformation of the homogeneous coordinates we consider only
a translation
\begin{align}
 \begin{pmatrix} x' \\ y' \\ {d'}_x \\ {d'}_y \\ 1 \end{pmatrix} &=
 \begin{pmatrix} 1 & 0 & 0 & 0 & t_x \\
                 0 & 1 & 0 & 0 & t_y \\
                 0 & 0 & 1 & 0 & D_x \\
                 0 & 0 & 0 & 1 & D_y \\
                 0 & 0 & 0 & 0 & 1 \\
 \end{pmatrix}
 \begin{pmatrix} x \\ y \\ d_x \\ d_y \\ 1 \end{pmatrix}\,.
\end{align}
Therefore we get
\begin{subequations}
\begin{align}
 x' &= x + t_x\,,\\
 y' &= y + t_y\,,\\
 {d'}_x &= d_x + D_x\,,\\
 {d'}_y &= d_y + D_y\,.
\end{align}
\end{subequations}
which corresponds to a absolute change in intersection points and ray direction.
The new direction vector also has to be a unit vector such that
\begin{align}
 \underbrace{{\Vector{d}}^{\prime 2}}_{=1} &= \underbrace{\Vector{d}^2}_{=1} + 2 \scpm{\Vector{d}}{\Vector{D}} + \Vector{D}^2\,,
\end{align}
gives a condition to the third component of $\Vector{D}$, namely
\begin{align}
 D_z &= -\sqrt{1 - d_x^2 - d_y^2} \pm \sqrt{1 - (d_x + D_x)^2 - (d_y + D_y)^2}\,.
\end{align}
Maybe this extension of the $XYUV$ (non linear) transfer formalism can be useful as a further generalisation of
the optical design calculation. By multiplying the direction vector with the optical index one immediately gets
a so called phase space formulation of the optical ray. In this formulation several theorems from Hamiltonian
mechanics and phase space theory in general hold. These formulation will be investigated in further subsections.

\subsection{\texorpdfstring{$PQ$}{PQ}-2D-Formalism}

By using the equations of motion \eqref{eq:H2Deom} for appropriate materials one can write down non linear transfers for
some special materials.

\subsubsection{Nonlinear transfer formalism}

Transfer through homogeneous material:
\begin{align}
 \begin{pmatrix}
  \Vector{Q} \\
  \Vector{P}
 \end{pmatrix} &\mapsto
 \begin{pmatrix}
  \Vector{Q} + (z' -z) \frac{\Vector{P}}{n \sqrt{1 - \frac{\Vector{P}^2}{n^2}}} \\
  \Vector{P}
 \end{pmatrix}\,.
\end{align}
Here $H_{\text{2D}}$ stays constant because the 3D momentum is conserved.
According to Ibn-Sal-Snell's law the coordinate change at surface on surface $z = f(x,y)$ at $(\Vector{Q} , f(\Vector{Q}))$ is given by
\begin{align}
 &\begin{pmatrix}
  \Vector{Q} \\
  \Vector{P}
 \end{pmatrix} \mapsto\nonumber\\&
 \begin{pmatrix}
  \Vector{Q} \\
  \Vector{P} - \left[\scpm{\Vector{P}}{\Vector{N}(\Vector{Q})} + Z(\Vector{Q}) H + \sqrt{n_2^2 - n_1^2 + [\scpm{\Vector{P}}{\Vector{N}(\Vector{Q})} + Z(\Vector{Q})H]^2}\right] \Vector{N}(\Vector{Q})
 \end{pmatrix}\,,
\end{align}
where $H = -n_1 \sqrt{1 - \Vector{P}^2/n^2}$, 
$\Vector{N}(\Vector{Q}) = (\text{grad} f)(\Vector{Q})/\sqrt{1 + (\text{grad} f)^2(\Vector{Q})}$, 
$Z(\Vector{Q}) = 1/\sqrt{1 + (\text{grad} f)^2(\Vector{Q})}$.
For a complete optical system one has to perform this transformations in a succesive manner on the starting configuration of the ray.
Due to the non-conservation of the Hamiltonian it will also change at the boundary between two layers in the following manner
\begin{align}
 H &\mapsto H + \left[\scpm{\Vector{P}}{\Vector{N}(\Vector{Q})} + Z(\Vector{Q}) H + \sqrt{n_2^2 - n_1^2 + [\scpm{\Vector{P}}{\Vector{N}(\Vector{Q})} + Z(\Vector{Q})H]^2}\right] Z(\Vector{Q})\,.
\end{align}



\subsubsection{Linear transfer formalism}

To get the paraxial approximation we need to expand the terms to linear order in $\Vector{P}$ (equivalent to a small angle approximation) and linear order in $\Vector{Q}$
(corresponding to near axis rays). An expansion only in $\Vector{Q}$ corresponds to near axis rays with finite angles and an $\Vector{P}$-only expansion corresponds to
small angle with finite field heights. The linearisation leads to a matrix formulation of the transfer functions.

For the transfer in a homogeneous material one gets
\begin{align}
 \begin{pmatrix}
  \Vector{Q} \\
  \Vector{P}
 \end{pmatrix} &\stackrel{\text{linear}}{\mapsto}
 \begin{pmatrix}
  \Vector{Q} + (z' -z) \frac{\Vector{P}}{n} \\
  \Vector{P}
 \end{pmatrix} =
 \begin{pmatrix}
  1 & \frac{z' - z}{n} \\
  0 & 1
 \end{pmatrix}
 \begin{pmatrix}
  \Vector{Q} \\
  \Vector{P}
 \end{pmatrix} 
\,.
\end{align}
For the refraction between two homogeneous materials $n_1$ and $n_2$ we get (Assume that the surface at $\Vector{Q}=0$ is not tilted and the surface
itself is a biconic with the two local curvatures in the respective coordinate directions $c_x$, $c_y$.)
\begin{align}
 \begin{pmatrix}
  \Vector{Q} \\
  \Vector{P}
 \end{pmatrix} &\stackrel{\text{linear}}{\mapsto}
 \begin{pmatrix}
  \Vector{Q} \\
  \Vector{P} + (n_1 - n_2) \begin{pmatrix} c_x & 0 \\ 0 & c_y \end{pmatrix}  \Vector{Q}
 \end{pmatrix} =
 \begin{pmatrix}
  1 & 0 & 0 & 0 \\
  0 & 1 & 0 & 0 \\
  (n_1 - n_2)c_x & 0 & 1 & 0   \\
  0 & (n_1 - n_2)c_y & 0 & 1  
 \end{pmatrix}
 \begin{pmatrix}
  \Vector{Q} \\
  \Vector{P}
 \end{pmatrix} 
\,.
\end{align}
Also the Hamitonian changes. This change is to first order in $\Vector{Q}$ and $\Vector{P}$ trivially
be given by
\begin{align}
 H_\text{2D} & \stackrel{\text{linear}}{\mapsto} -n_1 + (n_2 - n_1)\,.
\end{align}
But since a Hamiltonian in linear approximation is not sufficient to generate equations of motion,
one has to approximate at least to second order
\begin{align}
 \Delta H_\text{2D} &\stackrel{\text{quadratic}}{=} n_2-n_1
   +\frac{1}{2} \left(\frac{1}{n_1}-\frac{1}{n_2}\right) p_x^2
   + \frac{1}{2}  \left(\frac{1}{n_1}-\frac{1}{n_2}\right) p_y^2 \nonumber\\&
  -\frac{c_x^2 q_x^2 (n_1-n_2)^2}{2 n_2}+c_x p_x q_x \left(1-\frac{n_1}{n_2}\right)
   -\frac{c_y^2 q_y^2 (n_1-n_2)^2}{2 n_2}+c_y p_y q_y \left(1-\frac{n_1}{n_2}\right)
\,.
\end{align}


\subsection{\texorpdfstring{$pq$}{pq}-3D-Formalism}

The more general consideration of GRIN media in 3D phase space also fits into the formerly given transfer formalisms.
A ray starts at the point $(x,y,d_x,d_y)$ which means in a certain coordinate system
and a fixed $z$ position the ray starting point is $(x,y,z)$ and the direction vector
is $(d_x,d_y,\sqrt{1-d_x^2-d_y^2})$. This fixes the initial conditions of the ray propagation
uniquely. The 3D formulation is useful if the user parametrizes for another paramter then the distance $z$ on the optical
axis. In this case the Hamiltonian separates well in a kinetic and a potential term. Further the Hamiltonian is conserved and
equal zero.

\subsubsection{Nonlinear transfer}

Transfer through homogeneous medium to arc length $s$
\begin{align}
 \begin{pmatrix}
  \Vector{q} \\
  \Vector{p}
 \end{pmatrix} &\mapsto
 \begin{pmatrix}
  \Vector{q} + s \frac{\Vector{p}}{n} \\
  \Vector{p}
 \end{pmatrix}\,.
\end{align}
This transformation is linear.
Refraction at optical surface $f(x,y)$.
\begin{align}
 \begin{pmatrix}
  \Vector{q} \\
  \Vector{p}
 \end{pmatrix} &\mapsto
 \begin{pmatrix}
  \Vector{q} \\
  \Vector{p} - \left[\scpm{\Vector{p}}{\Vector{n}(\Vector{q})} + \sqrt{n_2^2 - n_1^2 + \scpm{\Vector{p}}{\Vector{n}(\Vector{q})}^2}\right] \Vector{n}(\Vector{q})
 \end{pmatrix}\,.
\end{align}
The Hamiltonian $H_\text{3D} = \Vector{p}^2 - n^2 = 0$ is conserved.
GRIN-Medium:
\begin{align}
 \begin{pmatrix}
  \Vector{q} \\
  \Vector{p}
 \end{pmatrix} &\stackrel{\text{ODE solver}}{\mapsto}
 \begin{pmatrix}
  \Vector{q}(s) \\
  \Vector{p}(s)
 \end{pmatrix}\,.
\end{align}


\subsubsection{Linearized transfer}
Only for refraction (this time also with tilted surface, where $\Vector{n}_0 = \Vector{n}(0)$ is the surface tilt vector):
\begin{align}
 \begin{pmatrix}
  \Vector{q} \\
  \Vector{p}
 \end{pmatrix} &\stackrel{\text{linear}}{\mapsto}
 \begin{pmatrix}
  \Vector{q} \\
  -\sqrt{n_2^2 - n_1^2} \Vector{n}_0 + \Vector{p} - \scpm{\Vector{p}}{\Vector{n}_0} \Vector{n}_0 - \sqrt{n_2^2 - n_1^2} [\scpm{\Vector{q}}{\Vector{\nabla}} \Vector{n}]_0
 \end{pmatrix}\nonumber\\&=
 \begin{pmatrix}
  1 & 0\\
  -\sqrt{n_2^2-n_1^2} (\Vector{\nabla} \otimes \Vector{n})_0 & 1 - \Vector{n}_0 \otimes \Vector{n}_0
 \end{pmatrix}
 \begin{pmatrix}
  \Vector{q} \\
  \Vector{p}
 \end{pmatrix} -
 \begin{pmatrix}
  0 \\
  \sqrt{n_2^2-n_1^2} \Vector{n}_0
 \end{pmatrix}
\,.
\end{align}
The absolute term which arises due to surface tilt may be absorbed into a homogenious formulation by using the 5x5 formalism mentioned above.

\subsection{Combined 2D 5x5 formalism}

We will start at a nonlinear homogeneous matrix transfer function of 3D variables and try to implement the formalisms discussed above
\begin{align}
 \begin{pmatrix}
  \Vector{q}\,{}'\\
  \Vector{p}\,{}'\\
  1
 \end{pmatrix}
 &=
 \underbrace{\begin{pmatrix}
  A(\Vector{q},\Vector{p}) & B(\Vector{q},\Vector{p}) & \Vector{t}(\Vector{q},\Vector{p})\\
  C(\Vector{q},\Vector{p}) & D(\Vector{q},\Vector{p}) & \Vector{\delta}(\Vector{q},\Vector{p})\\
  0 & 0 & 1 
 \end{pmatrix}}_\mathcal{T}
 \begin{pmatrix}
  \Vector{q}\\
  \Vector{p}\\
  1
 \end{pmatrix}\,.
\end{align}
As already discussed above this homogeneous transformation is able to include rotation and translation of the optical axis (aka coordinate breaks) 
in a unified manner. Due to the nonlinearity the formulation of the transfer matrix is not unique. The components of $\Vector{p}$ and $\Vector{p}\,{}'$
are not fully independent from each other. For unification with 2D formalism we split up the canonical coordinates and momenta into 2D and 3-components, i.e.
$\Vector{q} = (\Vector{Q}, z) = (Q_\alpha, z)$ where indices from the beginning of the greek alphabet are running over $1,2$. Now we also use Einstein summation
convention. Therefore the equation $\Vector{q}\,{}' = A \Vector{q}$ is $(2+1)$ decomposed like the following
\begin{align}
 Q^\prime_\alpha &= A_{\alpha\beta} Q_\beta + A_{\alpha 3} z\,,\\
 z^\prime &= A_{3 \beta} Q_\beta + A_{33} z\,.
\end{align}
For the readers convenience we omit the $(\Vector{q},\Vector{p})$ dependency of the sub matrices and vectors in the large transfer matrix.
In its full form the transfer reads ( $H$ and $H'$ are the respective 2D Hamiltonians corresponding to the negative 3-component of the 2D momentum)
\begin{subequations}
\label{eq:transfermatrixeqns}
\begin{align}
 Q^\prime_\alpha &= A_{\alpha\beta} Q_\beta + A_{\alpha 3} z + B_{\alpha\beta} P_\beta - B_{\alpha 3} H + t_\alpha\,,\\
 z^\prime &= A_{3 \beta} Q_\beta + A_{33} z + B_{3\beta} P_\beta - B_{33} H + t_3\,,\\
 P^\prime_\alpha &= C_{\alpha\beta} Q_\beta + C_{\alpha 3} z + D_{\alpha\beta} P_\beta - D_{\alpha 3} H + \delta_\alpha\,,\\
 H' &= -C_{3\beta} Q_\beta - C_{33} z - D_{3\beta} P_\beta + D_{33} H - \delta_3\,,
\end{align}
\end{subequations}
The transfer function in this form is only valid for non-GRIN media. I would suggest to treat the transfer function of a GRIN medium
like a box where you put in $Q_\alpha, P_\alpha, z$ and after the propagation to $z'$ it spits out the primed coordinates.
\subsubsection{Transfer matrix functions}
For transfer through material with constant optical index $n$ we have
\begin{align}
 \begin{pmatrix}
  \Vector{Q} \\
  \Vector{P}
 \end{pmatrix} &\mapsto
 \begin{pmatrix}
  \Vector{Q} + (z' -z) \frac{\Vector{P}}{n \sqrt{1 - \frac{\Vector{P}^2}{n^2}}} \\
  \Vector{P}
 \end{pmatrix}\,.
\end{align}
So \eqref{eq:transfermatrixeqns} gives
\begin{align}
 A_{\alpha\beta} &= \delta_{\alpha\beta}\,,\quad A_{\alpha 3} = 0\,,\quad B_{\alpha\beta} = \frac{L}{\sqrt{n^2 - \Vector{P}^2}} \delta_{\alpha\beta}\,,\quad B_{\alpha3} = 0\,,\quad t_\alpha = 0 \\
 A_{3\beta} &= 0\,,\quad A_{33} = 1\,,\quad B_{3\beta} = 0\,,\quad B_{33} = 0\,,\quad t_3 = L\,,\\
 D_{\alpha\beta} &= \delta_{\alpha\beta}\,,\quad D_{33} = 1\,,
\end{align}
all other zero. Therefore\remark{check!}
\begin{align}
 \mathcal{T}_n &=
 \begin{pmatrix}
  \delta_{\alpha\beta} & 0 &  \frac{L}{\sqrt{n^2 - \Vector{P}^2}} \delta_{\alpha\beta} & 0 & 0\\
   0                   & 1 & 0 & 0 & L\\
   0                   & 0 & \delta_{\alpha\beta} & 0 & 0\\
   0                   & 0 & 0 & 1 & 0 \\
   0 & 0 & 0 & 0 & 1
 \end{pmatrix}
\end{align}
For refraction\remark{check!}
\begin{align}
 \mathcal{T}_{n_1\mapsto n_2} &=
 \begin{pmatrix}
  \delta_{\alpha\beta} & 0 &  0 & 0 & 0\\
   0                   & 1 & 0 & 0 & 0\\
   0                   & 0 & \delta_{\alpha\beta} & Z(\Vector{Q}) N_\alpha & -\biggl[\scpm{\Vector{P}}{\Vector{N}} - \sqrt{n_2^2 - n_1^2 + (\scpm{\Vector{P}}{\Vector{N}} + Z(\Vector{Q}) H)^2}\biggr] N_\alpha \\
   0                   & 0 & 0 & 1 + Z(\Vector{Q})^2 & -\biggl[\scpm{\Vector{P}}{\Vector{N}} - \sqrt{n_2^2 - n_1^2 + (\scpm{\Vector{P}}{\Vector{N}} + Z(\Vector{Q}) H)^2}\biggr] Z(\Vector{Q}) \\
   0 & 0 & 0 & 0 & 1
 \end{pmatrix}
\end{align}
For reflection\remark{check!}
\begin{align}
 \mathcal{T}_{\text{mirror}} &=
 \begin{pmatrix}
  \delta_{\alpha\beta} & 0 &  0 & 0 & 0\\
   0                   & 1 & 0 & 0 & 0\\
   0                   & 0 & \delta_{\alpha\beta} & 2 Z(\Vector{Q}) N_\alpha(\Vector{Q}) & -2 \scpm{\Vector{P}}{\Vector{N}} N_\alpha(\Vector{Q}) \\
   0                   & 0 & 0 & 1 - 2 Z(\Vector{Q})^2 & 2 \scpm{\Vector{P}}{\Vector{N}} Z(\Vector{Q}) \\
   0 & 0 & 0 & 0 & 1
 \end{pmatrix}
\end{align}
For a proper treatment of refraction or reflection at a curved surface one has to start a ray in the reference plane of the 
optical surface and propagate up to the surface. The intersection calculation is not trivial and only for special cases
available in closed form. After this intersection calculation one may calculate the normal vector at the intersection point
and refract the ray.

For coordinate break (rotation of momentum direction)
\begin{align}
 \mathcal{T}_{\text{CB,pR}} &=
 \begin{pmatrix}
  \delta_{\alpha\beta} & 0 &  0 & 0 & 0\\
   0                   & 1 & 0 & 0 & 0\\
   0                   & 0 & R_{\alpha\beta} & R_{\alpha 3} & 0 \\
   0                   & 0 & R_{3\beta} & R_{33} & 0 \\
   0 & 0 & 0 & 0 & 1
 \end{pmatrix}
\end{align}
For coordinate break (translation of intersection point)
\begin{align}
 \mathcal{T}_{\text{CB,T}} &=
 \begin{pmatrix}
  \delta_{\alpha\beta} & 0 &  0 & 0 & t_\alpha\\
   0                   & 1 & 0 & 0 & t_3\\
   0                   & 0 & \delta_{\alpha\beta} & 0 & 0 \\
   0                   & 0 & 0 & 1 & 0 \\
   0 & 0 & 0 & 0 & 1
 \end{pmatrix}
\end{align}
For coordinate break (rotation of intersection point inplane)
\begin{align}
 \mathcal{T}_{\text{CB,QR}} &=
 \begin{pmatrix}
  R_{\alpha\beta} & 0 &  0 & 0 & 0\\
   0                   & 1 & 0 & 0 & 0\\
   0                   & 0 & \delta_{\alpha\beta} & 0 & 0 \\
   0                   & 0 & 0 & 1 & 0 \\
   0 & 0 & 0 & 0 & 1
 \end{pmatrix}
\end{align}
These transformations can be combined in a matrix manner since they
are linear and have no dependency from the coordinates and momentum variables.
The nonlinear transfer functions could also be combined in a matrix manner but
one has to take care about the right arguments to the components.


\subsubsection{Linearisation}

For a linearized theory one has to treat the coordinate part ($A$, $B$, $C$, $D$) in a different manner than the absolute part ($\Vector{t}$, $\Vector{\delta}$).
The former has to be approximated only to zeroth order since they are multiplied by momenta and coordinates. In contrast the latter one
has to be approximated at first order to generate missing linear terms. Therefore the structure of the linearized theory can still change, but
at least there is a unique formalism to derive the paraxial approximation.
\begin{align}
  \mathcal{T} &= \begin{pmatrix}
  A_{ab}(\Vector{q},\Vector{p}) & B_{ab}(\Vector{q},\Vector{p}) & t_{a}(\Vector{q},\Vector{p})\\
  C_{ab}(\Vector{q},\Vector{p}) & D_{ab}(\Vector{q},\Vector{p}) & \delta_{a}(\Vector{q},\Vector{p})\\
  0 & 0 & 1 
 \end{pmatrix}
 \approx \nonumber\\&
  \begin{pmatrix}
  A(0,0) & B(0,0) & t_a(0,0) + (\partial^q_b t_a)(0,0) q_b + (\partial^p_b t_a)(0,0) p_b\\
  C(0,0) & D(0,0) & \delta_a(0,0) + (\partial^q_b \delta_a)(0,0) q_b + (\partial^p_b \delta_a)(0,0) p_b\\
  0 & 0 & 1 
 \end{pmatrix} \nonumber\\&
  \begin{pmatrix}
  A_{ab}(0,0)  + (\partial^q_b t_a)(0,0) & B_{ab}(0,0)+ (\partial^p_b t_a)(0,0) & t_a(0,0) \\
  C_{ab}(0,0)+ (\partial^q_b \delta_a)(0,0) & D_{ab}(0,0)  + (\partial^p_b \delta_a)(0,0)& \delta_a(0,0)\\
  0 & 0 & 1 
 \end{pmatrix}
\end{align}
After a ($2+1$) decomposition it is clear how to transform the 2D components of momentum and coordinates. (
as mentioned earlier for the Hamiltonian, i.e. the negative 3-component of the momentum, it is not sufficient
to approximate to first order to generate the appropriate equations of motion. This means the $C_3$ and $D_3$ components
have to approximated at least to first order and the $\delta_3$ component up to second order.) Simple propagation in 2D is already linear.

Refraction:
\begin{align}
  \mathcal{T}_{n_1\mapsto n_2} \approx \begin{pmatrix}
  \delta_{ab} & 0 & 0 \\
   -\sqrt{n_2^2 - n_1^2} (\partial_b n_a)(0) & \delta_{ab} - n_a(0) n_b(0)& -\sqrt{n_2^2 - n_1^2} n_a(0)\\
  0 & 0 & 1 
 \end{pmatrix}
\end{align}
The first term second row depends on the curvature of the surface in the different directions. The second term is
a projection operator perpendicular to the unit normal vector of the surface at the optical axis.

Reflection:
\begin{align}
  \mathcal{T}_{\text{mirror}} \approx \begin{pmatrix}
  \delta_{ab} & 0 & 0 \\
   0 & \delta_{ab} - 2 n_a(0) n_b(0)& 0\\
  0 & 0 & 1 
 \end{pmatrix}
\end{align}
The higher order corrections in the taylor expansion give rise to the aberrations.

\subsection{Linearisation for non-zero interaction point with optical axis}
\newcommand{\dqv}{\Delta \Vector{q}}
\newcommand{\dpv}{\Delta \Vector{p}}

\newcommand{\dqi}[1]{\Delta q_{#1}}
\newcommand{\dpi}[1]{\Delta p_{#1}}


\newcommand{\qnv}{\Vector{q}_0}
\newcommand{\pnv}{\Vector{p}_0}

\newcommand{\qni}[1]{q_{0\,#1}}
\newcommand{\pni}[1]{p_{0\,#1}}
Let the optical axis at a certain surface start at $\Vector{q}_0$ and $\Vector{p}_0$ then (in 3D formalism)
the matrix deviation from it in linear approximation is given by
\begin{align}
  \mathcal{T} &= 
 \begin{pmatrix}
  A_{ab}(\qnv + \dqv,\pnv + \dpv) & B_{ab}(\qnv + \dqv,\pnv + \dpv) & t_{a}(\qnv + \dqv,\pnv + \dpv)\\
  C_{ab}(\qnv + \dqv,\pnv + \dpv) & D_{ab}(\qnv + \dqv,\pnv + \dpv) & \delta_{a}(\qnv + \dqv,\pnv + \dpv)\\
  0 & 0 & 1 
 \end{pmatrix}
\end{align}
A complete transformation is therefore split up into two parts
\begin{align}
 \begin{pmatrix}
  \qnv^\prime + \dqv^\prime\\
  \pnv^\prime + \dpv^\prime\\
  1
 \end{pmatrix} &=
  \mathcal{T}_{0} 
 \begin{pmatrix}
  \qnv \\
  \pnv \\
  1
 \end{pmatrix} +
  \Delta \mathcal{T}
 \begin{pmatrix}
  \dqv \\
  \dpv \\
  1
 \end{pmatrix}\,.
\end{align}
$\mathcal{T}_0$ is depending only on $\qnv$ and $\pnv$. It is obviously given by
\begin{align}
 \mathcal{T}_0 &= 
 \begin{pmatrix}
  A_{ab}(\qnv,\pnv) & B_{ab}(\qnv,\pnv) & t_{a}(\qnv,\pnv)\\
  C_{ab}(\qnv,\pnv) & D_{ab}(\qnv,\pnv) & \delta_{a}(\qnv,\pnv)\\
  0 & 0 & 1 
 \end{pmatrix}
\end{align}
This is the part transforming the optical axis or some kind of reference ray. The second part $\Delta \mathcal{T}$ comes from the transformation
of the linear deviation components with respect to the optical axis or the reference ray.
\begin{align}
 &\Delta \mathcal{T} = \nonumber\\&
 \begin{pmatrix}
  A_{ab} |_{0}+ \partial^q_b A_{ac}|_{0} \qni{c}  + \partial^q_b B_{ac} |_{0} \pni{c} + \partial^q_b t_a |_{0} &
   B_{ab} |_0 +  \partial^p_b A_{ac} |_{0} \qni{c} + \partial^p_b B_{ac} |_0 \pni{c} + \partial^p_b t_a |_0 & 0\\
  C_{ab} |_{0} + \partial^q_b C_{ac}|_{0} \qni{c} + \partial^q_b D_{ac} |_{0} \pni{c} + \partial^q_b \delta_a |_{0} &
   D_{ab} |_0 + \partial^p_b C_{ac} |_{0} \qni{c} + \partial^p_b D_{ac} |_0 \pni{c} + \partial^p_b \delta_a |_0 & 0\\
  0 & 0 & 0 
 \end{pmatrix}
\end{align}
Due to this split one may transform the optical axis in a first step and afterwards calculate the ``paraxial approximation'' of rays which
only deviate a little from the optical axis. All quantities have to be evaluated at the coordinates of the optical axis. The first terms
in the transfer matrix are the same as for the optical axis. All other terms (the derivatives) represent the deviations from it.
For an appropriate 5x5 treatment one has to decompose the matrix components further in a 2+1 manner.
\section{New coordinate convention with pilot ray}
\begin{enumerate}
 \item Defining optical surfaces in the usual manner (this means starting from object via thicknesses and coordinate breaks)
 \item Starting pilot ray at object position $(0, 0)$ direction $(0, 0, 1)$ (default values) at primary wavelength with real raytracing routine
 \item Intersection point between surface and pilot ray in correspoding surface coordinate system
 \item Ask surface for local Hessian matrix (to obtain local curvature)
\end{enumerate}
For a spherical surface between media $n_1$ and $n_2$ where $\alpha = n_1/n_2$ we consider a ray bundle
of finite incident angle $\phi$.
The local effective focal length is given by
\begin{align}
 EFFL &= R \sqrt{\frac{\left(1- \alpha ^2 \sin ^2(\phi )\right)^3}{\left(\alpha ^2 \sin ^2(\phi )+\alpha \cos (\phi ) \sqrt{1-\alpha ^2 \sin ^2(\phi )}-1\right)^2}}\,.
\end{align}
A comment: The curvature radius $R$ is given by the local curvature of a general surface and $\cos(\phi) = \scpm{\Vector{i}}{\Vector{n}}$
where $\Vector{n}$ is the local normal unit vector and $\Vector{i}$ is the local incident vector (oriented towards the surface).



\end{document}
