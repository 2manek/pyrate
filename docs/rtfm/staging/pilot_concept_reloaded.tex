\documentclass[12pt,a4paper,twoside,openright,BCOR10mm,headsepline,titlepage,abstracton,chapterprefix,final]{scrreprt}

\usepackage{ae}
\usepackage[ngerman, english]{babel}
%\usepackage{SIunits}

\usepackage{amsmath}
\usepackage{amssymb}
\usepackage{amsfonts}
\usepackage{xcolor}
\usepackage{setspace}
\usepackage{dsfont}

% load hyperref as the last package to avoid redefinitions of e.g. footnotes after hyperref invokation

\RequirePackage{ifpdf}  % flag for pdf or dvi backend
\ifpdf
  \usepackage[pdftex]{graphicx}
  \usepackage[pdftitle={RTFM on Imaging Theory and Basics of Optical Raytracing},%
              pdfsubject={},%
              pdfauthor={M. Esslinger, J. Hartung, U. Lippmann},%
              pdfkeywords={},%
              bookmarks=true,%
%              colorlinks=true,%
              urlcolor=blue,%
              pdfpagelayout=TwoColumnRight,%
              pdfpagemode=UseNone,%
              pdfstartview=Fit,%
	      pdfpagelabels,
              pdftex]{hyperref}
\else
  \usepackage[dvips]{graphicx}
  \usepackage[colorlinks=false,dvips]{hyperref}
\fi
%\DeclareGraphicsRule{.jpg}{eps}{.jpg}{`convert #1 eps:-}

\usepackage{ae}
%\usepackage[ngerman, english]{babel}

%\usepackage{SIunits}
\newcommand\elementarycharge{\textrm{e}}
\newcommand\sccm{\textrm{sccm}}
\newcommand\mbar{\milli\textrm{bar}}


\usepackage{amsmath}
%\usepackage{amssymb}
\usepackage{setspace}

%\widowpenalty = 1000


\newcommand*{\doi}[1]{\href{https://doi.org/\detokenize{#1}}{doi: \detokenize{#1}}}

\newcommand\Vector[1]{{\mathbf{#1}}}
%\newcommand\Vector[1]{{\vec{#1}}}

\newcommand\vacuum{0}

\newcommand\location{r}
\newcommand\Location{\Vector{r}}


\newcommand\wavenumber{k}
\newcommand\vacuumWavenumber{\wavenumber_{\vacuum}}
\newcommand\Wavevector{\Vector{\wavenumber}}

\newcommand\Nabla{\Vector{\nabla}}
\newcommand\Laplace{\Delta}
\newcommand\timederivative[1]{\dot{{#1}}}
\newcommand\Tensor[1]{\hat{#1}}
\newcommand\conjugate[1]{\overline{#1}}
\newcommand\transpose[1]{#1^{T}}
\newcommand\Norm[1]{\left| #1 \right|}
\newcommand{\ket}[1]{\left\vert{#1}\right\rangle}
\newcommand{\bra}[1]{\left\langle{#1}\right\vert}
\newcommand{\braket}[2]{\left\langle{#1}\vert{#2}\right\rangle}
\newcommand{\bracket}[1]{\left\langle{#1}\right\rangle}

\newcommand{\scpm}[2]{(#1\,\cdot\,#2)}

\newcommand\unittensor{\mathds{1}}

\newcommand\Greenfunction{\Tensor{G}}

\newcommand\scalarEfield{E}
\newcommand\scalarBfield{B}
\newcommand\scalarHfield{H}
\newcommand\scalarDfield{D}
\newcommand\Efield{\Vector{\scalarEfield}}
\newcommand\Bfield{\Vector{\scalarBfield}}
\newcommand\Hfield{\Vector{\scalarHfield}}
\newcommand\Dfield{\Vector{\scalarDfield}}

\newcommand\permeability{\Tensor{\scalarpermeability}}
\newcommand\vacuumpermeability{\scalarpermeability_{\vacuum}}
\newcommand\scalarpermeability{\mu}
\newcommand\scalarrelativepermeability{\mu_{rel}}
\newcommand\relativepermeability{\Tensor{\mu}_{rel}}

\newcommand\permittivity{\Tensor{\scalarpermittivity}}
\newcommand\vacuumpermittivity{\scalarpermittivity_{\vacuum}}
\newcommand\scalarrelativepermittivity{\epsilon}
\newcommand\relativepermittivity{\Tensor{\scalarrelativepermittivity}}
\newcommand\scalarpermittivity{\varepsilon}

\newcommand\conductivity{\Tensor{\sigma}}
\newcommand\susceptibility{\Tensor{\chi}}
\newcommand\currentdensity{\Vector{j}}
\newcommand\chargedensity{\rho}
\newcommand\PoyntingVector{\Vector{S}}

\newcommand\ordi{\text{ord}}
\newcommand\eo{\text{eo}}

\newcommand\materialone{I}
\newcommand\materialtwo{{II}}

\newcommand{\kpa}[1]{{\wavenumber_{\parallel#1}}}
\newcommand\tr{\text{tr}}

\newcommand{\timeavg}[1]{{\langle\,#1\,\rangle}}

\newcommand{\remark}[1]{{\color{red}$\blacksquare$}\footnote{{\color{red}#1}}}
\newcommand{\chk}[1]{\color{green}{$\checkmark$#1}}

\newcommand{\orderof}[1]{\mathcal{O}(#1)}

\newcommand\ppol{p}
\newcommand\spol{s}
\newcommand\normconst{N}

\newcommand\kilogram{\textrm{kg}}
\newcommand\meter{\textrm{m}}
\newcommand\second{\textrm{s}}
\newcommand\ampere{\textrm{A}}
\newcommand\volt{\textrm{V}}
\newcommand\watt{\textrm{W}}
\newcommand\tesla{\textrm{T}}

\newcommand\im{\textrm{im}}
\newcommand\obj{\textrm{obj}}
\newcommand\tot{\textrm{tot}}

\newcommand\totald{\textrm{d}}

\begin{document}

\section{Parabasal Optics} \label{sec:ParabasalOptics}

\subsection{Motivation}

The ABCD ansatz allows for very fast calculations and is invaluable in the pre-design stage.
Its simplicity, however, comes with the restriction to meridional rays in centro-symmetric systems and the need for an optical axis.

In systems like Offner systems or off-axis parabolic mirrors, the center-of-field chief ray is no more aligned with the axis of rotational symmetry of the surfaces.
In these systems, it is no longer sufficient to obtain first order properties from the central curvature on the axis of symmetry of each surface.
In other systems, like freeshape Schiefspiegler telescopes, there is no axis of symmetry at all. 
Anamorphotic systems, on the other hand, often have an optical axis, but do not have the same effective surface curvature in all meridional planes.

In this section, we develop a description for the first order properties of optical systems 
and describe rotationally symmetric and non-symmetric systems with a single formalism.
It is a formalism for rays close to the center-of-field chief ray, which we call \emph{pilot ray} in the following.
This pilot ray takes the role of the optical axis in rotationally symmetric systems. 
By choosing the term pilot ray, we want to avoid confusions with mechanical symmetry axes of sub-systems.
In many systems, the intial pilot ray direction is perpendicular to the object plane. 
An exception is found, for example, in Scheimpflug systems.

\subsection{Operator Ansatz}
Our ansatz is to perform a real ray trace of the pilot ray and describe a small area around it by a Taylor expansion around the real ray intersection point.
From this Taylor expansion, we derive first order properties, like focal length, magnification, aperture size, pupils, and image position.
The concept is a parabasal approximation around a non-paraxial pilot ray.
It is a generalization of the ABCD formalism. 
We describe the vectorial deviation from the pilot ray position and direction.
We consider near-pilot rays, that have an intersection point and a wavevector close to the pilot ray.

\begin{tabular}{ l | l | l }
		    & pilot ray & near-pilot ray \\
& & \\ \hline & & \\
intersection point  & $\Vector{r}_{pilot}$ & ${\Vector{r}} = \Vector{r}_{pilot} + \Delta\Vector{r}$ \\
wavevector  & $\Wavevector_{pilot}$ & ${\Wavevector} = \Wavevector_{pilot} + \Delta\Wavevector$ \\
ray direction  & $\Vector{d}_{pilot}$ & ${\Vector{d}} = \Vector{d}_{pilot} + \Delta\Vector{d}$ \\
surface normal      & $\Vector{n}_{pilot} = \Vector{n}(\Vector{r}_{pilot})$ & ${\Vector{n}} = \Vector{n}({\Vector{r}}) = \Vector{n}_{pilot} + \Delta\Vector{n}$ \\
\end{tabular}\\[2ex]
We represent the propagation of these deviations through the optical system by a propagation operator $\hat{G}$
\begin{eqnarray}
 \begin{pmatrix}
  \Delta \Vector{r}_\im \\ \Delta \Wavevector_\im
 \end{pmatrix}
 &=&
 \hat{G}
 \begin{pmatrix}
  \Delta \Vector{r}_\obj \\ \Delta \Wavevector_\obj
 \end{pmatrix}
\end{eqnarray}
We are interested in rays close to the pilot ray, $\Delta \Vector{r}_\obj, \Delta \Wavevector_\obj \rightarrow 0$,
so we approximate the operators by a Taylor expansion around the pilot ray
\begin{eqnarray}
 \begin{pmatrix}
  \Delta \Vector{r}_\im \\ \Delta \Wavevector_\im
 \end{pmatrix}_i
 &=&
   G^{00}_i
 + G^{10}_{ij} \Delta r_j
 + G^{01}_{ij} \Delta k_j
 + G^{11}_{ijm} \Delta r_j \Delta k_m
 + G^{20}_{ijm} \Delta r_j \Delta r_m
 + G^{02}_{ijm} \Delta k_j \Delta k_m
 + ...\nonumber\\
\end{eqnarray}
The two upper indices denote the order in $\Delta \Vector{r}$ and $\Delta \Vector{k}$, respectively.
By definition, the propagator $\hat{G}$ projects the object sided pilot ray onto the image sided pilot ray,
that is $\hat{G}(\Vector{0}) = \Vector{0}$ or $\hat{G}^{00} = \hat{0}$. 
In analogy to the ABCD formalism, we neglect terms quadratic or higher in the change of direction or position, 
$\orderof{\Delta\Wavevector^2}$ and $\orderof{\Delta\Vector{r}^2}$.
\begin{eqnarray}
 \begin{pmatrix}
  \Delta \Vector{r}_\im \\ \Delta \Wavevector_\im
 \end{pmatrix}_i
 &\approx&
   G^{10}_{ij} \Delta r_j
 + G^{01}_{ij} \Delta k_j
 + G^{11}_{ijm} \Delta r_j \Delta k_m
\end{eqnarray}
In contrast to the ABCD formalism, we do not drop the bilinear $\hat{G}^{11}$ term right away.
\remark{todo: write a more elegant explanation why all quadratic forms but the bilinear one are dropped.}
In the following, we discuss the representation of the operators by matrices, 
the rank of the matrices and the base of $\Delta \Vector{r}$ and $\Delta \Wavevector$.

\subsection{Near-Pilot Base Systems}

The deviations may not be chosen freely, but we need to choose
$\Delta\Vector{r}$, $\Delta\Wavevector$ and $\Delta\Vector{d}$ 
in a way so that both pilot and near-pilot rays
have valid properties.
Intersection points must lay on the surface,
wavevectors must obey the dispersion relation 
and ray direction vectors must be of unit length.
This limits the degrees of freedom for the deviations.

\subsubsection{Intersection points}

We demand that near-pilot intersection points are on the surface.
In parabasal approximation, we assume each surface planar in a small area around the pilot ray intersection point.
That is, we neglect surface curvature increasing quadratically with the deviation. 
We represent the ray height in this plane by
\begin{eqnarray}
 \Delta\Vector{r} = X \Vector{e}_{X} + Y \Vector{e}_{Y}
 \label{eq:deltaR_equals_XY}
\end{eqnarray}
where $\Vector{e}_{X}$ and $\Vector{e}_{Y}$ are in the tangential surface plane.
$\Vector{e}_{X,Y}$ shall be real valued unit vectors and orthogonal on each other.

\subsubsection{Wavevectors}
Both pilot and near-pilot wavevector must fulfill the dispersion relation. 
In contrast to the real valued ray directions, wavevectors may be complex valued.
Using \eqref{eq:perturbed_wavevector}, we determine the 4 complex valued unit vectors,
along which perturbed wavevectors still obey the dispersion relation.
We represent the change of the wavevector $\Wavevector$ as
\begin{eqnarray}
 \Delta \Wavevector &=& U \Vector{e}_U + V \Vector{e}_V + W \Vector{e}_W + Q \Vector{e}_Q
\end{eqnarray}
where $U,V,W,Q$ are real valued prefactors.
The complex valued unit vectors $\Vector{e}_{U,V,W,Q}$ shall be orthogonal on each other.

\subsection{XYUV matrices}
We choose the object sided base vectors 
$\Vector{e}_{X,Y,U,V,W,Q,\,obj}$ 
in a way that both pilot and near-pilot rays start in the object plane and obey the dispersion relation, 
as described in the previous section.
We trace the pilot ray to the final surface 
and define the near-pilot base vectors in the final plane
$\Vector{e}_{X,Y,U,V,W,Q,\,fin}$.
We split up the $\hat{G}$ operators into separate operators for location and direction.
\begin{eqnarray}
 \Delta r_{\im,\,i}  &=& A_{ij} \Delta r_{\obj,\,j} + B_{ij} \Delta k_{\obj,\,j} + M_{ijm} \Delta r_{\obj,\,j} \Delta k_{\obj,\,m} \\ 
 \Delta \wavenumber_{\im,\,i} &=& C_{ij} \Delta r_{\obj,\,j} + D_{ij} \Delta k_{\obj,\,j} + N_{ijm} \Delta r_{\obj,\,j} \Delta k_{\obj,\,m}
\end{eqnarray}
We name the linear operators $\hat{A}, \hat{B}, \hat{C}, \hat{D}$ in analogy to the ABCD formalism.
We name the bilinear operators $\hat{M}$ and $\hat{N}$.

We abbreviate $(X,Y)$ as a vector of location deviation $\Vector{R}$
and $(U,V,W,Q)$ as a vector of wavevector deviation $\Vector{K}$.
\begin{eqnarray}
\Vector{R} &=&
 \begin{pmatrix}
  X \\ Y
 \end{pmatrix}
 \\
 \Vector{K} &=&
 \begin{pmatrix}
  U \\ V \\ W \\ Q 
 \end{pmatrix}
\end{eqnarray}
The propagator equation becomes
\begin{subequations}
\begin{eqnarray}
 \Vector{R}_\im &=& \hat{A} \cdot \Vector{R}_\obj + \hat{B} \cdot \Vector{K}_\obj + \hat{M} \cdot \Vector{R}_\obj \cdot \Vector{K}_\obj\label{eq:xyuv_imaging_equation}\\
 \Vector{K}_\im &=& \hat{C} \cdot \Vector{R}_\obj + \hat{D} \cdot \Vector{K}_\obj + \hat{N} \cdot \Vector{R}_\obj \cdot \Vector{K}_\obj
\end{eqnarray}
\end{subequations}
the operators $\hat{A},\hat{B},\hat{C},\hat{D},\hat{M},\hat{N}$ are represented by the Jacobi matrices
\begin{align}
 A_{ij} &=
   \left.
     \frac{\partial R_{i\,im}}{\partial R_{j\,obj}} 
   \right|_{\Vector{K_\obj}=0}
   & (2\times2 \,\textrm{matrix})
 \\
 B_{ij} &=
   \left.
     \frac{\partial R_{i\,im}}{\partial K_{j\,obj}} 
   \right|_{\Vector{R_\obj}=0}
   & (2\times4 \,\textrm{matrix})
 \\
 C_{ij} &=
   \left.
     \frac{\partial K_{i\,im}}{\partial R_{j\,obj}} 
   \right|_{\Vector{K_\obj}=0}
   & (4\times2 \,\textrm{matrix})
 \\
 D_{ij} &=
   \left.
     \frac{\partial K_{i\,im}}{\partial K_{j\,obj}} 
   \right|_{\Vector{R_\obj}=0}
   & (4\times4 \,\textrm{matrix})
 \\
 M_{ijl} &=
     \frac{\partial^2 R_{i\,im}}{\partial R_{j\,obj}\partial K_{l\,obj}} 
   & (2\times2\times4 \,\textrm{matrix})
 \\
 N_{ijl} &=
     \frac{\partial^2 K_{i\,im}}{\partial R_{j\,obj}\partial K_{l\,obj}} 
   & (4\times2\times4 \,\textrm{matrix})
\end{align}


To analytically obtain matrices of the total propagation from object to image, 
it is easiest to concatenate the matrices of all atomic propagation, refraction, reflection, and coordinate transformation operations,
using \eqref{eq:xyuv_total_propagation}.
To obtain these atomic matrices, we perform Taylor expansions around the pilot ray.

For systems where an explicit calculation is not possible
-- or the operator suffers from laziness --
it can be approximated by differential quotients of real ray traces of rays with small but finite differences.
While this numerical approach is easy to implement, it may introduce additional inaccuracies or instabilities.

\subsection{Sequence of Propagators}
In case the path from object to image is built up from subsystems that are subsequently passed, 
and we know the propagator of each subsystem $\hat{G}_i$,
we can write
\begin{eqnarray}
 \begin{pmatrix}
  \Delta \Vector{r}_\im \\ \Delta \Wavevector_\im
 \end{pmatrix}
 &=&
 \hat{G}_N \cdot \hat{G}_{N-1} \cdot ... \cdot \hat{G}_3 \cdot \hat{G}_2 \cdot \hat{G}_1 \cdot \hat{G}_0 \cdot
 \begin{pmatrix}
  \Delta \Vector{r}_\obj \\ \Delta \Wavevector_\obj
 \end{pmatrix}
\end{eqnarray}
In general, the operators do not commutate. $\hat{G}_0$ is the propagator representing the subsystem adjacent to the object.
In the following, we consider the most simple case: the case of two subsystems.
\begin{eqnarray}
 \Delta \Vector{r}_2  &=& \hat{A}_1 \Delta \Vector{r}_1 + \hat{B}_1 \Delta \Vector{k}_1 + \hat{M}_1 \Delta \Vector{r}_1 \Delta \Vector{k}_1 \\ 
 \Delta \Wavevector_2 &=& \hat{C}_1 \Delta \Vector{r}_1 + \hat{D}_1 \Delta \Vector{k}_1 + \hat{N}_1 \Delta \Vector{r}_1 \Delta \Vector{k}_1 \\
 \Delta \Vector{r}_1  &=& \hat{A}_0 \Delta \Vector{r}_0 + \hat{B}_0 \Delta \Vector{k}_0 + \hat{M}_0 \Delta \Vector{r}_0 \Delta \Vector{k}_0 \\ 
 \Delta \Wavevector_1 &=& \hat{C}_0 \Delta \Vector{r}_0 + \hat{D}_0 \Delta \Vector{k}_0 + \hat{N}_0 \Delta \Vector{r}_0 \Delta \Vector{k}_0
\end{eqnarray}
We neglect all quadratic or higher order terms except for the bilinear term and obtain the total operators, 
propagating from surface $0$ to surface $2$:
\begin{subequations}
\begin{eqnarray}
 \Delta \Vector{r}_2  &=& \hat{A}_\tot \Delta \Vector{r}_0 + \hat{B}_\tot \Delta \Vector{k}_0 + \hat{M}_\tot \Delta \Vector{r}_0 \Delta \Vector{k}_0 \\ 
 \Delta \Wavevector_2 &=& \hat{C}_\tot \Delta \Vector{r}_0 + \hat{D}_\tot \Delta \Vector{k}_0 + \hat{N}_\tot \Delta \Vector{r}_0 \Delta \Vector{k}_0
\end{eqnarray}
\end{subequations}
with
\begin{subequations}\label{eq:xyuv_total_propagation}
\begin{eqnarray}
 \hat{A}_\tot &=& \hat{A}_1 \hat{A}_0 + \hat{B}_1 \hat{C}_0 \\
 \hat{B}_\tot &=& \hat{A}_1 \hat{B}_0 + \hat{B}_1 \hat{D}_0 \\
 \hat{C}_\tot &=& \hat{C}_1 \hat{A}_0 + \hat{D}_1 \hat{C}_0\\
 \hat{D}_\tot &=& \hat{C}_1 \hat{B}_0 + \hat{D}_1 \hat{D}_0 \\
 \hat{M}_\tot &=& \hat{A}_1 \hat{M}_0 + \hat{B}_1 \hat{N}_0 + \hat{M}_1 \hat{A}_0 \hat{D}_0 + \hat{M}_1 \hat{B}_0 \hat{C}_0 \\
 \hat{N}_\tot &=& \hat{C}_1 \hat{M}_0 + \hat{D}_1 \hat{N}_0 + \hat{N}_1 \hat{A}_0 \hat{D}_0 + \hat{N}_1 \hat{B}_0 \hat{C}_0
\end{eqnarray}
 \end{subequations}
 
\remark{todo: explicit, component wise formula}



\subsection{Inverse Propagation}
Often, we know the propagation operator from plane 0 to plane 1, and we want to know the inverse operator,
propagating from plane 1 back to plane 0.
\begin{eqnarray}
 \begin{pmatrix}
  \Delta \Vector{r}_{0} \\ \Delta \Wavevector_{0}
 \end{pmatrix}
 &=&
 \hat{G}_{\text{inv}}
 \begin{pmatrix}
  \Delta \Vector{r}_{1} \\ \Delta \Wavevector_{1}
 \end{pmatrix}
\end{eqnarray}
To construct the inverse propagation form the forward propagator,
we consider a sequence of forward and backward propagation.
After applying both, we should end up where we started.
\begin{eqnarray}
 \begin{pmatrix}
  \Delta \Vector{r}_{0} \\ \Delta \Wavevector_{0}
 \end{pmatrix}
 &=&
 \hat{G}_{\text{inv}}
 \hat{G}
 \begin{pmatrix}
  \Delta \Vector{r}_{0} \\ \Delta \Wavevector_{0}
 \end{pmatrix}
 =  
 \hat{G}_{tot}
  \begin{pmatrix}
  \Delta \Vector{r}_{0} \\ \Delta \Wavevector_{0}
 \end{pmatrix}
\end{eqnarray}
We represent the operators $\hat{G}, \hat{G}_{\text{inv}}$ and $\hat{G}_{tot}$ by matrices $\hat{A},\hat{B},\hat{C},\hat{D},\hat{M},\hat{N}$.
The matrices representing the \emph{inverse propagator} are marked with an index $_\text{inv}$.
In contrast, an upper $^{-1}$ denotes the \emph{inverse matrix}.
We apply the propagation concatenation equation \eqref{eq:xyuv_total_propagation} on our sequence of forward and inverse propagation and yield:
\begin{subequations}
\begin{eqnarray}
 \hat{A}_\tot &= \unittensor =& \hat{A}_{\text{inv}} \hat{A} + \hat{B}_{\text{inv}} \hat{C} \\
 \hat{B}_\tot &= 0           =& \hat{A}_{\text{inv}} \hat{B} + \hat{B}_{\text{inv}} \hat{D} \\
 \hat{C}_\tot &= 0           =& \hat{C}_{\text{inv}} \hat{A} + \hat{D}_{\text{inv}} \hat{C}\\
 \hat{D}_\tot &= \unittensor =& \hat{C}_{\text{inv}} \hat{B} + \hat{D}_{\text{inv}} \hat{D} \\
 \hat{M}_\tot &= 0           =& \hat{A}_{\text{inv}} \hat{M} + \hat{B}_{\text{inv}} \hat{N} + \hat{M}_{\text{inv}} \hat{A} \hat{D} + \hat{M}_{\text{inv}} \hat{B} \hat{C} \\
 \hat{N}_\tot &= 0           =& \hat{C}_{\text{inv}} \hat{M} + \hat{D}_{\text{inv}} \hat{N} + \hat{N}_{\text{inv}} \hat{A} \hat{D} + \hat{N}_{\text{inv}} \hat{B} \hat{C}
\end{eqnarray}
\end{subequations}
We consider the subsystem of the first four equations and solve it for the inverse propagation matrices.
\begin{eqnarray}
 \begin{pmatrix}
  \hat{A}_{\text{inv}} & \hat{B}_{\text{inv}} \\ \hat{C}_{\text{inv}} & \hat{D}_{\text{inv}}
 \end{pmatrix}
 &=&
 \begin{pmatrix}
  \hat{A} & \hat{B} \\ \hat{C} & \hat{D}
 \end{pmatrix}^{-1}
\end{eqnarray}
or, explicitly,
\begin{eqnarray}
 \begin{pmatrix}
  \hat{A}_{\text{inv}} & \hat{B}_{\text{inv}} \\ \hat{C}_{\text{inv}} & \hat{D}_{\text{inv}}
 \end{pmatrix}
 &=&
 \begin{pmatrix}
  A_{11\,\text{inv}} & A_{12\,\text{inv}} & B_{11\,\text{inv}} & B_{12\,\text{inv}} & B_{13\,\text{inv}} & B_{14\,\text{inv}} \\
  A_{21\,\text{inv}} & A_{22\,\text{inv}} & B_{21\,\text{inv}} & B_{22\,\text{inv}} & B_{23\,\text{inv}} & B_{24\,\text{inv}} \\
  C_{11\,\text{inv}} & C_{12\,\text{inv}} & D_{11\,\text{inv}} & D_{12\,\text{inv}} & D_{13\,\text{inv}} & D_{14\,\text{inv}} \\
  C_{21\,\text{inv}} & C_{22\,\text{inv}} & D_{21\,\text{inv}} & D_{22\,\text{inv}} & D_{23\,\text{inv}} & D_{24\,\text{inv}} \\
  C_{31\,\text{inv}} & C_{32\,\text{inv}} & D_{31\,\text{inv}} & D_{32\,\text{inv}} & D_{33\,\text{inv}} & D_{34\,\text{inv}} \\
  C_{41\,\text{inv}} & C_{42\,\text{inv}} & D_{41\,\text{inv}} & D_{42\,\text{inv}} & D_{43\,\text{inv}} & D_{44\,\text{inv}}
 \end{pmatrix}
 \\
 &=&
 \begin{pmatrix}
  A_{11} & A_{12} & B_{11} & B_{12} & B_{13} & B_{14} \\
  A_{21} & A_{22} & B_{21} & B_{22} & B_{23} & B_{24} \\
  C_{11} & C_{12} & D_{11} & D_{12} & D_{13} & D_{14} \\
  C_{21} & C_{22} & D_{21} & D_{22} & D_{23} & D_{24} \\
  C_{31} & C_{32} & D_{31} & D_{32} & D_{33} & D_{34} \\
  C_{41} & C_{42} & D_{41} & D_{42} & D_{43} & D_{44}
 \end{pmatrix}^{-1}
\end{eqnarray}
The inverse bilinear matrices read
\begin{subequations}
\begin{eqnarray}
 \hat{M}_{\text{inv}} &=& - (\hat{A}_{\text{inv}} \hat{M} + \hat{B}_{\text{inv}} \hat{N}) \cdot \left( ( \hat{A} \hat{D} + \hat{B} \hat{C} )^{-1} \right)\\
 \hat{N}_{\text{inv}} &=& - (\hat{C}_{\text{inv}} \hat{M} + \hat{D}_{\text{inv}} \hat{N}) \cdot \left( ( \hat{A} \hat{D} + \hat{B} \hat{C} )^{-1} \right)
\end{eqnarray}
\end{subequations}

\remark{todo: explicit, component wise formula}

\subsection{Propagation in Homogeneous Media}
We consider the propagation between two planes in a homogeneous medium.
In the homogeneous medium, rays are straight lines and the wavevector of each ray is the same in both planes.
We assume the coordinate system to describe wavevector changes is the same in both plane 0 and plane 1.
\begin{eqnarray}
C_{ij} &=& 0 \\
D_{ij} &=& \delta_{ij} \\
N_{ij\ell} &=& 0 \\
\text{for}\qquad \Vector{e}_{U,V,W,Q\,1} &=& \Vector{e}_{U,V,W,Q\,0} \nonumber
\end{eqnarray}
The task to determine $\hat{A},\hat{B}, \hat{M}$ remains. 
We consider the ray position in both planes without any Taylor approximation
\begin{align}
 \text{plane 0:} && \Location_0 &= \Location_{0\,\text{pilot}} + X_0 \Vector{e}_{X0} + Y_0 \Vector{e}_{Y0} \\
 \text{plane 1:} && \Location_1 &= \Location_{1\,\text{pilot}} + X_1 \Vector{e}_{X1} + Y_1 \Vector{e}_{Y1} \\
 \text{ray :} && \Location_1 &= \Location_{0} + s \cdot \Vector{d}
\end{align}
where $s$ is the geometric path length, and $\Vector{d}$ is the direction unit vector of the ray.
\begin{eqnarray}
\Location_{1\,\text{pilot}} + X_1 \Vector{e}_{X1} + Y_1 \Vector{e}_{Y1} &=& \Location_{0\,\text{pilot}} + X_0 \Vector{e}_{X0} + Y_0 \Vector{e}_{Y0}  + s \cdot \Vector{d}
\end{eqnarray}

\begin{eqnarray}
  X_1 \Vector{e}_{X1} + Y_1 \Vector{e}_{Y1} 
  &=& 
  \Location_{0\,\text{pilot}} - \Location_{1\,\text{pilot}} + X_0 \Vector{e}_{X0} + Y_0 \Vector{e}_{Y0}  + (s_{\text{pilot}} + \Delta s) ( \Vector{d}_{\text{pilot}} + \Delta \Vector{d} )
\\
  &=& 
  \Location_{0\,\text{pilot}} - \Location_{1\,\text{pilot}} + X_0 \Vector{e}_{X0} + Y_0 \Vector{e}_{Y0}  
  + s_{\text{pilot}} \Vector{d}_{\text{pilot}} 
  + s_{\text{pilot}} \Delta \Vector{d} 
  + \Delta s \Vector{d}_{\text{pilot}}
  + \Delta s \Delta \Vector{d}
\\
  &=& 
  X_0 \Vector{e}_{X0} + Y_0 \Vector{e}_{Y0}  
  + s_{\text{pilot}} \Delta \Vector{d} 
  + \Delta s \Vector{d}_{\text{pilot}}
  + \Delta s \Delta \Vector{d}
\end{eqnarray}
We build a cartesian coordinate system from $\Vector{e}_{X1}, \Vector{e}_{Y1}$ and $\Vector{e}_{Z1} = \Vector{e}_{X1} \times \Vector{e}_{Y1}$.
The unit vector $\Vector{e}_{Z1}$ points out of plane 1. We represent all vectors in this base

\begin{eqnarray}
  X_1 \Vector{e}_{X1} + Y_1 \Vector{e}_{Y1} 
  =&& 
      X_0 ( (\Vector{e}_{X0} \Vector{e}_{X1}) \Vector{e}_{X1} + (\Vector{e}_{X0} \Vector{e}_{Y1}) \Vector{e}_{Y1} + (\Vector{e}_{X0} \Vector{e}_{Z1} ) \Vector{e}_{Z1}) \nonumber \\
  &+& Y_0 ( (\Vector{e}_{Y0} \Vector{e}_{X1}) \Vector{e}_{X1} + (\Vector{e}_{Y0} \Vector{e}_{Y1}) \Vector{e}_{Y1} + (\Vector{e}_{Y0} \Vector{e}_{Z1} ) \Vector{e}_{Z1}) \nonumber \\  
  &+& (s_{\text{pilot}}+ \Delta s )( (\Delta \Vector{d} \Vector{e}_{X1}) \Vector{e}_{X1} + (\Delta \Vector{d} \Vector{e}_{Y1}) \Vector{e}_{Y1} + (\Delta \Vector{d} \Vector{e}_{Z1} ) \Vector{e}_{Z1}) \nonumber \\  
  &+& \Delta s ( (\Vector{d}_{\text{pilot}} \Vector{e}_{X1}) \Vector{e}_{X1} + (\Vector{d}_{\text{pilot}} \Vector{e}_{Y1}) \Vector{e}_{Y1} + (\Vector{d}_{\text{pilot}} \Vector{e}_{Z1} ) \Vector{e}_{Z1})  
\end{eqnarray}
and compare each coordinate
\begin{eqnarray}
  X_1 &=& X_0 (\Vector{e}_{X0} \Vector{e}_{X1}) + Y_0 (\Vector{e}_{Y0} \Vector{e}_{X1}) + (s_{\text{pilot}}+ \Delta s )(\Delta \Vector{d} \Vector{e}_{X1}) + \Delta s (\Vector{d}_{\text{pilot}} \Vector{e}_{X1}) \\
  Y_1 &=& X_0 (\Vector{e}_{X0} \Vector{e}_{Y1}) + Y_0 (\Vector{e}_{Y0} \Vector{e}_{Y1}) + (s_{\text{pilot}}+ \Delta s )(\Delta \Vector{d} \Vector{e}_{Y1}) + \Delta s (\Vector{d}_{\text{pilot}} \Vector{e}_{Y1}) \\
  0   &=& X_0 (\Vector{e}_{X0} \Vector{e}_{Z1}) + Y_0 (\Vector{e}_{Y0} \Vector{e}_{Z1}) + (s_{\text{pilot}}+ \Delta s )(\Delta \Vector{d} \Vector{e}_{Z1}) + \Delta s (\Vector{d}_{\text{pilot}} \Vector{e}_{Z1} )
\end{eqnarray}
The result is
\begin{subequations}\label{eq:propagation_between_planes_exact_formula}
\begin{eqnarray}
  X_1      &=& \left( X_0 \Vector{e}_{X0} + Y_0 \Vector{e}_{Y0} + s_{\text{pilot}}\Delta \Vector{d} + \Delta s \Delta \Vector{d} + \Delta s \Vector{d}_{\text{pilot}} \right) \Vector{e}_{X1} 
  \\
  Y_1      &=& \left( X_0 \Vector{e}_{X0} + Y_0 \Vector{e}_{Y0} + s_{\text{pilot}}\Delta \Vector{d} + \Delta s \Delta \Vector{d} + \Delta s \Vector{d}_{\text{pilot}} \right) \Vector{e}_{Y1} 
\end{eqnarray}
\end{subequations}
with
\begin{eqnarray}
  \Delta s &=& -\frac{ \left( X_0 \Vector{e}_{X0} + Y_0 \Vector{e}_{Y0} + s_{\text{pilot}} \Delta \Vector{d} \right) \Vector{e}_{Z1}}
                     { \left( \Vector{d}_{\text{pilot}} + \Delta \Vector{d} \right)\Vector{e}_{Z1}  }
\end{eqnarray}
This is the exact equation for propagation between planes without any approximations.
To fit this result to the XYUV formalism, we approximate in first order
\begin{eqnarray}
  \Delta s &\approx& - \frac{ X_0 e_{X0\,i}e_{Z1\,i} + Y_0 e_{Y0\,i}e_{Z1\,i} + s_{\text{pilot}} \Delta d_i e_{Z1\,i}  }
                            {d_{\text{pilot}\,j} e_{Z1\,j}}
                            \cdot
                            \left(
                            1 - \frac{ \Delta d_{\ell} e_{Z1\,\ell}}{d_{\text{pilot}\,j} e_{Z1\,j}}
                            \right)
\end{eqnarray}
and further
\begin{eqnarray}
  \Delta s \approx &-& \frac{  e_{X0\,i}e_{Z1\,i} }{d_{\text{pilot}\,j} e_{Z1\,j}} X_0 \nonumber\\
                   &-& \frac{  e_{Y0\,i}e_{Z1\,i} }{d_{\text{pilot}\,j} e_{Z1\,j}} Y_0 \nonumber\\
                   &-& \frac{ s_{\text{pilot}}   }{d_{\text{pilot}\,j} e_{Z1\,j}}e_{Z1\,i} \Delta d_i \nonumber\\
                   &+& \frac{ e_{X0\,i}e_{Z1\,i} }{ (d_{\text{pilot}\,j} e_{Z1\,j} )^2 }
                             e_{Z1\,\ell} \Delta d_{\ell} X_0  \nonumber\\
                   &+& \frac{ e_{Y0\,i}e_{Z1\,i} }{ (d_{\text{pilot}\,j} e_{Z1\,j} )^2 }
                             e_{Z1\,\ell} \Delta d_{\ell} Y_0 
\end{eqnarray}
We model the ray direction change in first order as
\begin{eqnarray}
 \Delta \Vector{d} &\approx& \frac{\totald \Vector{d}}{\totald \alpha} \cdot \alpha
\end{eqnarray}
with $\alpha \in \{ U,V,W,Q \}$. Now we can directly read the XYUV matrices from \eqref{eq:propagation_between_planes_exact_formula}.
Additionally, we copy the matrices $\hat{C},\hat{D},\hat{N}$ from above.
\begin{subequations}
\begin{eqnarray}
 \hat{A} &=& a_{j\ell}
 \begin{pmatrix}
  e_{X0j} e_{X1\ell} & e_{Y0j} e_{X1\ell} \\
  e_{X0j} e_{Y1\ell} & e_{Y0j} e_{Y1\ell}
 \end{pmatrix}
\\
 \hat{B} &=& s_{\text{pilot}} a_{j\ell}
 \begin{pmatrix}
 \frac{\totald d_j}{\totald U} e_{X1\ell} & \frac{\totald d_j}{\totald V} e_{X1\ell} & \frac{\totald d_j}{\totald W} e_{X1\ell} & \frac{\totald d_j}{\totald Q} e_{X1\ell} \\
 \frac{\totald d_j}{\totald U} e_{Y1\ell} & \frac{\totald d_j}{\totald V} e_{Y1\ell} & \frac{\totald d_j}{\totald W} e_{Y1\ell} & \frac{\totald d_j}{\totald Q} e_{Y1\ell}
 \end{pmatrix}
\\
\hat{C} &=& 0 \\
\hat{D} &=& \unittensor \\
M_{111} &=& m_{i\ell m} \frac{\totald d_{m}}{\totald U} e_{X0\,i} e_{X1\ell}  \\
\hat{N} &=& 0
\end{eqnarray}
with
\begin{eqnarray}
 \Vector{e}_{U,V,W,Q\,1} &=& \Vector{e}_{U,V,W,Q\,0} 
 \\
 a_{j\ell} &=& \delta_{j\ell} -  \frac{e_{Z1j} d_{\text{pilot}\ell} }{e_{Z1i} d_{\text{pilot}i}}
 \\
 m_{i\ell m} &=& \left(   - \frac{ e_{Z1\,i} }{d_{\text{pilot}\,j} e_{Z1\,j}} \delta_{\ell m} 
                  +   \frac{ e_{Z1\,i} }{ (d_{\text{pilot}\,j} e_{Z1\,j} )^2 } e_{Z1\,m}  d_{\text{pilot}\ell} 
           \right)
\end{eqnarray}
\end{subequations}


\subsection{Refraction}
We assume two parallel planes 1,2 just in front of and behind a refracting surface.
The planes shall be tangential to the surface 
and shall have identical bases $\Vector{e}_{X,Y}$.
The infinitessimal propagation length through the surface does not change the ray height
\begin{eqnarray}
 \hat{A} &=& \unittensor \\
 \hat{B} &=& 0 \\
 \hat{M} &=& 0
\end{eqnarray}
Deviations from the tangential plane (sag differences due to curvature) 
are an effect quadratic in ray height and are neglected in this linear formalism.
For the remaining matrices, we consider the wavevector.
\begin{eqnarray}
 \Wavevector_2 &=& \Wavevector_{\parallel2} + \xi_2 \Vector{n} \\
               &=& \Wavevector_{\parallel2} + (\xi_{pilot 2} + \Delta \xi) ( \Vector{n}_{pilot} + \Delta \Vector{n})
\end{eqnarray}
with
\begin{eqnarray}
 \Wavevector_{\parallel2} 
 &=& \Wavevector_{\parallel1} 
\\
 &=& \Wavevector_1 - (\Wavevector_1 \Vector{n}) \Vector{n} 
\\
 \wavenumber_{\parallel2i}
 &=& (\wavenumber_{pilot1i} + \Delta \wavenumber_{1i}) - ( \wavenumber_{pilot1j} + \Delta \wavenumber_{1j} )( n_{pilotj} + \Delta n_j )( n_{piloti} + \Delta n_i )
%\\
% &=& \wavenumber_{pilot1i} - \wavenumber_{pilot1j}   n_{pilotj} n_{piloti}\nonumber \\
% &&+ (\delta_{ij} -  n_{piloti} n_{pilotj}) \Delta \wavenumber_{1j}\nonumber\\
% &&+ (- \wavenumber_{pilot1j}    n_{piloti} - \wavenumber_{pilot1\ell}   n_{pilot\ell} \delta_{ij}) \Delta n_j \nonumber\\
% &&+( - n_{piloti} \delta_{j\ell} - n_{pilotj} \delta_{i\ell}) \Delta \wavenumber_{1j} \Delta n_\ell\nonumber\\
% &&- \wavenumber_{pilot1j}   \Delta n_j \Delta n_i - \Delta \wavenumber_{1j} \Delta n_j \Delta n_i
\end{eqnarray}
We factor out all terms and yield
\begin{align}
 \wavenumber_{2i} =&& \wavenumber_{pilot 2i} \nonumber \\
 &+& (\xi_{pilot 2i} \delta_{ij} - \wavenumber_{pilot1j}    n_{piloti} - \wavenumber_{pilot1\ell}   n_{pilot\ell} \delta_{ij}) &\Delta n_j \nonumber\\
 &+& (\delta_{ij} -  n_{piloti} n_{pilotj}) &\Delta \wavenumber_{1j}\nonumber\\
 &+& n_{pilot i} &\Delta \xi_i \nonumber\\
 &+& ( - n_{piloti} \delta_{j\ell} - n_{pilotj} \delta_{i\ell}) &\Delta \wavenumber_{1j} \Delta n_\ell\nonumber\\
 &+& &\Delta n_i \Delta \xi_i \nonumber\\
 &-& \wavenumber_{pilot1j}  & \Delta n_j \Delta n_i \nonumber\\
 &-& &\Delta \wavenumber_{1j} \Delta n_j \Delta n_i
\end{align}
In the next steps, we express $\Delta \Vector{n}$ and $\Delta \xi$ in terms of $\Delta \Vector{r}$ and $\Delta \Wavevector$.


\chapter{The Staging Attic}

\subsection{Inversion and Backward Propagation - Perturbation Ansatz}

\remark{How to invert with dr dk term?}

\newcommand{\dr}[2]{\Delta {r}_{#1\,#2}}
\newcommand{\dk}[2]{\Delta {k}_{#1\,#2}}

Starting point:

\begin{eqnarray}
 \dr{\text{im}}{i} & = A_{ij} \dr{\text{obj}}{j} + B_{ij} \dk{\text{obj}}{j} + M_{ij\ell} \dr{\text{obj}}{j} \dk{\text{obj}}{\ell}\,,\\
 \dk{\text{im}}{i} & = C_{ij} \dr{\text{obj}}{j} + D_{ij} \dk{\text{obj}}{j} + N_{ij\ell} \dr{\text{obj}}{j} \dk{\text{obj}}{\ell}\,.
\end{eqnarray}

First case:

Second case:

Perturbative inversion:

\begin{eqnarray}
 \dr{2}{i} & = A_{ij} \dr{1}{j} + B_{ij} \dk{1}{j} + M_{ij\ell} \dr{1}{j} \dk{1}{\ell}\,,\\
 \dk{2}{i} & = C_{ij} \dr{1}{j} + D_{ij} \dk{1}{j} + N_{ij\ell} \dr{2}{j} \dk{1}{\ell}\,.
\end{eqnarray}

\begin{eqnarray}
 \dr{1}{i} & = \dr{(0)}{i} + \alpha \dr{(1)}{i}\\
 \dk{1}{i} & = \dk{(0)}{i} + \alpha \dk{(1)}{i}\,.
\end{eqnarray}

\begin{eqnarray}
 \dr{2}{i} & = A_{ij} \dr{(0)}{j} + \alpha A_{ij} \dr{(1)}{j} + B_{ij} \dk{(0)}{j} + \alpha B_{ij}\dk{(1)}{j}\nonumber\\
  & + M_{ij\ell} (\dr{(0)}{j} + \alpha \dr{(1)}{j}) (\dk{(0)}{\ell} + \alpha \dk{(1)}{\ell})\,,\\
 \dk{2}{i} & = C_{ij} \dr{(0)}{j} + \alpha C_{ij} \dr{(1)}{i} + D_{ij} \dk{(0)}{j} + \alpha D_{ij}\dk{(1)}{j} + N_{ij\ell} \dr{1}{j} \dk{1}{\ell}\,.
\end{eqnarray}

The zeroth order also has to be inverted for being able to perform the backward propagation.


\subsection{Propagation Between Planes in Homogeneous Media (contains errors)}
We consider two planes 1 and 2.
We consider a homogeneous medium, a pilot ray $( \Vector{r}_{pilot1}, \Vector{r}_{pilot2}, \Wavevector_{pilot} )$
and a near pilot ray $( \Vector{r}_{1}, \Vector{r}_{2}, \Wavevector )$. 
In the homogeneous medium, the rays are straight lines and the wavevector of each ray is the same in both planes.
We describe the rays as lines with geometric path length $s$ and a unit direction vector $\Vector{d}$
\begin{eqnarray}
 \Vector{r}_{2} &=& \Vector{r}_{1} + \Vector{d} s  \\
 \Vector{r}_{pilot2} &=& \Vector{r}_{pilot1} + \Vector{d}_{pilot} s_{pilot}
\end{eqnarray}
Next we consider some special cases, that allow us to fix the components of the propagator step by step.
\begin{eqnarray}
 \Delta \Vector{r}_2 &=& \Vector{r}_{2} - \Vector{r}_{pilot2} \\
 \Delta \Vector{r}_2 &=& \Delta \Vector{r}_1 + \Delta s \Vector{d}_{pilot} \\
 \Delta \Vector{r}_2 &=& \Vector{r}_{1} + \Vector{d} s  - \Vector{r}_{pilot1} - \Vector{d}_{pilot} s_{pilot}
\end{eqnarray}

\subsubsection{The A-Matrix (contains errors)}
We consider a ray parallel to the pilot ($\Vector{K} = \Vector{0}$ and $\Vector{d} = \Vector{d}_{pilot}$).
\begin{eqnarray}
 \Delta \Vector{r}_2 &=& \Vector{r}_{1} - \Vector{r}_{pilot1} + \Vector{d} ( s - s_{pilot} ) \\
 X_2 \Vector{e}_{X2} +  Y_2 \Vector{e}_{Y2} - \Delta s \Vector{d}_{pilot} &=&  X_1 \Vector{e}_{X1} +  Y_1 \Vector{e}_{Y1} 
\end{eqnarray}
The three unknowns $X_2 , Y_2, \Delta s$ can be calculated from this 3-dimensional vector equation.
One possibility to do so is to employ a cartesian base $(x,y,z)$ and write
\begin{eqnarray}
 \begin{pmatrix}
  X_2 \\ Y_2 \\ \Delta s
 \end{pmatrix}
 &=&
 \begin{pmatrix}
  e_{X2x} & e_{Y2x} & d_{pilotx} \\
  e_{X2y} & e_{Y2y} & d_{piloty} \\
  e_{X2z} & e_{Y2z} & d_{pilotz}  
 \end{pmatrix}^{-1}
 \cdot
 \left(
 X_1 \Vector{e}_{X1} + Y_1 \Vector{e}_{Y1}
 \right)
\end{eqnarray}
From which we extract the coefficients of the $\hat{A}$ matrix, 
linking $(X_1,Y_1)$ to $(X_2,Y_2)$.

\subsubsection{The B-Matrix (contains errors)}
We consider a ray with the same starting point as the pilot, $\Vector{R} = \Vector{0}$ or $\Vector{r}_{1} = \Vector{r}_{pilot1}$. 
The remaining independent points $\Vector{r}_{1}, \Vector{r}_{2}, \Vector{r}_{pilot2}$ form a 2D triangle in 3D space.
The triangle has the sides $\Vector{d} s ; \Vector{d}_{pilot} s_{pilot}$ and $\Delta \Vector{r}_2$. 
Further we know that $\Delta \Vector{r}_2$ is in the image plane.

\begin{eqnarray}
  \Delta \Vector{r}_2  &=& \Vector{d} s - \Vector{d}_{pilot} s_{pilot} \\
  \Delta \Vector{r}_2  &=& X_2 \Vector{e}_{X2} + Y_2 \Vector{e}_{Y2}
\end{eqnarray}
We substitute
\begin{eqnarray}
 \Vector{d} &=& \Vector{d}_{pilot} + \Delta \Vector{d} \\
 s &=& s_{pilot} + \Delta s
\end{eqnarray}
leading to
\begin{eqnarray}
  \Delta \Vector{r}_2  &=& \Vector{d}_{pilot} \Delta s + \Delta \Vector{d} s_{pilot} + \Delta \Vector{d} \Delta s \\
  \Delta \Vector{r}_2  &=& X_2 \Vector{e}_{X2} + Y_2 \Vector{e}_{Y2}
\end{eqnarray}
$\Delta \Vector{d}$ originates from $\Delta \Wavevector$ and can be seen as given quantity.
The 3 unknowns $X_2 , Y_2 , \Delta s$ can be solved, for example, in a cartesian base
\begin{eqnarray}
 \begin{pmatrix}
  e_{X2x} & e_{Y2x} & - ( d_{pilotx} + \Delta d_x ) \\
  e_{X2y} & e_{Y2y} & - ( d_{piloty} + \Delta d_y ) \\
  e_{X2z} & e_{Y2z} & - ( d_{pilotz} + \Delta d_z ) 
 \end{pmatrix}^{-1}
 \Delta \Vector{d} s_{pilot} &=& 
 \begin{pmatrix}
  X_2 \\ Y_2 \\ \Delta s
 \end{pmatrix} 
\end{eqnarray}
We approximate up to first order in $\Delta \Vector{d}$
\begin{eqnarray}
 \begin{pmatrix}
  X_2 \\ Y_2 \\ \Delta s
 \end{pmatrix} 
 &\approx& s_{pilot}
 \begin{pmatrix}
  e_{X2x} & e_{Y2x} & - d_{pilotx} \\
  e_{X2y} & e_{Y2y} & - d_{piloty} \\
  e_{X2z} & e_{Y2z} & - d_{pilotz} 
 \end{pmatrix}^{-1}
 \begin{pmatrix}
  \Delta d_x \\ \Delta d_y \\ \Delta d_z
 \end{pmatrix}
\end{eqnarray}
Now we have to link the wavevector change $\Delta \Wavevector$ to the ray direction change $\Delta \Vector{d}$ in first order.
In isotropic media, we find
\begin{eqnarray}
 \Delta \Vector{d}_{iso} 
 &\approx& \frac{\Re \Delta \Wavevector}{|\Re \Wavevector_{pilot}|}
 \\
 &\approx&
  \frac{
    U(\Re \Vector{e}_U)
    + V(\Re \Vector{e}_V)
    + W(\Re \Vector{e}_W)
    + Q(\Re \Vector{e}_Q)
  }{|\Re \Wavevector_{pilot}|}
\end{eqnarray}

\remark{todo: general case, maybe shift whole direction perturbation in ray dir section \ref{sec:raydir}}

\subsubsection{The C-Matrix (contains errors)} 
Two rays that are parallel in plane 1 shall also be parallel in plane 2, 
independent of their point of origin.
\begin{eqnarray}
\hat{C} &=& 0 
\end{eqnarray}

\subsubsection{The D-Matrix (contains errors)}
The homogeneous medium shall not change the ray direction.
However, as the ray bases $( \Vector{e}_{U,V,W,Q\,1} )$ and $( \Vector{e}_{U,V,W,Q\,2} )$
may differ, its coordinates may also differ.
\begin{eqnarray}
 \Delta \Wavevector 
 &=& U_1 \Vector{e}_{U1} + V_1 \Vector{e}_{V1} + W_1 \Vector{e}_{W1} + Q_1 \Vector{e}_{Q1} \nonumber \\
 &=& U_2 \Vector{e}_{U2} + V_2 \Vector{e}_{V2} + W_2 \Vector{e}_{W2} + Q_2 \Vector{e}_{Q2}
\end{eqnarray}
We recommend to stick to a single coordinate system when propagating through homogeneous media, 
$( \Vector{e}_{U,V,W,Q\,1} ) = ( \Vector{e}_{U,V,W,Q\,2} )$. 
In this case, the $\hat{D}$ matrix simplifies to unity.
\begin{eqnarray}
 \hat{D} &=& \unittensor
\end{eqnarray}


\subsubsection{Example (contains errors)}
We assume two parallel planes and identical bases $\Vector{e}_{X,Y,U,V,W,Q}$ in both planes. 
The pilot ray direction shall be normal to the planes.
We collect the sub-matrices $\hat{A}, \hat{B}, \hat{C}, \hat{D}$ and yield

\begin{eqnarray}
 \hat{G} &=&
 \begin{pmatrix}
  1 & 0 & B_{11} & B_{12} & B_{13} & B_{14} \\
  0 & 1 & B_{21} & B_{22} & B_{23} & B_{24} \\
  0 & 0 & 1 & 0 & 0 & 0 \\
  0 & 0 & 0 & 1 & 0 & 0 \\
  0 & 0 & 0 & 0 & 1 & 0 \\
  0 & 0 & 0 & 0 & 0 & 1  
 \end{pmatrix}
 \label{eq:xyuv_propagation_homogeneous}
\end{eqnarray}
with\begin{eqnarray}
 \hat{B} &=&
 \frac{ s_{pilot} }{| \Re \Wavevector_{pilot} |}
 \begin{pmatrix}
  (\Re \Vector{e}_U) \Vector{e}_X & (\Re \Vector{e}_V) \Vector{e}_X & (\Re \Vector{e}_W) \Vector{e}_X & (\Re \Vector{e}_Q) \Vector{e}_X \\
  (\Re \Vector{e}_U) \Vector{e}_Y & (\Re \Vector{e}_V) \Vector{e}_Y & (\Re \Vector{e}_W) \Vector{e}_Y & (\Re \Vector{e}_Q) \Vector{e}_Y
 \end{pmatrix}
\end{eqnarray}
In lossless media, one often chooses real $\Vector{e}_{U,V}$ and imaginary $\Vector{e}_{W,Q}$.
Half of the matrix $\hat{B}$ becomes zeros.
Still, it is important to forward the wavevector components 
$W \Vector{e}_W$ and $Q \Vector{e}_Q$
to the next surface with the $\hat{D}$ matrix, 
as these wavevector components may become relevant in following materials.

\subsection{Refraction (contains errors)}
We assume two parallel planes 1,2 just in front of and behind a refracting surface.
The planes shall be tangential to the surface 
and shall have identical bases $\Vector{e}_{X,Y}$.
In general, it is not possible to find identical base vectors $\Vector{e}_{U,V,W,Q}$ in both planes.

\subsubsection{A and B-Matrix (contains errors)}
The infinitessimal propagation length through the surface does not change the ray height
\begin{eqnarray}
 \hat{A} &=& \unittensor \\
 \hat{B} &=& 0
\end{eqnarray}
Deviations from the tangential plane (sag differences due to curvature) 
are an effect quadratic in ray height and are neglected in this linear formalism.

\subsubsection{C and D-Matrix (contains errors)}
First, we consider the in-plane wavevector component $\Wavevector_\parallel$.
This component is conserved during refraction.
\begin{eqnarray}
 \Wavevector_{\parallel2} 
 &=& \Wavevector_{\parallel1} \\
 &=& \Wavevector_1 - (\Wavevector_1 \Vector{n}) \Vector{n} \\
 \wavenumber_{\parallel2i}
 &=& \wavenumber_{pilot1i} + \Delta \wavenumber_{1i} - ( \wavenumber_{pilot1j} + \Delta \wavenumber_{1j} )( n_{pilotj} + \Delta n_j )( n_{piloti} + \Delta n_i )
% \\
% &=& \wavenumber_{pilot1i} + \Delta \wavenumber_{1i} +\nonumber \\
% &&- \wavenumber_{pilot1j}   n_{pilotj} n_{piloti} \nonumber\\
% &&- \Delta \wavenumber_{1j} n_{pilotj} n_{piloti} \nonumber\\
% &&- \wavenumber_{pilot1j}   \Delta n_j n_{piloti} \nonumber\\
% &&- \Delta \wavenumber_{1j} \Delta n_j n_{piloti} \nonumber\\
% &&- \wavenumber_{pilot1j}   n_{pilotj} \Delta n_i \nonumber\\
% &&- \Delta \wavenumber_{1j} n_{pilotj} \Delta n_i \nonumber\\
% &&- \wavenumber_{pilot1j}   \Delta n_j \Delta n_i \nonumber\\
% &&- \Delta \wavenumber_{1j} \Delta n_j \Delta n_i
\end{eqnarray}
In our linear approximation, we neglect quadratic forms ($\Delta \Vector{n}^2$) or mixed terms ($\Delta \Wavevector \Delta \Vector{n}$)
%\begin{eqnarray}
% \wavenumber_{\parallel2i}
% &=& 
% \wavenumber_{pilot1i} - \wavenumber_{pilot1j}   n_{pilotj} n_{piloti} \nonumber\\
% &&+ \Delta \wavenumber_{1i} 
%   - \Delta \wavenumber_{1j} n_{pilotj} n_{piloti} \nonumber\\
% &&- \wavenumber_{pilot1j}   \Delta n_j n_{piloti}
%   - \wavenumber_{pilot1j}   n_{pilotj} \Delta n_i
%\end{eqnarray}
\begin{eqnarray}
 \wavenumber_{\parallel2i}
 &=& 
 \wavenumber_{\parallel pilot1i} \nonumber\\
 &&+ ( \delta_{ij} -  n_{piloti} n_{pilotj} ) \Delta \wavenumber_{1j} \nonumber\\
 &&- ( n_{piloti} \wavenumber_{pilot1j} + \wavenumber_{pilot1m}   n_{pilotm} \delta_{ij}) \Delta n_j
\end{eqnarray}
We apply this intermediate result on the full wavevector of the near-pilot ray after refraction
\begin{eqnarray}
 \Wavevector_2 &=& \Wavevector_{\parallel} + \xi_2(\Wavevector_{\parallel}) \Vector{n} \\
 \Wavevector_2 &=& \Wavevector_{\parallel}
                 + \left(
                     \xi_{2 pilot} +
                     \left. \frac{\partial \xi_2}{\partial \Wavevector_{\parallel}} \right|_{\Wavevector_{\parallel pilot}}
                     \cdot \Delta \Wavevector_{\parallel}
                   \right)
                   \left(
                     \Vector{n}_{pilot} + \Delta \Vector{n} 
                   \right)
 \\ 
% \wavenumber_{2i}
% &=& 
% \wavenumber_{2pilot i} 
% + \xi_{2 pilot} \Delta n_i
% + \left( 
%     \delta_{ij}
%     + \left. \frac{\partial \xi_2}{\partial \Wavevector_{\parallel}} \right|_j n_{pilot i}
%   \right) \Delta \wavenumber_{\parallel j}
% \\
% \Delta \wavenumber_{\parallel j}
% &=&  ( \delta_{jl} -  n_{pilot j} n_{pilot l} ) \Delta \wavenumber_{1l} \nonumber\\
% &&- ( n_{pilot j} \wavenumber_{pilot1 l} + \wavenumber_{pilot1 m}   n_{pilot m} \delta_{jl}) \Delta n_l
% \\
 \wavenumber_{2i}
 &=& 
 \wavenumber_{2pilot i} \nonumber \\
 &&+ \left( 
       \delta_{ij}
       + \left. \frac{\partial \xi_2}{\partial \Wavevector_{\parallel}} \right|_j n_{pilot i}
     \right) 
     ( \delta_{jl} -  n_{pilot j} n_{pilot l} ) \Delta \wavenumber_{1l} \nonumber\\
 &&+ \Bigg[
     \xi_{2 pilot} \delta_{il} \nonumber\\
 &&\quad- 
     \left( 
       \delta_{ij}
       + \left. \frac{\partial \xi_2}{\partial \Wavevector_{\parallel}} \right|_j n_{pilot i}
     \right) 
     ( n_{pilot j} \wavenumber_{pilot1 l} + \wavenumber_{pilot1 m}   n_{pilot m} \delta_{jl}) 
   \Bigg] \Delta n_l
\end{eqnarray}

\subsubsection{The D-Matrix (contains errors)}
We consider rays intersecting the surface at the pilot interesction position $\Vector{R} = 0$.
The surface normal for all these rays is the pilot surface normal $\Delta \Vector{n} = 0$.
We introduce the ($3\times3$) matrix $\hat{M}$ as
\begin{eqnarray}
 \Delta \wavenumber_{2i} &=& M_{il} \Delta \wavenumber_{1l} \\
 M_{il}
 &=& 
 \left( 
       \delta_{ij}
       + \left. \frac{\partial \xi_2}{\partial \Wavevector_{\parallel}} \right|_j n_{pilot i}
 \right) 
 ( \delta_{jl} -  n_{pilot j} n_{pilot l} )
\end{eqnarray}
Lengthy coordinate transformations can be avoided by choosing
\begin{eqnarray}
  \Vector{e}_{K_i\,2} &=& \frac{ \hat{M} \Vector{e}_{K_i\,1} }{\sqrt{ (\hat{M} \Vector{e}_{K_i\,1})\cdot \overline{(\hat{M} \Vector{e}_{K_i\,1})} }}
  \label{eq:eUeVVectors}
\end{eqnarray}
The unit vectors can be complex valued.
Our result for the $\hat{D}$ matrix is
\begin{eqnarray}
 D_{ij} &=&  \delta_{ij}
 \sqrt{ (\hat{M} \Vector{e}_{K_i\,1})\cdot \overline{(\hat{M} \Vector{e}_{K_i\,1})} } 
\end{eqnarray}
\remark{todo: double check before use}
For the unit vectors chosen above \eqref{eq:eUeVVectors}, the $D$-matrix is diagonal.

\subsubsection{The C-Matrix (contains errors)}
We consider rays parallel to the pilot ray, $\Vector{K} = 0$.
\begin{eqnarray}
 \Delta \wavenumber_{2i} &=& 
 \Bigg[
     \xi_{2 pilot} \delta_{il} \nonumber\\
 &&- 
     \left( 
       \delta_{ij}
       + \left. \frac{\partial \xi_2}{\partial \Wavevector_{\parallel}} \right|_j n_{pilot i}
     \right) 
     ( n_{pilot j} \wavenumber_{pilot1 l} + \wavenumber_{pilot1 m}   n_{pilot m} \delta_{jl}) 
   \Bigg] \Delta n_l
   \label{eq:refraction_C_matrix_deltaK_to_deltaN}
\end{eqnarray}
We assume a surface with local curvatures $\rho_{XX}, \rho_{XY}, \rho_{YY}$ in our coordinate system
$(\Vector{e}_X, \Vector{e}_Y, \Vector{n}_{pilot})$ according to equation \eqref{eq:quadratic_form_sag}.
The surface normal is (see \eqref{eq:quadratic_form_near_axis_normal})
\begin{eqnarray}
 \Vector{n} &\approx& 
 \Vector{n}_{pilot}
 - (\rho_{XX} X + \rho_{XY} Y) \Vector{e}_X
 - (\rho_{YY} Y + \rho_{XY} X) \Vector{e}_Y
 \label{eq:refraction_C_matrix_deltaN_to_deltaR}
\end{eqnarray}
Combining the last two equations \eqref{eq:refraction_C_matrix_deltaK_to_deltaN} and \eqref{eq:refraction_C_matrix_deltaN_to_deltaR},
we find a relation between the change in wavevector after refraction and the change in position,
\begin{eqnarray}
 \Delta \Wavevector_2 &=& 
 \hat{\tilde{M}}
 \begin{pmatrix}
  X \\ Y
 \end{pmatrix}
\end{eqnarray}
mediated by a ($3\times2$) matrix $\hat{\tilde{M}}$.
We add a variable $\nu=0$ to the left hand side and solve the system in a cartesian base
\begin{eqnarray}
   U_2 \Vector{e}_{U2}
 + V_2 \Vector{e}_{V2}
 + \nu \Vector{n}_{pilot}
 &=& 
 \hat{\tilde{M}}
 \begin{pmatrix}
  X \\ Y
 \end{pmatrix}
 \\
 \begin{pmatrix}
  U_2 \\ V_2 \\ \nu
 \end{pmatrix}
 &=&
 \begin{pmatrix}
  e_{U2x} & e_{U2y} & e_{U2z} \\
  e_{V2x} & e_{V2y} & e_{V2z} \\
  n_{pilot x} & n_{pilot y} & n_{pilot z}
 \end{pmatrix}^{-1}
 \hat{\tilde{M}}
 \begin{pmatrix}
  X \\ Y
 \end{pmatrix}
\end{eqnarray}
Adding $\nu$ allows us to use standard matrix inversion algorithms.
The result should reproduce $\nu=0$ within numerical accuracy.
We consider the positions $(dX,0)$ and $(0,dY)$, and obtain the corresponding wavevector changes $(dU_2,dV_2, dW_2, dQ2)$ for both cases.
We use this result to fill the $(2\times4)$ matrix $\hat{C}$.

\remark{a subsubsection about coordinate breaks is missing.}

\subsection{Image Plane Position in a Finite Conjugate Image Space (contains errors)} \label{sec:xyuv_image_plane_position}
A ray fan is being imaged, 
if for a common ray starting position $\Vector{R}_\obj$, all multiples of a
ray directions $\Vector{K}_\obj \neq 0$ are propagated on a common point in image space.
The imaging equation \eqref{eq:xyuv_imaging_equation} becomes
\begin{eqnarray}
 \Vector{R}_\im &=& \hat{A} \cdot \Vector{R}_\obj + 0
\end{eqnarray}
or, in other words,
\begin{eqnarray}
 \hat{B} \cdot \Vector{K}_\obj &=& 0
\end{eqnarray}
The matrix $\hat{B}$ is of rank 2, the input vector $\Vector{K}$ contains 4 independent real valued degrees of freedom.
For a unique solution, we need to impose more constraints on the input vector. 
We consider the plane waves of the pilot ray and a near pilot ray, 
$\Wavevector_{pilot}$ and 
$\Wavevector = \Wavevector_{pilot} + U \Vector{e}_U + V \Vector{e}_V + W \Vector{e}_W + Q \Vector{e}_Q$.
Both rays propagate from the object to the image.
We assume a coherent source.
Both rays interfere and form a sinusoidal grating intensity pattern
\begin{eqnarray}
 I &\propto& |\Efield_{0,pilot}|^2 + |\Efield_{0}|^2 + 2 \Re \left( \overline{\Efield}_{0}\Efield_{0,pilot} \exp(i(\Wavevector_{pilot} - \Wavevector)\Location) \right)
\end{eqnarray}
In the object plane, we desire the modulation depth of the intensity pattern to be independent of position.
This condition is fulfilled if $i$ times the exponent is real.
\begin{eqnarray}
 (\Wavevector_{pilot} - \Wavevector)\Location \in \mathbb{R} \label{eq:xyuvRealGratingConstraint}
\end{eqnarray}
This shall hold for all positions $\Vector{R}=(X,Y)$, so
\begin{subequations}
\begin{eqnarray}
 \Im \left( ( U \Vector{e}_U + V \Vector{e}_V + W \Vector{e}_W + Q \Vector{e}_Q ) \Vector{e}_X \right) &=& 0 \\
 \Im \left( ( U \Vector{e}_U + V \Vector{e}_V + W \Vector{e}_W + Q \Vector{e}_Q ) \Vector{e}_Y \right) &=& 0 
\end{eqnarray}
\label{eq:restriction_to_2d_kspace} 
\end{subequations}
With 4 variables and 2 equations, we can find 2 independent solutions $\Vector{e}_{\kappa1}$,$\Vector{e}_{\kappa2}$ that fulfill the two conditions.
$\Vector{e}_{\kappa i}$ are four-dimensional, have real valued components and can be transfered to a cartesian wavevector deviation via
\begin{eqnarray}
 \Delta \Wavevector(\Vector{e}_{\kappa i}) &=& e_{\kappa i\,1} e_{U} + e_{\kappa i\,2} e_{V} + e_{\kappa i\,3} e_{W} + e_{\kappa i\,4} e_{Q}
\end{eqnarray}
The base vectors $\Vector{e}_{\kappa i}$ shall be normalized and orthogonal, 
$\Delta \Wavevector(\Vector{e}_{\kappa i}) \cdot \overline{\Delta \Wavevector (\Vector{e}_{\kappa j})} = \delta_{ij}$.
Choosing the wavevector deviation in the object plane $\Delta \Wavevector_\obj$ as a multiple of these two directions,
we ensure a real valued grating in the object plane. 
The intensity distribution in intermediate planes and in the image plane, however, 
may contain also imaginary components in $\Delta \Wavevector \cdot \Vector{e}_{X,Y}$.
\begin{eqnarray}
 \Delta \Wavevector_\obj &=& \kappa_1 \Vector{e}_{\kappa1} + \kappa_2 \Vector{e}_{\kappa2} \label{eq:deltaK_as_kappa}
\end{eqnarray}
$\kappa_{1,2} \in \mathbb{R}$ are coordinates in this 2-dimensional representation of wavevector deviations.
Our imaging condition reduces to
\begin{eqnarray}
 \hat{B}
 \begin{pmatrix}
 e_{\kappa11} & e_{\kappa21} \\
 e_{\kappa12} & e_{\kappa22} \\
 e_{\kappa13} & e_{\kappa23} \\
 e_{\kappa14} & e_{\kappa24} \\ 
 \end{pmatrix}
 \begin{pmatrix}
 \kappa_1 \\ \kappa_2 
 \end{pmatrix}
 &=& 0
\end{eqnarray}
Nontrivial solutions are found if the determinant of $\hat{B} \cdot(\Vector{e}_{\kappa1}, \Vector{e}_{\kappa2})$ vanishes.
In case the determeninant is nonzero, 
we refocus by inserting a homogeneous material propagation \eqref{eq:xyuv_propagation_homogeneous} 
\remark{The referenced equation is only valid for homogeneous media, parallel planes and a pilot normal to the planes. Todo: Check if the following equations also work in the general case.} 
by a distance $s$ 
in front of the image plane.
We obtain the matrices describing the imaging system after refocus and projection on our 2 wavevector solutions
\begin{eqnarray}
 \tilde{A} &=& TODO \\
 \tilde{B}_{i,n} &=& \left( B_{ij} + \sum_m \frac{\partial^2 R_i}{\partial s \partial K_m} D_{mj} \cdot s \right) e_{\kappa j,n} \label{eq:tildeB}
\end{eqnarray}
The matrix $\hat{\tilde{B}}$ is a 2 by 2 matrix.

The imaging condition becomes
\begin{eqnarray}
 \hat{\tilde{B}}(s) \begin{pmatrix} \kappa_1 \\ \kappa_2 \end{pmatrix} &=& 0
\end{eqnarray}
We are interested in solutions $s$ for which the determinant of $\hat{\tilde{B}}$ vanishes.
The characteristic polynomial is quadratic in $s$.
This problem is similar to an eigenvalue problem, so we call the solutions $s_{1,2}$ quasi-eigenvalues.

\subsubsection{The Stigmatic Case (contains errors)}
If both solutions $s_1 = s_2$ are degenerate, the corresponding eigen-space is two-dimensional.
Any combination $(\kappa_1, \kappa_2)$ will be imaged on the pilot image point.
\begin{eqnarray}
 \Vector{R}_\im &=& \hat{\tilde{A}}(s) \cdot \Vector{R}_\obj
\end{eqnarray}
For example, centro-symmetric systems have no (on-axis) astigmatism if the pilot ray is chosen on the optical axis.

\subsubsection{The Astigmatic Case (contains errors)}
If both solutions $s_{1,2}$ are not degenerate, each solution has a one-dimensional eigen-space.
Focusing to any of the two solutions $s_i$, 
we can find an eigen ray fan $(\kappa_{1\,eig\,si},\kappa_{2\,eig\,si})$ that is imaged on the pilot ray.
We normalize the eigen-vectors
\begin{eqnarray}
 \begin{pmatrix} \kappa_{1\,eig\,si} \\   \kappa_{2\,eig\,si} \end{pmatrix} \cdot  \begin{pmatrix} \overline{\kappa}_{1\,eig\,si} \\ \overline{\kappa}_{2\,eig\,si} \end{pmatrix}
 &=& 1
\end{eqnarray}
Any multiple of the direction deviation eigen-vectors are imaged on the pilot.
Deviation components orthogonal to the eigen directions, however, will not be imaged on the pilot ray.
We represent the 2-dimensional wavevector deviation in the base of the eigen direction and the orthogonal direction
\begin{eqnarray}
     \begin{pmatrix} \kappa_1            \\   \kappa_2            \end{pmatrix} &=& 
   a \begin{pmatrix} \kappa_{1\,eig\,si} \\   \kappa_{2\,eig\,si} \end{pmatrix} 
 + b \begin{pmatrix} \kappa_{2\,eig\,si} \\ - \kappa_{1\,eig\,si} \end{pmatrix}
\end{eqnarray}

The imaging equation \eqref{eq:xyuv_imaging_equation} becomes

\begin{eqnarray}
 \Vector{R}_\im &=& \hat{\tilde{A}}(s_i) \cdot \Vector{R}_\obj + \hat{\tilde{B}}(s_i) \cdot b(s_i,\Delta\Wavevector) \cdot \begin{pmatrix} \kappa_{2\,eig\,si} \\ - \kappa_{1\,eig\,si} \end{pmatrix}
\end{eqnarray}
The coefficient $b$ depends on the object sided wavevector deviation.
We introduce the projection operator $\hat{P}$ with
\begin{eqnarray}
 \Vector{R}_\im &=& \hat{\tilde{A}}(s_i) \cdot \Vector{R}_\obj + \hat{P} \cdot \begin{pmatrix} \kappa_{1} \\ \kappa_{2} \end{pmatrix} \\
 \hat{P} \cdot \begin{pmatrix} \kappa_{1} \\ \kappa_{2} \end{pmatrix} &=& 
    \left( \hat{\tilde{B}}(s_i) \cdot \begin{pmatrix} \kappa_{2\,eig\,si} \\ - \kappa_{1\,eig\,si} \end{pmatrix} \right) 
    \cdot
    \left( \begin{pmatrix} \kappa_{2\,eig\,si} \\ - \kappa_{1\,eig\,si} \end{pmatrix} \cdot \begin{pmatrix} \kappa_{1} \\ \kappa_{2} \end{pmatrix} \right)
\end{eqnarray}
The image position $\Vector{R}_\im$ of a near-pilot ray depends on both object position $\Vector{R}_\obj$ and wavevector choice $(\kappa_1,\kappa_2)$.
In general, a point in the object plane is imaged to a line in the image plane.
To characterize this line, we can use, for example,
\begin{itemize}
 \item the wavevector deviation at the border of the aperture (marginal ray) that produces the largest image distance from the pilot.
 \item the RMS image position deviation from the pilot when rastering the pupil.
       For accurate results, the angular power distribution of the object should be considered as weighting factor.
 \item the RMS image position deviation from the center of mass.
\end{itemize}
Based on the criterion for the line length chosen, we define a line vector in image space $\Vector{h}_{line}$.
The line vector is a measure for the amount of astigmatism in the corresponding image plane -- by its line length and orientation.


\subsubsection{Circle of Least Confusion (contains errors)}
For some systems, like anamorphotic laser line projectors, it makes sense to pick one of the two solutions $s_{1,2}$ as the image plane.
Other systems, like imaging systems with weak on-axis astigmatism, 
are often charcaterized by an intermediate plane between $s_{1,2}$ as compromise image plane for both wavevector directions.
The spot size in this intermediate plane is called \emph{circle of least confusion}.

We consider 2 rays: 
The first ray coincides with the pilot ray in plane $s_1$, but has an infinitessimal wavevector deviation from the pilot.
The wavevector deviation is chosen orthogonal to the eigen-wavevector direction in plane $s_2$.
In other words, this ray is chosen to have a maximum astigmatic image height per wavevector deviation in plane $s_2$.
The second ray shall coincide with the pilot ray in plane $s_2$, but have a maximum image height in plane $s_1$.

We restrict ourselves to homogeneous media and calculate the ray height $\Vector{h}_2$ of the first ray in plane $s_2$ 
and the ray height $\Vector{h}_1$ of the second ray in plane $s_1$.

\begin{align*}
\textrm{Ray 1:} && \Vector{R}(s_1) &= \Vector{0} && \Vector{R}(s_2) = \Vector{h}_2 && \begin{pmatrix} \kappa_1\\\kappa_2 \end{pmatrix} = \begin{pmatrix} \kappa_{2\,eig\,s2} \\ - \kappa_{1\,eig\,s2} \end{pmatrix}
\\
\textrm{Ray 2:} && \Vector{R}(s_1) &= \Vector{h}_1 && \Vector{R}(s_2) = \Vector{0} && \begin{pmatrix} \kappa_1\\\kappa_2 \end{pmatrix} = \begin{pmatrix} \kappa_{2\,eig\,s1} \\ - \kappa_{1\,eig\,s1} \end{pmatrix}
\end{align*}
\remark{This holds for apertures circular in object K-space only. todo: wavevectors should be scaled with pupil size in their direction. But what about funny shaped apertures ?}

In the plane $s_{coc}$ of the circle of least confusion, both rays shall have the same ray height.
One ray increases its height, the other ray decreases its height linearly with distance, so the solution is found as

\begin{eqnarray}
 s_{coc} &=& \alpha s_1 + (1-\alpha) s_2 \\
 \alpha  &=& \frac{|\Vector{h}_2|}{|\Vector{h}_1| + |\Vector{h_2}|}
\end{eqnarray}

For weakly astigmatic systems, $\lim \Vector{h}_{1,2} \rightarrow \Vector{0}$, 
the calculation of $\alpha$ becomes numerically instable.
For systems with finite aperture extensions in both directions, 
weak astigmatism also means that both planes converge, $s_1 \rightarrow s_2$,
so the accuracy required decreases.
To prevent instabilities, one may consider to clip $\alpha$ to the interval $[0,1]$ after calculation.

\remark{How to determine the image plane if there's image plane tilt?}

\subsection{Parabasal Chief Rays (contains errors)}
We consider a surface $surf1$ in front of the optical system.
For example, it may be chosen identical to the object.
For infinite conjugate objects, it may be chosen as the first surface with finite distance from the optical system.
We assume the XYUV matrix from $surf1$ to the aperture is known.
The propagation reads
\begin{eqnarray}
 \Vector{R}_{aper}
  &=&
  \hat{A}_{surf1\rightarrow aper}  
  \Vector{R}_{surf1}
  +
  \hat{B}_{surf1\rightarrow aper}
  \Vector{K}_{surf1}
\end{eqnarray}
The chief ray has zero ray height in the aperture compared to the pilot,
\begin{eqnarray}
 \Vector{R}_{chief, aper} &=& \Vector{0} 
\end{eqnarray}
For the chief ray, position and wavevector on $surf1$ are no longer independent.
In case the (parabasal) chief ray wavevector is given, we obtain the chief ray height on $surf1$ as
\begin{eqnarray}
 \Vector{R}_{chief, surf1}
 &=&
 - \hat{A}^{-1}_{surf1\rightarrow aper} \hat{B}_{surf1\rightarrow aper} \Vector{K}_{surf1}
\end{eqnarray}
In case the chief ray height is given, there are several wavevectors that fulfill the equation.
If we restrict ourselves to wavevectors that create a constant object modulation over position \eqref{eq:deltaK_as_kappa},
we find
\begin{eqnarray}
 \begin{pmatrix} \kappa_{1 chief,surf1} \\ \kappa_{2 chief,surf1} \end{pmatrix}
  &=&
  - \hat{\tilde{B}}^{-1}_{surf1\rightarrow aper} \hat{A}_{surf1\rightarrow aper} \Vector{R}_{surf1} \label{eq:kappachief_as_function_of_R}
\end{eqnarray}

\subsection{Parabasal Magnification (contains errors)}
We assume an optical system in which the imaging condition between object and image plane is fulfilled.
We define magnification as the relation between image sided and object sided chief ray field coordinate.
There are basically two different ways how a field can be defined: by ray height or by wavevector.
For infinite conjugates, the field must be defined by the wavevector, or, equivalently, by the angle of the collimated bundle.
For finite conjugates, both definitions are possible.
An exception is found in Telecentric systems, where the field cannot be defined by chief ray angle.
We address the four cases of object and image sided field definition by ray height or wavevector, individually.

\subsubsection{Object Ray Height, Image Ray height (contains errors)}
We use the dependence of chief ray wavevector on the field coordinate \eqref{eq:kappachief_as_function_of_R} and obtain
\begin{eqnarray}
 \Vector{R}_{chief, im} &=& \hat{M} \Vector{R}_{chief, obj} \\
 \hat{M} &=& \hat{A}_{obj\rightarrow im} - \hat{B}_{obj\rightarrow im} \hat{\tilde{B}}^{-1}_{obj\rightarrow aper} \hat{A}_{obj\rightarrow aper}
\end{eqnarray}
The magnification $\hat{M}$ is unitless. 
For this result, we restricted the wavevector space to two dimensions, as introduced in \eqref{eq:restriction_to_2d_kspace}.

\subsubsection{Object Ray Height, Image Wavevector}
TODO
\subsubsection{Object Wavevector, Image Ray Height}
TODO
\subsubsection{Object Wavevector, Image Wavevector}
TODO

todo: continue with a common section for all four cases, interpreting the magnification matrix and its eigen-rays.

\subsection{Entrance Pupil (contains errors)} \label{sec:xyuv_entrance_pupil}
We consider a sequential raytrace with the sequency
\begin{eqnarray*}
 \textrm{object} \rightarrow 
 \textrm{entrance pupil} \rightarrow 
 \textrm{object} \rightarrow
 \\ \rightarrow 
 \textrm{optical system front group} \rightarrow
 \textrm{stop} \rightarrow
 \textrm{optical system rear group} \rightarrow
 \textrm{image}
\end{eqnarray*}
We assume the propagation from object to entrance pupil takes place in the same material surrounding the object.
If the object is embedded in a GRIN material, 
we assume a homogeneous continuation of the same refractive index as at the intersection of object and pilot ray.
For the back-propagation from the entrance pupil to the object, we use the inverse propagator.
The inverse propagator is not the same as the forward-propagator with swapped start and end planes.
For example, the inverse propagator restores evanescent waves.

We consider the propagation between the conjugate planes
\begin{eqnarray*}
 \textrm{entrance pupil} \rightarrow 
 \textrm{object} \rightarrow
 \textrm{optical system front group} \rightarrow
 \textrm{stop} 
\end{eqnarray*}
First, we calculate the XYUV matrix from object to stop. 
Then we add the propagation from object to pupil as a function of pupil position.
We then choose the pupil position such that pupil and stop are imaging conjugates, as shown in \ref{sec:xyuv_image_plane_position}.

\remark{What is THE pupil position if the optical system front group suffers from astigmatism ?}

\subsection{Exit Pupil (contains errors)}
In close analogy to the entrance pupil \ref{sec:xyuv_entrance_pupil}, 
we insert an exit pupil plane between the rear group of optical elements and the image.
We consider the propagation between the conjugate planes
\begin{eqnarray*}
 \textrm{stop} \rightarrow
 \textrm{optical system rear group} \rightarrow
 \textrm{image} \rightarrow
 \textrm{exit pupil}
\end{eqnarray*}
We then choose the pupil position such that pupil and stop are imaging conjugates, as shown in \ref{sec:xyuv_image_plane_position}.

\remark{What is THE pupil position if the optical system rear group suffers from astigmatism ?}


\section{Ray Aiming}
When shooting rays at an optical system, we need initial properties like ray starting position and wavevector.
But which properties do I have to choose for the marginal, chief, coma or any other ray ?
Ray aiming allows to approximate initial ray properties for a given optical system.
Our ansatz of ray aiming consists of the following steps:
\begin{itemize}
 \item Calculate the stop size in paraxial or parabasal approximation.
 \item Calculate ray starting points and wavevectors for given field and pupil coordinates in paraxial or parabasal approximation
 \item Convert the paraxially approximated wavevectors to real wavevectors,
       that obey the dispersion relation in the object material
 \item Optional: Perform real ray tracing. Check whether the rays hit the desired coordinates 
       and whether the stop size fulfills the designer's aperture definition.
       Optimize the aiming iteratively.
 \item Adapt the ray intensities so that the ray bundle resembles the angular power distribution $\frac{\partial P}{\partial \Omega}$ of the light source.
\end{itemize}

\subsection{Calculating the Stop Size}
The aperture of an optical system is often defined indirectly.
Some common definitions are via stop size, pupil diameter, aperture number (f-stop), or numerical aperture.
However, not all of these quantities can be applied to all tpyes of system.
A definition of pupil size and position would become numerically instable for telecentric systems.
Numerical aperture or aperture number definitions can only be defined for finite conjugate systems.

Instead, we use the stop size as a universal measure for the aperture definition.
The stop, as physical part of the optical system, is always finite sized and at a finite distance.
In the following, we show how to convert different aperture definitions to a stop size.
The stop does not need to be circular. 

In the implementation, it is better not to include the calculated aperture object in the optical system.
Otherwise, small numerical errors might cause marginal or coma rays to lie just outside the aperture.
These rays could be clipped, confusing automated optimization routines.
Rather we use the aperture object only to calculate the ray starting properties.
We leave the stop plane in the optical system without an aperture.


\subsubsection{Entrance Pupil Size}
TODO

\subsubsection{Exit Pupil Size}
TODO

\subsubsection{Infinite Conjugate Object Space Aperture Number}
TODO, uncommon definition

\subsubsection{Infinite Conjugate Image Space Aperture Number}
TODO

\subsubsection{Working Object Space Aperture Number}
TODO, uncommon definition

\subsubsection{Working Image Space Aperture Number}
TODO

\subsubsection{Object Space Numerical Aperture}
TODO

\subsubsection{Image Space Numerical Aperture}
TODO

\remark{todo: consider anamorphotic systems with fast and slow NA, f-number; consider non-circular stops and pupils (elliptic and funny shaped)}


\subsection{Ray Origin for Finite Conjugate Objects}
In object sided finite conjugate systems, the ray origin defines the field, and the ray wavevector defines the pupil position.

\subsubsection{Specification of Object Ray Height}
In case the field is specified by the object sided ray height,
this definition directly fixes the ray origin in the object plane.
A parabasal calculation is error-free.

\subsubsection{Specification of Object Sided Chief Ray Angle}
This definition is uncommon for finite conjugate systems. 
It makes sense for large object distances and small apertures.
It is unsuitable for object sided telecentric systems.

The ray origin is obtained approximately like this \remark{TODO: explain properly}
\begin{itemize}
 \item Calculation of entrance pupil position 
       \remark{TODO: what is the pupil position if there is on-axis astigmatism between stop and pupil ?}
 \item Calculation of Pilot direction (normalized pilot Poynting vector)
 \item Chief direction = Pilot direction, rotated by the specified field angle
 \item The chief ray is a ray with the chief direction, that intersects the pilot in the pupil
 \item Ray origin = intersection of object plane and chief ray
\end{itemize}

Using real rotation matrices (Rodrigues matrices) makes this algorithm more robust for finite fields.
The results, however, may still need iterative adaption for increased accuracy, as the real pupil position may move with field.

\subsubsection{Specification of Image Sided Ray Height}
The object sided ray height is obtained from the image sided ray height by the inverse of the XYUV matrix between object and image.
\remark{TODO: what is THE object position belonging to an image position if there is on-axis astigmatism ? Chief ?}



\subsection{Ray Direction for Finite Conjugate Objects}
The ray originates from a point $\Vector{R}_\obj$, defined by the field definition (see above).
Additionally to this constraint, it shall hit the stop plane at a certain ray height $\Vector{R}_{stop}$.
In the following, we search for a ray direction to fulfill this constraint.
We reduce the 4 dimensional wavevector deviation $\Vector{K}$ to the 2 dimensional $\kappa$ base according to \eqref{eq:deltaK_as_kappa}
and find in parabasal approximation
\begin{eqnarray}
% \Vector{R}_{stop} &=& \hat{\tilde{A}}_{obj\rightarrow stop} \cdot \Vector{R}_\obj
%                  +    \hat{\tilde{B}}_{obj\rightarrow stop} \cdot \begin{pmatrix}\kappa_1 \\ \kappa_2\end{pmatrix}
% \\
% \Vector{R}_{stop} - \hat{\tilde{A}}_{obj\rightarrow stop} \cdot \Vector{R}_\obj &=& 
%                  \hat{\tilde{B}}_{obj\rightarrow stop} \cdot \begin{pmatrix}\kappa_1 \\ \kappa_2\end{pmatrix}
% \\
 \begin{pmatrix}\kappa_1 \\ \kappa_2\end{pmatrix}
 &=&
 \hat{\tilde{B}}_{obj\rightarrow stop}^{-1}
 \left(
    \Vector{R}_{stop} - \hat{\tilde{A}}_{obj\rightarrow stop} \cdot \Vector{R}_\obj 
 \right) 
\end{eqnarray}


\subsection{Ray Direction for Infinite Conjugate Objects}
In object sided infinite conjugate systems, the desired field defines the ray direction, 
and the ray height on the first surface of the finite sized optical system defines the pupil position.

We restrict the starting wavevector deviation from the pilot to the 2 dimensional base of $\kappa$ according to \eqref{eq:deltaK_as_kappa}.
We calculate the change in Poynting vector when the wavevector is changed infinitessimally in direction of $\kappa_1$ or $\kappa_2$
and conclude the change of ray direction $\Vector{d}$
\begin{eqnarray}
 \Vector{d} \approx \Vector{d}_{pilot} + \frac{\partial \Vector{d}}{\partial \kappa_i} \kappa_i
\end{eqnarray}
\remark{TODO: explicit formula}


\subsection{Ray Origin for Infinite Conjugate Objects}
The ray wavevector is defined by the field definition (see above).
Additionally to this constraint, it shall hit the stop plane at a certain ray height $\Vector{R}_{stop}$.
To achieve this, we set ray height $\Vector{R}_{surf1}$ on the first surface of our finite sized optical system. 
We find in parabasal approximation
\begin{eqnarray}
 \Vector{R}_{surf1}                 
 &=&
 \hat{\tilde{A}}_{surf1\rightarrow stop}^{-1}
 \left(
   \Vector{R}_{stop} - \hat{\tilde{B}}_{surf1\rightarrow stop} \cdot \begin{pmatrix}\kappa_1 \\ \kappa_2\end{pmatrix}
 \right)
\end{eqnarray}

\subsection{Conversion to Wavevectors Obeying the Dispersion}
todo

\subsection{Angular Power Distribution}
todo





\end{document}
