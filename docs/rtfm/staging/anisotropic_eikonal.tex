\documentclass[12pt,a4paper,twoside,openright,BCOR10mm,headsepline,titlepage,abstracton,chapterprefix,final]{scrreprt}

\usepackage{ae}
\usepackage[ngerman, english]{babel}
%\usepackage{SIunits}

\usepackage{amsmath}
\usepackage{amssymb}
\usepackage{amsfonts}
\usepackage{xcolor}
\usepackage{setspace}

% load hyperref as the last package to avoid redefinitions of e.g. footnotes after hyperref invokation

\RequirePackage{ifpdf}  % flag for pdf or dvi backend
\ifpdf
  \usepackage[pdftex]{graphicx}
  \usepackage[pdftitle={RTFM on Imaging Theory and Basics of Optical Raytracing},%
              pdfsubject={},%
              pdfauthor={M. Esslinger, J. Hartung, U. Lippmann},%
              pdfkeywords={},%
              bookmarks=true,%
%              colorlinks=true,%
              urlcolor=blue,%
              pdfpagelayout=TwoColumnRight,%
              pdfpagemode=UseNone,%
              pdfstartview=Fit,%
	      pdfpagelabels,
              pdftex]{hyperref}
\else
  \usepackage[dvips]{graphicx}
  \usepackage[colorlinks=false,dvips]{hyperref}
\fi
%\DeclareGraphicsRule{.jpg}{eps}{.jpg}{`convert #1 eps:-}

\usepackage{ae}
%\usepackage[ngerman, english]{babel}

%\usepackage{SIunits}
\newcommand\elementarycharge{\textrm{e}}
\newcommand\sccm{\textrm{sccm}}
\newcommand\mbar{\milli\textrm{bar}}


\usepackage{amsmath}
%\usepackage{amssymb}
\usepackage{setspace}

%\widowpenalty = 1000


\newcommand*{\doi}[1]{\href{https://doi.org/\detokenize{#1}}{doi: \detokenize{#1}}}

\newcommand\Vector[1]{{\mathbf{#1}}}

\newcommand\vacuum{0}

\newcommand\location{r}
\newcommand\Location{\Vector{r}}


\newcommand\wavenumber{k}
\newcommand\vacuumWavenumber{\wavenumber_{\vacuum}}
\newcommand\Wavevector{\Vector{\wavenumber}}

\newcommand\Nabla{\Vector{\nabla}}
\newcommand\Laplace{\Delta}
\newcommand\timederivative[1]{\dot{{#1}}}
\newcommand\Tensor[1]{\hat{#1}}
\newcommand\conjugate[1]{\overline{#1}}
\newcommand\transpose[1]{#1^{T}}
\newcommand\Norm[1]{\left| #1 \right|}
\newcommand{\ket}[1]{\left\vert{#1}\right\rangle}
\newcommand{\bra}[1]{\left\langle{#1}\right\vert}
\newcommand{\braket}[2]{\left\langle{#1}\vert{#2}\right\rangle}
\newcommand{\bracket}[1]{\left\langle{#1}\right\rangle}

\newcommand{\scpm}[2]{(#1\,\cdot\,#2)}

\newcommand\Greenfunction{\Tensor{G}}

\newcommand\scalarEfield{E}
\newcommand\scalarBfield{B}
\newcommand\scalarHfield{H}
\newcommand\scalarDfield{D}
\newcommand\scalarTipfield{T}
\newcommand\scalarSamplefield{S}
\newcommand\scalarDipolarmoment{p}
\newcommand\Efield{\Vector{\scalarEfield}}
\newcommand\Bfield{\Vector{\scalarBfield}}
\newcommand\Hfield{\Vector{\scalarHfield}}
\newcommand\Dfield{\Vector{\scalarDfield}}
\newcommand\Dipolarmoment{\Vector{\scalarDipolarmoment}}

\newcommand\permeability{\Tensor{\mu}}
\newcommand\vacuumpermeability{\mu_{\vacuum}}
\newcommand\permittivity{\Tensor{\epsilon}}
\newcommand\generalPermittivity{\Tensor{\tilde\epsilon}}
\newcommand\vacuumpermittivity{\epsilon_{\vacuum}}
\newcommand\scalarpermittivity{\epsilon}
\newcommand\conductivity{\Tensor{\sigma}}
\newcommand\susceptibility{\Tensor{\chi}}
\newcommand\currentdensity{\Vector{j}}
\newcommand\Current{\Vector{I}}
\newcommand\chargedensity{\rho}
\newcommand\PoyntingVector{\Vector{S}}

\newcommand\ordi{\text{ord}}
\newcommand\eo{\text{eo}}

\newcommand{\timeavg}[1]{{\langle\,#1\,\rangle}}

\newcommand{\remark}[1]{{\color{red}$\blacksquare$}\footnote{{\color{red}#1}}}
\newcommand{\chk}[1]{\color{green}{$\checkmark$#1}}


\newif\ifdraft
\draftfalse % \drafttrue




\begin{document}


\section{Derivation of Eikonal Equation in inhomogenous, anisotropic Media}

\subsection{Reducing the Maxwell field equations to Eikonal-type equations}
Starting at the source-free Maxwell equations \eqref{eq:sourcefreemaxwell} 
in general complex, anisotropic and inhomogenous
permittivity tensor field $\hat{\epsilon}_{ij}(\Vector{r})$.
We start at the generalisation of the Helmholtz wave equation for anisotropic media
\begin{align}
 (\delta_{ij} \Delta - \partial_i \partial_j + \omega^2 \mu_0 \permittivity_{ij}) E_j = 0\,,
\end{align}
which's derivation only relies on\remark{is that really the case?}
\begin{itemize}
 \item no memory effects in $\permittivity_{ij}$ (no time convolution in the material equations)
 \item monochromaticity (no mixing of different $\omega$ due to material properties)\,.
\end{itemize}
By contracting this equation by $\partial_i$ one gets back the divergence equation which is also zero.
To derive the eikonal equations we make the ansatz
\begin{align}
 E_k &= A_k(\Vector{r}) \exp[i \psi(\Vector{r})]\,,
\end{align}
for the $\Vector{E}$ field. This is since we cannot assume that in such a complicated wave equations 
the solutions arise as plane waves with constant amplitude. Insertion of the ansatz gives [omitting the 
exponential factor and separating by real and imaginary parts]
\begin{align}
    & \left[
     \omega^2 \mu_0\, \permittivity_{k\ell}A_\ell
    -A_k \partial_i \psi \partial_i \psi
   +\Delta A_k 
   +A_\ell \partial_k \psi \partial_\ell \psi
   -\partial_k \partial_\ell A_\ell
   \right] \nonumber\\&
   +i \left[
    A_k \Delta \psi
    +2\partial_i \psi \partial_i A_k
    -A_\ell \partial_k \partial_\ell \psi
    -\partial_\ell A_\ell \partial_k \psi
    -\partial_\ell \psi \partial_k A_\ell
   \right] = 0\,.
\end{align}
Neglecting the second derivatives terms gives [proportional to $k^2$]
\begin{align}
    & \left[
    \omega^2 \mu_0\, \permittivity_{k\ell}A_\ell 
    -A_k \partial_i \psi \partial_i \psi
   +A_\ell \partial_k \psi \partial_\ell \psi
   \right] 
   +i \left[
    2\partial_i \psi \partial_i A_k
    -\partial_\ell A_\ell \partial_k \psi
    -\partial_\ell \psi \partial_k A_\ell
   \right] = 0\,.\label{eq:protoeikonal}
\end{align}
In the usual derivation of the Eikonal equation
from the Helmholtz wave equation there are only
two summands appearing in every part of the complex number.
This comes from the divergence part which was not eliminated, yet.
Taking $\partial_i D_i = \partial_i (\permittivity_{ij} E_j) = 0$ can be
reduced to (by using the ansatz above)
\begin{align}
 \partial_i (\permittivity_{ij} E_j) = \partial_i (\permittivity_{ij} A_j) \exp(i \psi) + i \permittivity_{ij} A_j \partial_i \psi \exp(i \psi) = 0\,.
\end{align}
After omitting the exponential factor we see that from the divergence condition follows that
\begin{align}
 \partial_i (\permittivity_{ij} A_j) &= 0\,,\label{eq:divaeps}\\
 \permittivity_{ij} A_j \partial_i \psi &= 0\,.\label{eq:divapsi}
\end{align}
Contracting the real part of \eqref{eq:protoeikonal} by $\partial_k \psi$ leads to the divergence condition \eqref{eq:divapsi}.
From the real part we could extract an eigen value equation via
\begin{align}
 \left[\omega^2 \mu_0 \permittivity_{k\ell} - \delta_{k\ell} (\partial_i \psi)^2 + (\partial_k \psi)(\partial_\ell \psi)\right] A_\ell  &= 0 \label{eq:anisoeikonal}\,.
\end{align}

% \begin{align}
%      (\partial_i \psi)(2 \partial_i A_k
%     -\partial_\ell A_\ell \delta_{ik} 
%     -\delta_{i\ell} \partial_k A_\ell)
% \end{align}
The last two terms together are a projector perpendicular to $\partial_k \psi$. They are coming from
the Laplacian in the Helmholtz wave equation and the corresponding divergence term. In the plane wave limes (later)
they become the direction $\Vector{k}$ vectors of the plane wave and therefore are responsible for the transverse
property of the waves. In an anisotropic, inhomogenous material the situation is not that easy anymore.

Eq. \eqref{eq:anisoeikonal} could be solved by using
\begin{align}
 \det  \left[\omega^2 \mu_0 \permittivity_{k\ell} - \delta_{k\ell} (\partial_i \psi)^2 + (\partial_k \psi)(\partial_\ell \psi)\right] &= 0\,.
\end{align}
This leads to a highly non-linear PDE for $\psi$ which is given by
\begin{align}
 F\equiv(\omega^2 \mu_0)^3 \det \permittivity + \omega^2 \mu_0 (\partial_i \psi)(\partial_j \psi)\left[\permittivity_{ij} (\omega^2 \mu_0 \text{tr}\permittivity + (\partial_k \psi)^2) + \omega^2 \mu_0 \permittivity_{ik} \permittivity_{jk}\right] &= 0\,.
\end{align}

This equation is numerically tractable with the method of characteristics (treating $\psi$, $\psi_{,k}$ and $x_i$ as variables depending on a curve parameter $s$)
\begin{align}
 \frac{\text{d}\psi_{,i}}{\text{d}s} &= -\frac{\partial F}{\partial x_i} - \frac{\partial F}{\partial \psi} \psi_{,i}\,,\\
 \frac{\text{d}\psi}{\text{d}s} &= \frac{\partial F}{\partial \psi_{,i}} \psi_{,i}\,,\\
 \frac{\text{d}x_i}{\text{d}s} &= \frac{\partial F}{\partial \psi_{,i}}\,,\quad \text{(Assumption to remove 2nd derivatives!)}\,. 
\end{align}
The partial derivative for $x$ means no chain rule for $\psi$ and $\psi_{,i}$, due to the independency of $x$ from these two other variables.
The $\psi$ derivative cancels out because the determinant only depends on gradients of $\psi$.

For the isotropic equation
\begin{align}
 \omega^2 \mu_0 \permittivity(\Vector{x}) - (\psi_{,k})^2&= 0\,,
\end{align}
the ODE system degenerates to
\begin{align}
 \frac{\text{d}\psi_{,i}}{\text{d}s} &= -\omega^2 \mu_0 \permittivity_{,i}\,,\\
 \frac{\text{d}\psi}{\text{d}s} &= -2 (\psi_{,i})^2\,,\\
 \frac{\text{d}x_i}{\text{d}s} &= -2 \psi_{,i}\,. 
\end{align}

\end{document}
