\pdfminorversion=5
\pdfcompresslevel=9
\pdfobjcompresslevel=9

% notes can sometimes not be printed. it is better to filter them through ghostscript before:
%	gs -sDEVICE=pdfwrite -dCompatibilityLevel=1.5 -dNOPAUSE -dQUIET -dBATCH -sOutputFile=blabla_notes.pdf blabla.pdf
% command to print notes with 2 slides on one sheet is
%	lp -o landscape -o number-up=2 -o fitplot -o sides=two-sided-long-edge -P 1,2,3-5,6 -d PRINTER PDF
% or use pdfnup from the pdfjam package
% -P pageranges
\documentclass[
10pt,
%notes=only
]{beamer}

\mode<presentation>
{
  \usetheme{Madrid}
  \setbeamercovered{transparent}
% remove navigation symbols
  \setbeamertemplate{navigation symbols}{}
}

\usepackage{ifthen}

%\setbeameroption{show only notes}
%\setbeameroption{show notes}
\newboolean{notesonly}
\setboolean{notesonly}{false}
\newboolean{notnotesonly}
\ifnotesonly
\setbeameroption{show only notes}
\setboolean{notnotesonly}{false}
\else
\setboolean{notnotesonly}{true}
\fi


%\setbeameroption{show notes on second screen=right}

\usepackage{color}
\usepackage{sfmath}
\usepackage{mathrsfs}
\usepackage{times}
\usepackage{eulervm}
\usepackage[absolute,overlay]{textpos}
\usepackage{wasysym}
\usepackage{cancel}
\setlength{\fboxsep}{0pt}

% unicode
\usepackage[utf8x]{inputenc}
\usepackage[ngerman]{babel}

\usepackage{amsmath,amssymb,amsbsy}

\usepackage{booktabs}

%\usepackage{natbib}

\usepackage{tikz}
%\usepackage{pgfplots}
\usetikzlibrary{decorations}
\usetikzlibrary{shapes,arrows}
\usetikzlibrary{shadows}


\tikzstyle{decision} = [diamond, draw, fill=blue!20, 
    text width=4.5em, text badly centered, node distance=3cm, inner sep=0pt]
\tikzstyle{block} = [rectangle, draw, fill=blue!20, 
    text width=5em, text centered, rounded corners, minimum height=4em]
\tikzstyle{line} = [draw, -latex']
\tikzstyle{cloud} = [draw, ellipse,fill=red!20, node distance=2cm,
    minimum height=2em]



%\input{macros2}
%\input{overlayboxes}
\newcommand{\remark}[1]{{\color{red}{#1}}}
\newcommand{\myref}[1]{{\tiny #1}}

%Backup slides
\newcommand{\backupbegin}{
   \newcounter{framenumberappendix}
   \setcounter{framenumberappendix}{\value{framenumber}}
}
\newcommand{\backupend}{
   \addtocounter{framenumberappendix}{-\value{framenumber}}
   \addtocounter{framenumber}{\value{framenumberappendix}} 
}


\allowdisplaybreaks

\title[Pyrate -- Raytracing in Python]{Pyrate --- Optical Raytracing Based on Python}


\author[M. Esslinger, J. Hartung, T. Heinze, U. Lippmann]{\texorpdfstring{$\bullet$}{}M.~Esslinger, \texorpdfstring{$\bullet$}{}J.~Hartung, T.~Heinze, U.~Lippmann}

%\institute[]{}

\titlegraphic{}

\date[EuroScipy 2017]{EuroScipy 2017}


\logo{}



\begin{document}

\begin{frame}
\titlepage
\end{frame}

\begin{frame}
 \frametitle{Outline}
 \tableofcontents
\end{frame}

\section{Introduction and Motivation}

\begin{frame}
 \frametitle{Technical Optical Design}
  \begin{itemize}
   \item show some pictures of objectives, some applications
   \item break down: point -> point mapping
   \item trend to small, high performance, several optical elements, freeform optics
   \item several trade-offs between design goals and money needed
   \item needs sophisticated methods for finding best solution which is technical realizable and effortable
  \end{itemize}

\end{frame}

\begin{frame}
 \frametitle{History of Pyrate}
 \begin{itemize}
  \item Reason for starting project: commercial programms very expensive 
      $\Rightarrow$ not for hobbyist, using at home; 
      further not free, not flexible enough for upcoming freeform optics
  \item Reason for using python: very flexible, needs no compilation, 
      very fast using numpy and scipy (optics needs several rays traced through system at once);
      plugins possible; strong introspection capabilities
  \item Reason for using Scipy: sophisticated scientific tasks necessary for optics already 
      implemented (linalg (QEV, GEV); optimization; solution of nonlinear equations)
 \end{itemize}

\end{frame}

\section{Optical Systems}

\begin{frame}
 \frametitle{Some Examples}
 \begin{itemize}
  \item from simple to complicated (see demo files)
  \item introduce field points
 \end{itemize}

\end{frame}

\begin{frame}
 \frametitle{The Job of the Optical Designer}
 \begin{itemize}
  \item start system (literature research, patent research, experience)
  \item choose few variables to match the design goals roughly
  \item introduce more DOF, optimize them
  \item find materials which are cheap and still solve the problem
  \item \dots
 \end{itemize}
\end{frame}

\section{Optimization}

\begin{frame}
 \frametitle{Which Quantities should be optimized?}
\end{frame}

\begin{frame}
 \frametitle{The Merit Function}
 \begin{itemize}
  \item show spot criterium, wave front error, several field points
  \item show plots over 2 design variables
  \item clarify that merit function geometry is highly complicated
 \end{itemize}
\end{frame}

\begin{frame}
 \frametitle{The Implementation in Scipy I/II/III}
 \begin{itemize}
  \item optimizable variables
  \item class with optimizable variables (fixed, variable, pickup; possible by introspection possibilies of python)
  \item wrapper for scipy backend (goal: call scipy.minimize(vector, args))
  \item perform optimization
  \item usually more than one optimization steps necessary
 \end{itemize}

\end{frame}


\section{Results and Outlook}

\begin{frame}
 \frametitle{Other parts of Pyrate using Scipy}
  \begin{itemize}
   \item QEV, GEV for solution of the Helmholtz equation
   \item Numerical solution of the intersection equations
   \item \dots
  \end{itemize}

\end{frame}


\begin{frame}
 \frametitle{Some Optical Systems generated with Pyrate}
 \begin{itemize}
  \item maybe live demo; do not show any demo example
 \end{itemize}

\end{frame}

\begin{frame}
 \frametitle{Outlook: GUI}
 \begin{itemize}
  \item FreeCAD: also good python interface, good interface to geometry kernel
  \item good point and click GUI with pyside interface, highly customizable
  \item offers opportunity to couple with optical design to achieve the post design steps
    (mechanical design/FEA, manufacturing/G-Codes, measurement/Point Clouds)
 \end{itemize}

\end{frame}

\begin{frame}
 \begin{center}
  Thank you for your attention!\remark{nice picture}
 \end{center}
\end{frame}



\end{document}
